%%% This file contains definitions of various useful macros and environments %%%
%%% Please add more macros here instead of cluttering other files with them. %%%

%%% Minor tweaks of style

% These macros employ a little dirty trick to convince LaTeX to typeset
% chapter headings sanely, without lots of empty space above them.
% Feel free to ignore.
\makeatletter
\def\@makechapterhead#1{
  {\parindent \z@ \raggedright \normalfont
   \Huge\bfseries \thechapter. #1
   \par\nobreak
   \vskip 20\p@
}}
\def\@makeschapterhead#1{
  {\parindent \z@ \raggedright \normalfont
   \Huge\bfseries #1
   \par\nobreak
   \vskip 20\p@
}}
\makeatother

% This macro defines a chapter, which is not numbered, but is included
% in the table of contents.
\def\chapwithtoc#1{
\chapter*{#1}
\addcontentsline{toc}{chapter}{#1}
}

% Draw black "slugs" whenever a line overflows, so that we can spot it easily.
\overfullrule=1mm

%%% Macros for definitions, theorems, claims, examples, ... (requires amsthm package)

\theoremstyle{plain}
\newtheorem{theorem}{Theorem}
\newtheorem{lemma}[theorem]{Lemma}
\newtheorem{claim}[theorem]{Claim}
\newtheorem{definition}[theorem]{Definition}
\newtheorem{proposition}[theorem]{Proposition}
\newtheorem{example}[theorem]{Example}
\newtheorem{corollary}[theorem]{Corollary}

\theoremstyle{remark}
\newtheorem*{remark}{Remark}


%%% An environment for proofs

\newenvironment{myproof}{
  \par\medskip\noindent
  \textit{Proof}.
}{
\newline
\rightline{$\qedsymbol$}
}

%%% An environment for typesetting of program code and input/output
%%% of programs. (Requires the fancyvrb package -- fancy verbatim.)

\DefineVerbatimEnvironment{code}{Verbatim}{fontsize=\small, frame=single}

%%% The field of all real and natural numbers
\newcommand{\C}{\mathbb{C}}
\newcommand{\K}{\mathbb{K}}
\newcommand{\R}{\mathbb{R}}
\newcommand{\Z}{\mathbb{Z}}
\newcommand{\N}{\mathbb{N}}

%%% Useful operators for statistics and probability
\DeclareMathOperator{\pr}{\textsf{P}}
\DeclareMathOperator{\E}{\textsf{E}\,}
\DeclareMathOperator{\var}{\textrm{var}}
\DeclareMathOperator{\sd}{\textrm{sd}}

%%% Transposition of a vector/matrix
\newcommand{\T}[1]{#1^\top}

%%% Various math goodies
\newcommand{\goto}{\rightarrow}
\newcommand{\gotop}{\stackrel{P}{\longrightarrow}}
\newcommand{\maon}[1]{o(n^{#1})}
\newcommand{\abs}[1]{\left|{#1}\right|}
\newcommand{\isqr}[1]{\frac{1}{\sqrt{#1}}}

%%% Various table goodies
\newcommand{\pulrad}[1]{\raisebox{1.5ex}[0pt]{#1}}
\newcommand{\mc}[1]{\multicolumn{1}{c}{#1}}


%%% Custom commands
\renewcommand{\d}[1]{\, \mathrm{d}#1 \,}
\renewcommand{\i}{\mathrm{i}}
\newcommand{\Cont}{C}
\newcommand{\Hilb}{\mathcal{H}}
\newcommand{\Schwa}{\mathcal{S}}
\newcommand{\Domain}{\mathrm{D}}
\newcommand{\Fourier}{\mathcal{F}}
\newcommand{\Nbhood}{\mathcal{U}}
\newcommand{\Hf}{\mathscr{H}}
\newcommand{\Af}{\mathscr{A}}
\newcommand{\ve}{\varepsilon}
\newcommand{\hypF}{\vphantom{\mathrm{F}}_1\hspace{-0.7pt}\mathrm{F}_{\!1}}

\newcommand{\Sp}{\sigma}
\newcommand{\SpP}{\sigma_{\mathrm{p}}}
\newcommand{\SpAc}{\sigma_{\mathrm{ac}}}
\newcommand{\SpSc}{\sigma_{\mathrm{sc}}}
\newcommand{\SpD}{\sigma_{\mathrm{disc}}}
\newcommand{\SpE}{\sigma_{\mathrm{ess}}}


\def\ph{\phantom}
\def\vph{\vphantom}
\def\hph{\hphantom}
\def\rzw{\mathrlap}
\def\lzw{\mathllap}
\def\czw{\mathclap}
\newcommand*{\mask}[2]{\mathord{\makebox[\widthof{\(#1\)}]{\(#2\)}}}
\newcommand{\scalemath}[2]{\scalebox{#1}{\mbox{\ensuremath{\displaystyle #2}}}}


\newcommand{\coloneqq}{:=}
\newcommand{\eqqcolon}{=:}

\newcommand{\digamma}{\Psi} % haha dobrý

\newcommand\numberthis{\addtocounter{equation}{1}\tag{\theequation}}

%%%%%%%%%%%%%%%%%%%%%%%%%%%%%%%%%%%%%%%%%%%%%%%%%%%%%%%%%%%%%%%
% Mejroslav

%%% My own symbols
\newcommand{\vc}[1]{\boldsymbol{#1}} % vektor
\newcommand{\mat}[1]{\mathbf{#1}} % matice

\newcommand{\norm}[1]{\left \Vert #1 \right \Vert} % norma vektoru
\newcommand{\set}[1]{ \left \lbrace #1 \right \rbrace} % množina
\newcommand{\const}{\mathrm{const}} % konstanta

%Značení derivací a integrálů
\newcommand{\der}[2]{\frac{\mathrm{d}#1}{\mathrm{d}#2}} % obyčejná derivace
\newcommand{\pder}[2]{\frac{\partial #1}{\partial #2}} % parciální derivace
\newcommand{\ppder}[3]{\frac{\partial^2 #1}{\partial #2 \partial #3}} % parciální derivace
\newcommand{\fder}[2]{\frac{\delta #1}{\delta #2}} % funkcionální derivace
\newcommand{\ffder}[3]{\frac{\delta^2 #1}{\delta #2 \delta #3}} % funkcionální derivace
\newcommand{\D}{\mathrm{d} } % integrační znamení
\newcommand{\DD}{\mathrm{D} } % velké déčko


% Značení posloupností, limit a sum
\newcommand{\sequence}[2]{ \left \lbrace #1 \right \rbrace_{#2=1}^\infty} % posloupnost
\newcommand{\sumnorm}[1]{\sum_{#1}^\infty} 
\newcommand{\limplus}[1]{\lim_{#1 \rightarrow + \infty}}
\newcommand{\limminus}[1]{\lim_{#1 \rightarrow - \infty}}


%Značení distribucí
\newcommand{\dual}[2]{\left \langle #1 ,#2 \right \rangle} % dualita
\newcommand{\test}{\mathcal{D}} % prostor testovacích funkcí
\newcommand{\dis}{\mathcal{D'}} % prostor distribucí

\newcommand{\weakstar}{\rightharpoonup^*} % slabá* konvergence 
\newcommand{\supp}{\mathrm{supp}\,} % nosič funkcí

\newcommand{\Dom}{\mathrm{Dom}} % domain