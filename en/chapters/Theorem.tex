\chapter{The connection between Riemannian geometry and Hamiltonian mechanics}

In this chapter we shall prove the theorem of the connection between the hydrodynamic Poisson bracket and the metrics found in the homogeneous system of hydrodynamic type. The original result was given by Dubrovin and Novikov and can be found in \cite{Dubrovin-Novikov}. The proof itself and the preceding lemmas are based on a considerable amount of technical computations, which we shall clarify in detail in order to offer the reader a comprehensive understanding of the chapter in question.





\section{Systems of hydrodynamic type}

\begin{definition}[Homogeneous system of hydrodynamic type]
    A homogeneous system of hydrodynamic type is an equation of the form
    \begin{align}
        \label{eq:h-system}
        \pder{u^i}{t} = f_j^{i \alpha} (u) \pder{u^j}{x^\alpha} 
    \end{align}
    where $u^i(t,x^\alpha)$ are unknown functions, $i = 1, \dots , N$, $\alpha = 1, \dots , d$.
\end{definition}

Sometimes we refer to the number of equations $N$ as the dimension of the hydrodynamical phase space and to the number of spatial variables $d$ as the dimension of the configuration space.

Before we go any further, let us present a few examples.

\begin{example}
    The energy is given by the functional
    \begin{align}
        E = \int e(\rho, \vc m, s, \vc F) \, \D \vc x
    \end{align}
    The evolution equations in the Eulerian frame are
    \begin{align}
        \partial_t \rho =& - \partial_i (\rho E_{m_i}) 
        \\
        \partial_t m_i =& - \partial_j(m_i E_{m_j}) - \rho \partial_i E_\rho - m_j \partial_i E_{m_j} - m_j \partial_i E_{m_j} 
        \\
        -& s \partial_i E_s - F^j_J \partial_i E_{F^j_J} + \partial_j(F^j_I E_{F^i_I} + F^i_O E_{F^i_I}) 
        \\
        \partial_t s =& - \partial_i (sE_{m_i})
        \\
        \partial_t F^i_I =& - E_{m_k} \partial_k F^i_I + F^j_I \partial_j E_{m_i} \:.
    \end{align}
    Total energy density $e$ can be prescribed as
    \begin{align}
        e = \frac{\vc m^2}{2 \rho} + \epsilon(\rho,s , \vc F) \:,
    \end{align}
    where $\epsilon$ is the elastic and internal energy. In particular, $E_{\vc m} = \frac{\vc m}{\rho} = \vc v$ becomes the velocity. The evolution equation for the deformation gradient then gets the explicit form
    \begin{align}
        \partial_t \vc F = -(\vc v \cdot \nabla) \vc F + \nabla \vc V \cdot \vc F \:,
    \end{align}
    which is the usual evolution equation for $\vc F$ in the Eulerian frame.
\end{example}

It was Riemann who noticed that the functions $f^{i \alpha}_j$ in \eqref{eq:h-system} are in fact tensors [CITACE]\cite{Dubrovin-Novikov}.

\begin{proposition}[Transformation of $f^{i \alpha}_j$] 
    \label{prop:transformace-A} 
    Under smooth change of variables $u^i \mapsto v^a$ of the form
    \begin{align}
        u^i = u^i(v^1,\cdots, v^N)
    \end{align}
    the functions $f^{i\alpha}_{j}$ transform for each $\alpha$ according to the tensor law
    \begin{align}
        f^{a \alpha}_{b}(v) = \pder{v^a}{u^i} \pder{u^j}{v^b} f^{i \alpha}_j (u) \:.
    \end{align}
\end{proposition}
\begin{proof}
    Using a direct calculation we obtain
    \begin{align}
        \pder{u^i}{t}(v) = \pder{u^i}{v^a} \pder{v^a}{t} \:, \\
        \pder{u^j}{x^\alpha}(v) = \pder{u^j}{v^b} \pder{v^b}{x^\alpha}\:,
    \end{align}
    so
    \begin{align}
        \pder{u^i}{v^a} \pder{v^a}{t} = \pder{u^j}{v^b} f^{i \alpha}_j (v(u)) \pder{v^b}{x^\alpha}
    \end{align}
    and therefore
    \begin{align}
        \pder{v^a}{t} = \underbrace{\pder{v^a}{u^i} \pder{u^j}{v^b} f^{i \alpha}_j(v(u))}_{f^{a \alpha}_b(v)} \pder{v^b}{x^\alpha} \:.
    \end{align}
\end{proof}

According to the previous lemma, if we denote the space of points $(t,x^\alpha) \subset \R \times \R^N$ as some manifold $M$, then the unknown variables $u^i(t,x^\alpha)$ can be seen as the local coordinates of that manifold. The functions $f^{i \alpha}_j$ are transformed as tensors of the type (1,1) in the $i$ and $j$ indices.

Our next purpose is to gain a richer geometry of that space $M$. If the functions $f^{i \alpha}_j$ have a special form, we can introduce a geometry of the Poisson brackets.

\begin{definition}
    \begin{enumerate}
        \item A Poisson bracket of the hydrodynamic type of two functionals $I_1$ a $I_2$ is defined as
        \begin{align}
            \set{I_1,I_2} = \int \D x 
            \left[ \frac{\delta I_1}{\delta u^q(x)} A^{qp} \frac{\delta I_2}{\delta u^p(x)} \right]
        \end{align}
        where
        \begin{align}
            A = (A^{qp}) = \left( g^{qp \alpha}(u) \der{}{x^\alpha} + b^{qp\alpha}_s(u) \pder{u^s}{x^\alpha} \right) \:.
        \end{align}
        where $g^{ij \alpha}$ and $b_k^{ij \alpha}$ are certain functions, $i,j,k = 1, \dots, N$ and $\alpha = 1, \dots, d $.

        In particular
        \begin{align}
            \set{u^i(x), u^j(y)} = g^{ij \alpha} [u(x)] \delta_\alpha(x-y) + b_k^{ij \alpha} [u(x)] u_\alpha^k(x) \delta(x-y) \:.
        \end{align}
        
        \item The functional of hydrodynamic type is a functional of the form
        \begin{align}
            H[u] = \int_{\R^d} h(u(x)) \D^d x \:,
        \end{align}
        where $h$ is independent of $u_\alpha$, $u_{\alpha \beta}$. The function $h$ is called a hamiltonian density.

        \item The hamiltonian system of hydrodynamic type is a system of the form
        \begin{align}
            u^i_t (x) = 
            \left[ g^{ij \alpha} [u(x)] \ppder{h(u)}{u^j}{u^k} + b_k^{ij \alpha} [u(x)] \pder{h(u)}{u^j} \right] u_\alpha^k(x) \:,
        \end{align}
        where $H[u]$ is the functional of the hydrodynamic type and $h$ the hamiltonian density.
    \end{enumerate}
\end{definition}

\begin{proposition}
    \begin{align}
        u^i_t(x) = \set{u^i,H} \:.
    \end{align}
\end{proposition}
\begin{proof}
    \begin{align}
        \set{u^i(x),H} = \int \D x' \D y' \fder{u^i(x)}{u^p(x')} \left[ g^{pq}(u(x')) \der{}{x'} + b^{pq}_m \pder{u^m}{x} \right] \fder{H}{u^q(y')} \:.
    \end{align}
    \begin{align}
        \fder{u^i(x)}{u^p(x')} = \delta^i_p \delta(x-x')
    \end{align}
    \begin{align}
        H[u] = \int h(u(y')) \D y' \:, \quad \fder{H}{u^q(y')} = \int \pder{h}{u^q}(y) \delta(y-y')
    \end{align}
    \begin{align}
        \set{u^i(x),H} =& \int \D y' \left[ g^{iq}(u(x)) \der{}{x} + b^{iq}_m(u(x)) \pder{u^m}{x} \right] \pder{h}{u^q}(y) \delta(y-y') 
        \\ =& g^{iq}(u(x)) \pder{^2 h}{u^k \partial u^q} \pder{u^k}{x} + b^{iq}_m(u(x)) \pder{u^m}{x} \pder{h}{u^q}
        \\ =& \left[ g^{ij}(u(x)) \pder{^2 h}{u^j \partial u^k}  + b^{ij}_k(u(x))  \pder{h}{u^j} \right] \pder{u^k}{x} \:.
    \end{align}
\end{proof}
\todo[inline]{Dokončit a přidělat $\alpha$. kde se předělá $\der{}{x}$ na $\der{}{y}$?}

Note that we have yet to show that the Poisson bracket of the hydrodynamic type defined above is indeed a Poisson bracket in the classical sense. We must verify that it satisfies the three properties: skew-symmetry, Leibniz rule and Jacobi identity. In fact, this will be the key theorem of this chapter.

\section{One-dimensional case}

In this section we first consider the one-dimensional case in which the unknown functions $u^i$ depend on only two variables $(t,x)$. The hamiltonian system is then
\begin{align}
    \pder{u^i}{t} (t,x) =&
    \left[ g^{ij} [u(t,x)] \ppder{h(u)}{u^j}{u^k} + b_k^{ij} [u(t,x)] \pder{h(u)}{u^j} \right] \pder{u^k}{x}(t,x) 
\end{align}
and the Poisson bracket of hydrodynamic type is of the form
\begin{align}
    \set{I_1,I_2} = \int \D x
    \left[ \frac{\delta I_1}{\delta u^q(x)} A^{qp} \frac{\delta I_2}{\delta u^p(x)} \right] \:,\quad \text{where } A^{qp} = \left( g^{qp}(u) \der{}{x} + b^{qp}_s(u) \pder{u^s}{x} \right) \:.
\end{align}
In particular
\begin{align}
    \label{eq:Poisson1D}
    \set{u^i(x), u^j(y)} =& g^{ij} [u(t,x)] \partial_x \delta(x-y) + b_k^{ij} [u(t,x)] \pder{u^k}{x} \delta(x-y) \:. 
\end{align}

We say that the Poisson bracket is non-degenerate if $\det(g^{ij}) \neq 0$. We will show later that the non-degeneracy is invariant under the transformation of the coordinates.

\begin{definition}
    Let $(g^{ij}(u))$ be a non-degenerate matrix. We define functions $\Gamma^{i}_{jk}(u)$ by
        \begin{align}
            b_k^{ij}(u) = -g^{is}(u) \Gamma^j_{sk}(u) \:, \quad i,j,k,s=1,\dots,N \:. \label{eq:definice-Gamma}
        \end{align}
\end{definition}

The choice of denotation $g^{ij}$ and $\Gamma^{i}_{jk}$, respectively, has its purpose. We will show soon that these are in fact the metric and afinne coeficients, respectively, on $M$.

\begin{proposition}[Leibniz rule for Poisson bracket]
    The Poisson brackets of the hydrodynamic type satisfy the Leibniz rule
    \begin{align}
        \label{eq:transformace-zavorka}
        \set{v^p(u^i(x)),v^q(u^j(y))} = \pder{v^p}{u^i}(x) \pder{v^q}{u^j}(y) \set{u^i(x),u^j(y)} \:. 
    \end{align}
\end{proposition}
\begin{proof}
    The coordinates are written as functionals in the form
    \begin{align}
        v^p(u^i(x)) = \int \D x' v^p(u^i(x')) \delta (x-x') \:.
    \end{align}
    Applying this to the definition of the Poisson bracket \label{eq:Poisson1D}, we have
    \begin{align}
        \set{v^p(u^i(x)),v^q(u^j(y))} 
        =& \int \D x' \int \D y' \underbrace{\fder{v^p}{u^i(x')} }_{\pder{v^p}{u^i}(x') \delta(x-x')} \set{u^i(x'),u^j(y')} \underbrace{\fder{v^q}{u^j(y')} }_{\pder{v^q}{u^j}(y') \delta(y-y')}
        = \\ =& \pder{v^p}{u^i}(x) \pder{v^q}{u^j}(y) \set{u^i(x),u^j(y)} \:.
    \end{align}
\end{proof}


\begin{proposition}[Transformation of $g$ and $\Gamma$] \label{prop:transoformace}
    Let $u^i \mapsto v^a$ be the smooth change of variables. Then
    \begin{enumerate}
        \item The coeficients $g^{ij}$ transform as the (2,0) tensors, i.e. 
        \begin{align}
            g^{pq}(v) = \pder{v^p}{u^i} \pder{v^q}{u^j} g^{ij}[u(v)] \:, \quad p, q = 1, \dots, N \:. \label{eq:transformace-metrika}
        \end{align}
        \item The coeficients $\Gamma^i_{jk}$ transform like the afinne connection symbols, i.e.
        \begin{align}
            \Gamma^p_{qr} (v) = \pder{v^p}{u^i} \pder{u^j}{v^q} \pder{u^k}{v^r} \Gamma^i_{jk}(u) + \pder{v^p}{u^i} \ppder{u^i}{v^q}{v^r} \:. \label{eq:transformace-konexe}
        \end{align}
    \end{enumerate}
\end{proposition}

\begin{proof}
    \begin{enumerate}
        \item Let us apply the transformation equation \eqref{eq:transformace-zavorka} to the definition of the Poisson bracket \eqref{eq:Poisson1D}. We get 
        \begin{align}
            g^{pq}[v(u(x))] \partial_x \delta(x-y) + b^{pq}_s[v(u(x))] \pder{v^s}{u^k}(x) \pder{u^k}{x} \delta(x-y) \nonumber
            \\ = \pder{v^p}{u^i}(x) \pder{v^q}{u^j}(y) \left[ g^{ij}[u(x)] \partial_x \delta(x-y) + b^{ij}_k[u(x)] \pder{u^k}{x} \delta(x-y) \right] \:. \label{eq:dosad}
        \end{align}
        From now we denote for readibility
        \begin{align}
            T^a_k(x) := \pder{v^a}{u^k}(x) \:.
        \end{align}
        On the right-hand side of \eqref{eq:dosad} we apply Dirac identity (Lemma \vref{lemma:delta}) on $T^q_j(y) \partial_x \delta(x-y)$. All functions are then of variable $x$, so from now we won't write it explicitely. We have
        \begin{align}
            g^{pq}(v) \partial_x \delta(x-y) + b^{pq}_n(v) T^n_j \pder{u^j}{x} \delta(x-y) \nonumber
            \\ = T^p_i T^q_j g^{ij}(u) \partial_x \delta(x-y) 
            + \left( \ppder{v^q}{u^j}{u^k} + T^p_i T^q_j b^{ij}_k \right) \pder{u^k}{x} \delta(x-y) \:. \label{eq:porovnani}
        \end{align}
        The terms with $\partial_x \delta$ must be equal to each other, as must $\delta$ terms. From $\partial_x \delta$ terms we get
        \begin{align}
            g^{pq}(v) = T^p_i T^q_j g^{ij}(u) \:,
        \end{align}
        which is the transformation relation for $g$.

        \item From $\delta$ terms in \eqref{eq:porovnani}, we get
        \begin{align}
            b^{pq}_n(v) T^n_k =
            b^{ij}_k(u) T^p_i T^q_j + g^{ij}(u) T^p_i \ppder{v^q}{u^i}{u^k}
        \end{align}
        and after \eqref{eq:definice-Gamma}
        \begin{align}
            - g^{ps}(v) \Gamma^{q}_{sn}(v) T^n_k =
            - g^{il}(u) \Gamma^j_{lk}(u) T^p_i T^q_j + g^{ij}(u) T^p_i \ppder{v^q}{u^i}{u^k} \:.
        \end{align}
        On the left-hand side we use the transformation of $g$ \eqref{eq:transformace-metrika}
        \begin{align}
            -g^{ij}(u) \Gamma^q_{sm}(v) T^p_i T^s_j T^m_k = - g^{ij}(u) \Gamma^l_{jk} T^p_i T^q_l + g^{ij}(u) \ppder{v^q}{u^i}{u^k} T^p_i 
        \end{align}
        and after vectoring out $g^{ij}(u) T^p_i$
        \begin{align}
            \Gamma^l_{jk}(u) T^q_l = \Gamma^q_{sm}(v) T^s_j T^m_k + \ppder{v^q}{u^i}{u^k} \:.
        \end{align}
        All that remains is to multiply this equation with $(T^{-1})^a_q$ and we get the final transformation relation for $\Gamma$
        \begin{align}
            \Gamma^a_{jk}(u) = \Gamma^q_{sm}(v) \pder{v^s}{u^j} \pder{v^m}{u^k} \pder{u^a}{v^q} + \ppder{v^q}{u^i}{u^k} \pder{u^a}{v^q} \:.
        \end{align}
    \end{enumerate}
\end{proof}

Let us summarize the results. We have defined Poisson bracket in the way that it satisfies the Leibniz rule. It remains to investigate the skew-symmetry and Jacobi identity. We have also defined $g^{pq}$ and $\Gamma^p_{qr}$, which transform like tensors and Christoffel symbols, respectively. It remains to show that $g^{pq}$ is indeed symmetric and $\Gamma$ is the Levi-Civita connection, i.e. the corresponding covariant derivative annuls the metric. We are now ready to prove the key theorem of this chapter which shows that these properties are indeed equivalent. Moreover, the corresponding covariant derivative has zero curvature, which indicates that the phase space $M$ is in fact Euclidean space.

\begin{theorem}[Dubrovin, Novikov] \label{theorem1:ekvivalence}
    Let $\det(g^{ij}) \neq 0$. Then \eqref{eq:Poisson1D} defines a Poisson bracket with skew-symmetry, Leibniz rule and Jacobi identity if and only if these three conditions are satisfied:
    \begin{enumerate}
        \item $g^{ij} = g^{ji}$, that is, $g^{ij}$ is a metric on the phase space $M$.
        \item $\Gamma^{i}_{jk} = \Gamma^i_{kj}$, that is, $\Gamma^{i}_{jk}$ are affine coeficients, such that the corresponding covariant derivative has zero torsion and anihilate the metric $g^{ij}$.
        \item The covariant derivative corresponding to $\Gamma^{i}_{jk}$ has zero curvature.
    \end{enumerate}
\end{theorem}

\begin{corollary}
    There are local coordinates $w^i = w^i(u^1, \dots, u^N)$, $i = 1, \dots, N$ such that
    \begin{align}
        g^{ij}(w) = \tilde g^{ij} = \const \:, \quad \Gamma^i_{jk} (w) = 0 \:.
    \end{align}
    In these coordinates is Poisson bracket reduced to the constant one
    \begin{align}
        \set{w^i(x),w^j(y)} = \tilde{g}^{ij} \partial_x \delta(x-y) \:.
    \end{align}
    The only local invariant of the Poisson bracket is the signature of the metric $g^{ij}(u)$.
    \todo[inline]{Co se tímhle myslí? Pořád nemám rozmyšleno.}
\end{corollary}

\begin{proof}[Proof of theorem \ref{theorem1:ekvivalence}]
    The proof is divided into three steps.
    \\
    \textbf{Step 1: Skew-symmetry of Poisson bracket implies the symmetry of $g^{ij}$ and compatibility with $\Gamma^i_{jk}$.}

    Let the Poisson bracket be skew-symmetric, i.e.
    \begin{align}
        \set{u^i(x),u^j(y)} + \set{u^j(y),u^i(x)} = 0 \:.
    \end{align}
    The second bracket reads as
    \begin{align}
        \set{u^j(y),u^i(x)} =  g^{ji}[u(y)] \partial_x \delta(y-x) + b^{ji}_k[u(y)] \pder{u^k}{y} \delta(y-x) \:.
    \end{align}
    Let us apply Dirac identites (Lemma \vref{lemma:delta}) on $g^{ji}[u(y)]\partial_x \delta(y-x)$. We get
    \begin{align}
        \set{u^j(y),u^i(x)} =& \nonumber \\ 
        = - g^{ji} [u(x)] \partial_x \delta(x-y) -& \pder{g^{ji}}{u^k}[u(x)] \pder{u^k}{x} \delta(x-y) + b^{ji}_k[u(x)] \pder{u^k}{x} \delta(x-y)
        \:.
    \end{align}
    The skew-symmetry is then
    \begin{align}
        0 = \set{u^i(x),u^j(y)} + \set{u^j(y),u^i(x)} = \nonumber
        \\= \left[ g^{ij}[u(x)]  - g^{ji}[u(x)] \right] \partial_x \delta(x-y) + \left[ b^{ij}_k - \pder{g^{ji}}{u^k}  + b^{ji}_k  \right] \pder{u^k}{x} \delta(x-y) \:.
    \end{align}
    The right-hand side is equal to zero if and only if
    \begin{align}
        g^{ij}(u) = g^{ji}(u) \:, \label{eq:Th.Novikov-symetrie} \\
        \pder{g^{ij}}{u^k}(u) = b^{ij}_k(u) + b^{ji}_k(u)  \:. \label{eq:Th.Novikov-konexe}
    \end{align}
    Equation \eqref{eq:Th.Novikov-symetrie} says that $g^{ij}$ is a symmetric tensor. By assumption $\det g^{ij}$ is non-degenerate, so it defines a metric structure on $M$. Equation \eqref{eq:Th.Novikov-konexe} is equivalent to
    \begin{align}
        \pder{g^{ij}}{u^k} + g^{is} \Gamma_{sk}^j + g^{sj} \Gamma_{sk}^i = 0 \:.
    \end{align}
    If $\nabla$ is the covariant derivative corresponding to $\Gamma^i_{jk}$, then $\nabla_k g^{ij} = 0$. So $\Gamma_{ij}^k$ are affine connection coeficients and are compatible with $g^{ij}$.

    \textbf{Step 2: Jacobi identity is equivalent to zero torsion and curvature.} 
    
    Let us assume that Jacobi identity holds. We denote
    \begin{align}
        \mathcal J^{ijk}(x,y,z) = J^{ijk}(x,y,z)+ J^{jki}(y,z,x)+ J^{kij}(z,x,y) = 0
    \end{align}
    where
    \begin{align}
        J^{ijk}(x,y,z) = \set{u^k(z) , \set{u^i(x),u^j(y)}} \:.
    \end{align}
    \textbf{[TADY TO JEŠTĚ NENÍ HOTOVÉ.]}
    From now on we shall use the abbreviated notation
    \begin{align}
        g^{ij}_{,m} = \pder{g^{ij}}{u^m} \:, \quad b^{ij}_{k,m} = \pder{b^{ij}_k}{u^m} \:, \quad u^k_x = \pder{u^k}{x}  \:.
    \end{align}

    \begin{align*}
        J^{ijk}(x,y,z) =
        \int \D \xi \fder{u^k(z)}{u^n(\xi)}
        \left[ g^{mn}(\xi) \partial_\xi + b^{mn}_p \partial_\xi u^p \cdot \right] \fder{\set{u^i(x),u^j(y)}}{u^m(\xi)}  \:.
    \end{align*}
    The functional derivatives are
    \begin{align}
        \fder{u^k(z)}{u^n(\xi)} =& \delta^k_n \delta(z-\xi) \:, \\
        \fder{\set{u^i(x),u^j(y)}}{u^m(\xi)} =& \fder{}{u^m(\xi)} \left[ g^{ij}(x) \partial_x \delta(x-y) + b^{ij}_k(x) \partial_x u^k \delta(x-y)\right] = 
        \nonumber
        \\ =& \left[ g^{ij}_{,m} \partial_x \delta(x-y) + b^{ij}_{,m}(x) u^k_x \delta(x-y) \right] \delta (x-\xi) \:.
    \end{align}
    Thus we can eliminate the integration in $\xi$ and get rid of the integral.
    \begin{align*}
        & J^{ijk}(x,y,z) = \int \D \xi \delta^k_n \delta(z-\xi) \left[ g^{mn}(\xi) \partial_\xi + b^{mn}_p \partial_\xi u^p \cdot \right] \\ 
        & \left[ g^{ij}_{,m}(x) \partial_x \delta(x-y) \delta(x-\xi) + b^{ij}_{k,m}(x) u^k_x \delta(x-y) \delta(x-\xi) \right] 
        \\
        =& \left[ g^{mk}(\xi) \partial_z + b^{mk}_p \partial_z u^p \cdot \right] \left[ g^{ij}_{,m}(x) \partial_x \delta(x-y) \delta(x-\xi) + b^{ij}_{k,m}(x) u^k_x \delta(x-y) \delta(x-\xi) \right]
        \\
        =& \left[ g^{ij}_{,m} g^{mk} \partial_x u^m \partial_x \delta(x-y) + b^{ij}_{k,m} g^{mk} u^k_x \delta(x-y) \right] \partial_x \delta(z-x) + 
        \\
        +& \left[ g^{ij}_{,m} b^{mk}_p u^p_x \partial_x \delta(x-y) + b^{ij}_{k,m} b^{mk}_p u_x^p  u_x^k \delta(x-y) \right] \delta(z-x) \:.
    \end{align*}

    %\begin{align*}
    %    & J^{ijk}(x,y,z) = \\
    %    & \left[ g^{ij}_{,m}(x) \partial_x \delta(x-y) + b^{ij}_{k,m}(x) u^k_x \delta(x-y) \right] \left[ g^{mk}(x) \partial_x \delta(z-x) + b^{mk}_p(x) u^p_x \delta(z-x)\right] \:.
    %    \\ 
    %    =& \left[ g^{ij}_{,m} g^{mk} \partial_x u^m \partial_x \delta(x-y) + b^{ij}_{k,m} g^{mk} u^k_x \delta(x-y) \right] \partial_x \delta(z-x) + 
    %    \\
    %    +& \left[ g^{ij}_{,m} b^{mk}_p u^p_x \partial_x \delta(x-y) + b^{ij}_{k,m} b^{mk}_p u_x^p  u_x^k \delta(x-y) \right] \delta(z-x) \:.
    %\end{align*}


    We get four terms which are $\delta$ distributions and their derivatives.
    \begin{align*}
        J^{ijk}(x,y,z) =
        g^{ij}_{,m} g^{mk} \partial_x \delta(x-y) \partial_x \delta(z-x) 
        + b^{ij}_{k,m} g^{mk} u^k_x \delta(x-y) \partial_x \delta(z-x) 
        \\+ g^{ij}_m b^{mk}_p u^p_x  \partial_x \delta(x-y) \delta(z-x) 
        + b^{ij}_{k,m} b^{mk}_p u^p_x u^k_x \delta(x-y) \delta(z-x) \:.
    \end{align*}
    Jednotlivé členy označíme \begin{align*}
        J^{ijk}(x,y,z) = E(x) \partial_x \delta(x-y) \partial_x \delta(z-x) + F(x) \delta (x-y) \partial_x \delta(z-x) 
        \\+ G(x) \partial_x \delta(x-y) \delta(z-x) + H(x) \delta(x-y) \delta(z-x) \:.
    \end{align*}
    Díky symetrii, resp. antisymetrii $\delta$ funkcí
    \begin{align}
        \delta (x-y) = \delta(y-x) \:, \quad \partial_x \delta(x-y) = - \partial_y \delta(y-x)
    \end{align}
    Lze členy předělat na tvar
    \begin{align}
        J^{ijk}(x,y,z) = E(x) \partial_y \delta(y-x) \partial_z \delta(x-z) - F(x) \delta(y-x) \partial_z \delta(x-z) \\
        - G(x) \partial_y \delta(y-x) \delta(x-z) + H(x) \delta(x-y)  \delta(x-z)
    \end{align}


    \textbf{[ODTEĎ UŽ JE TO DODĚLANÉ.]}
    One of these equations reads as
    \begin{align}
        A_{02}^{ijk} = b^{ij}_s g^{sk} - b^{kj}_s g^{si} = -g^{ip} \Gamma^{j}_{ps} g^{sk} + g^{km} \Gamma^{j}_{ms} g^{si} =0 \:,
    \end{align}
    therefore, after multiplying with $g_{ai}g_{bk}$
    \begin{align}
        \Gamma^j_{ab} - \Gamma^j_{ba} = 0 \:.
    \end{align}
    So the corresponding covariant derivative has zero torsion.

    Another equation reads as
    \begin{align}
        A_{00}^{ijk} = B^{ijk}_p(u) u^p_{xx} + C^{ijk}_{pq}(u) u^p_x u^q_x = 0 \:,
    \end{align}
    where
    \begin{align}
        B^{ijk}_p 
        = (b^{jk}_{s,p} - b^{jk}_{p,s}) g^{si} + b_s^{ij} b_p^{sk} - b_s^{ik} b_p^{sj} \:.
    \end{align}
    The function $B^{ijk}_p$ beeing zero implies the zero curvature, which follows from
    \begin{align}
        B^{ijk}_p  
        =& -g^{is} \left( g^{jm}_{,p} \Gamma^k_{ms} + g^{jm} \Gamma^k_{ms,p} - g^{jm}_{,s} \Gamma^k_{mp} - g^{jm} \Gamma^k_{mp,s} \right) 
        \\ +& g^{im} \Gamma^j_{ms} g^{sn} \Gamma^{k}_{np} - g^{im} \Gamma^k_{ms} g^{sn} \Gamma^j_{np}
        = \\ =& g^{is} g^{jn} \left( \Gamma^m_{np} \Gamma^k_{ms} - \Gamma^{m}_{ns} \Gamma^k_{mp} + \Gamma^{k}_{mp,s} -  \Gamma^{k}_{ms,p} \right) 
        \\ +& g^{is} g^{nm} \left( \Gamma^j_{np} \Gamma^{k}_{ms} - \Gamma^j_{ns} \Gamma^k_{mp} \right) + g^{im} g^{sn} \left( \Gamma^j_{ms} \Gamma^k_{np} - \Gamma^k_{ms} \Gamma^j_{np} \right)
        = \\ =& -g^{is} g^{jn} \left( \Gamma^{m}_{ns} \Gamma^k_{mp}  - \Gamma^m_{np} \Gamma^k_{ms} + \Gamma^{k}_{ms,p} - \Gamma^{k}_{mp,s} \right)
        = \\ =& -g^{is} g^{jn} R^k_{nps} \:.
    \end{align}
    So far we proved that if hydrodynamical Poisson bracket satisfies skew-symmetry and Jacobi identity, the conditions 1-3 are satisfied.

    \textbf{Step 3:}

    If $\nabla$ has zero torsion and curvature, there are coordinates $w^i=w^i(u^1,\dots,u^N)$ for $i=1,\dots,N$ such that $g^{ij} = \const$ a $b^{ij}_k = 0$. In these coordinates
    \begin{align}
        \set{w^i(x),w^j(y)} = \tilde{g}^{ij} \partial_x \delta(x-y) \:.
    \end{align}
    Jacobi identity, skew-symmetry and Leibniz rule are for that bracket satisfied trivially. Therefore the three conditions are also sufficient. That finishes the proof.
\end{proof}

\subsection{Kdy je hydrodynamický systém hamiltonovský?}

Nyní můžeme explicitně přepsat podmínky, při kterých je obecný systém hydrodynamických rovnic hamiltonovský vůči nějaké nedegenerované Poissonově závorce. Nejprve si povšimneme, že funkce $f^i_j(u)$ lze přepsat pomocí Laplaceova–Beltramiho operátoru.

\begin{proposition}[O zápisu pomocí Laplaceova–Beltramiho operátoru]
    Mějme hamiltonovský systém rovnic
    \begin{align}
        \pder{u^i}{t} = f^i_k(u) \pder{u^k}{x} \:, \quad f^i_k(u) = g^{ij}(u) \ppder{h}{u^j}{u^k} - g^{is}(u) \Gamma^j_{sk}(u) \pder{h}{u^j} \:. \label{eq:hamiltonovsky-system}
    \end{align}
    Pak lze psát 
    \begin{align}
        f^i_k(u) = \nabla^i \nabla_k h(u) \:, 
    \end{align}
    kde $\nabla_j$ je Levi-Civitova kovariantní derivace metriky $g_{ij}$ a $\nabla^i = g^{is} \nabla_s$. 
\end{proposition}
\begin{proof}
    Přímým dosazením
    \begin{align}
        \nabla_k h(u) =& \pder{h}{u^k} \:, \\
        \nabla^i \nabla_k h(u) =& g^{is} \nabla_s \pder{h}{u^k} = g^{is} \left( \ppder{h}{u^s}{u^k} - \Gamma^j_{sk} \pder{h}{u^j} \right) = f^i_k(u) \:.
    \end{align}
\end{proof}

Pomocí tohoto zápisu snadno můžeme sepsat postačující podmínky hamiltonovskosti.
Důkaz plyne ihned z teorému \vref{theorem1:ekvivalence}.

\begin{theorem}[Postačující podmínka hamitonovskosti systému]
    Systém $u^i_t = f^i_j(u) u^j_x$ je hamiltonovský právě tehdy, když existuje nedegenerovaná metrika $g^{ij}(u)$, jejíž afinní konexe má nulovou křivost a splňuje
    \begin{align}
        g_{ij} f^k_j = g_{jk} f^k_i \:, \label{eq:gf=gf} \\
        \nabla_i f^k_j = \nabla_j f^k_i \label{eq:nablaf} \:.
    \end{align}
\end{theorem}

Speciálně vztah \eqref{eq:nablaf} říká, že $\nabla_i$ má nulovou torzi.


\subsection{Rekonstrukce metriky z $f^i_j(u)$?}

Máme-li zadaný hamiltonovský systém s maticí $f^i_j(u)$, lze metriku $g^{ij}(u)$ zkonstruovat jednoznačně? Tuto otázku nyní vyřešíme pro $N \geq 3$. 

Označme $\lambda_\alpha$ vlastní čísla matice $f^i_j(u)$. (Mohou být komplexní.) Předpokládejme, že jsou navzájem různá. Označme odpovídající bázi vlastních vektorů $e_\alpha(u)$. Definujme koeficient $c^\gamma_{\alpha \beta}(u)$ vztahem
\begin{align}
    [e_\alpha,e_\beta] = c^\gamma_{\alpha \beta} e_\gamma \:,
\end{align}
kde $[\cdot,\cdot]$ značí obyčejný komutátor funkcí. Předpokládejme dále, že pro navzájem různé $\alpha, \beta, \gamma$ je $c^\gamma_{\alpha \beta}$ různé od nuly .

\begin{definition}
    Matici $f^i_j(u)$ splňující podmínky výše nazveme hamiltonovskou maticí.
\end{definition}

\begin{theorem}[O rekonstrukci metriky]
    Nechť $N \geq 3$. Nechť je dána hamiltonovská matice $f^i_j(u)$. Pak lze zkonstruovat nedegenerovanou metriku $g^{ij}(u)$ s nulovou křivostí jednoznačně až na násobek konstantou.
\end{theorem}

\todo[inline]{Tomuhle nerozumím a myslím si, že v článku mají typo.}
\begin{proof}
    Z rovnice \eqref{eq:gf=gf} je vidět, že v bázi $e_\alpha$ je metrika $g^{ij}$ diagonální. 
    \todo[inline]{Opravdu?}
    V této bázi pak bude mít rovnice \eqref{eq:nablaf} tvar (zde se nesčítá přes opakované indexy)
    \begin{align}
        \partial_\alpha \lambda_\beta \delta_\beta^\gamma - \partial_\beta \lambda_\alpha \delta_\alpha^\gamma + (\Gamma_{\alpha \beta}^\gamma - \Gamma_{\beta \alpha}^\gamma ) \lambda_\gamma + \Gamma_{\beta \alpha}^\gamma (\lambda_\beta - \lambda_\alpha) = 0 \:.
    \end{align}
    Zde $\partial_\alpha$ je derivace ve směru $e_\alpha$ a konexe $\Gamma_{\alpha \beta}^\gamma$ jsou definované rovnostmi
    \begin{align}
        \nabla_{e_\beta} e_\alpha = \sum \Gamma^\gamma_{\alpha \beta} e_\gamma \:.
    \end{align}

    Normalizujme nyní vlastní vektory $e_\alpha$ tak, aby v této bázi byla metrika jednotková matice, tj. $g^{\alpha \beta} = \delta^{\alpha \beta}$.

    Výraz $c^\gamma_{\alpha \beta}$ má význam torze (z definice) a platí
    \begin{align}
        c^\gamma_{\beta \alpha} = \Gamma^\gamma_{\alpha \beta} - \Gamma^\gamma_{\beta \alpha} \:.
    \end{align} 

\end{proof}


\section{Multi-dimensional case}

\begin{proposition}[Changing the space coordinates]
    For any changes of the spatial variables $x^ \alpha \mapsto c^\alpha_\beta x^\beta$, where $\alpha = 1, \cdots, d$ and $\det (c^\alpha_\beta) = 1$, the metrics $g^{ij \alpha}$ and $b^{ij \alpha}_k$ transforms as components of a vector.
\end{proposition}



Ve vícerozměrném případě máme lineární bundle metrik a s nimi spojených konexí. Pro každou záměnu prostorových proměnných $x^\alpha \mapsto c^\alpha_\beta x^\beta$ pro $\alpha = 1, \cdots, d$ splňující $\det c^\alpha_\beta = 1$ se metrika $g^{ij \alpha}$ a konexe $b^{ij \alpha}_k$ transformují jako komponenty vektoru.

\begin{definition}
    Řekneme, že bundle metrik $g^{ij \alpha}$ je silně nedegenerovaný, jestliže pro nějakou sadu $c_\alpha$ je lineární kombinace $c_\alpha g^{ij \alpha}$ nedegenerovaná matice.
\end{definition}

\begin{theorem}
    Nechť je hamiltonovský systém silně nedegenerovaný.
    \begin{enumerate}
        \item Pro $N=1$ lze Poissonovu závorku redukovat na konstantní formu
        \begin{align}
            g^{ij}(u) = \tilde g^{ij}(u) \:.
        \end{align}

        \item Pro $N \geq 2$ lze Poissonovu závorku redukovat na lineární formu
        \begin{align}
            g^{ij \alpha}(u) = g^{ij \alpha}_k u^k + \tilde g^{ij \alpha} \:, \quad \alpha = 1, \cdots , d \:, \\
        \end{align}
        kde koeficienty $g^{ij \alpha}_k = b^{ij \alpha}_k + b^{ij \alpha}_k$, $\tilde g^{ij \alpha}$ a $b^{ij \alpha}_k$ jsou konstantní.
    \end{enumerate}
\end{theorem}
