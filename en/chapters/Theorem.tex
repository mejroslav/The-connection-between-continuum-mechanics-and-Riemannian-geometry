\chapter{Systems of hydrodynamic type: one-dimensional case}

In this chapter we prove the theorem of the connection between the hydrodynamic Poisson bracket and the metrics found in the homogeneous system of hydrodynamic type. The original result was given in 1983 by B. A. Dubrovin and S. P. Novikov and can be found in \cite{Dubrovin-Novikov}. Since then, many authors have contributed to the further development of this theory, e.g., Grinberg in \cite{Grinberg}, Ferapontov in \cite{Ferapontov} and Mochov in \cite{Mochov}.
The proof itself and the preceding lemmas are based on a considerable amount of technical computations which we shall clarify in detail to offer the reader a comprehensive understanding of the chapter in question.


\section{Systems of hydrodynamic type}

\begin{definition}[Homogeneous system of hydrodynamic type]
    A homogeneous system of hydrodynamic type is an equation of the form
    \begin{align}
        \label{eq:h-system}
        \pder{u^i}{t} = f_j^{i \alpha} (u) \pder{u^j}{x^\alpha} \:,
    \end{align}
    where $u^i(t,x^\alpha)$ are unknown functions and $i = 1, \dots , N$, $\alpha = 1, \dots , d$.
\end{definition}

Sometimes we refer to the number of equations $N$ as the dimension of the hydrodynamical phase space and to the number of spatial variables $d$ as the dimension of the configuration space.

It was Riemann who noticed that the functions $f^{i \alpha}_j$ in \eqref{eq:h-system} are in fact tensors \cite{Dubrovin-Novikov}.

\begin{proposition}[Transformation of $f^{i \alpha}_j$] 
    \label{prop:transformace-A} 
    Under smooth change of variables $u^i \mapsto v^a$ of the form
    \begin{align}
        u^i = u^i(v^1,\cdots, v^N) \:,
    \end{align}
    the functions $f^{i\alpha}_{j}$ transform for each $\alpha$ according to the tensor law
    \begin{align}
        f^{a \alpha}_{b}(v) = \pder{v^a}{u^i} \pder{u^j}{v^b} f^{i \alpha}_j (u) \:.
    \end{align}
\end{proposition}
\begin{proof}
    Using a direct calculation, we obtain
    \begin{align}
        \pder{u^i}{t}(v) = \pder{u^i}{v^a} \pder{v^a}{t} \:, \\
        \pder{u^j}{x^\alpha}(v) = \pder{u^j}{v^b} \pder{v^b}{x^\alpha}\:.
    \end{align}
    Hence
    \begin{align}
        \pder{u^i}{v^a} \pder{v^a}{t} = \pder{u^j}{v^b} f^{i \alpha}_j (v(u)) \pder{v^b}{x^\alpha} \:,
    \end{align}
    therefore
    \begin{align}
        \pder{v^a}{t} = \underbrace{\pder{v^a}{u^i} \pder{u^j}{v^b} f^{i \alpha}_j(v(u))}_{f^{a \alpha}_b(v)} \pder{v^b}{x^\alpha} \:.
    \end{align}
\end{proof}

According to the previous lemma if we denote the space of points $(t,x^\alpha) \subset \R \times \R^N$ as some manifold $M$, then the unknown variables $u^i(t,x^\alpha)$ can be seen as the local coordinates of that manifold. The functions $f^{i \alpha}_j$ are transformed as tensors of the type $(1,1)$ in the $i$ and $j$ indices.

In hydrodynamics, the functions $f^{i \alpha}_j$ are of a special form given by the energy of the system. Such systems will be the object of our investigation, and we shall introduce a much richer geometry for them - the geometry of Poisson brackets.

\begin{definition}[Functional of hydrodynamic type] \label{def:hamiltonian-functional}
    The functional of hydrodynamic type is a functional of the form
        \begin{align}
            H[u] = \int_{\R^d} h(u(x)) \D x \:,
        \end{align}
    where $h$ is independent of $u_\alpha$, $u_{\alpha \beta}$. The function $h$ is called a Hamiltonian density.
\end{definition}


\begin{definition}[Hamiltonian system of hydrodynamic type] \label{def:hamiltonian-system}
    The Hamiltonian system of hydrodynamic type is a system of the form
        \begin{align}
            \pder{u^i}{t} = 
            \left[ g^{ij \alpha} [u(x)] \ppder{h(u)}{u^j}{u^k} + b_k^{ij \alpha} [u(x)] \pder{h(u)}{u^j} \right] \pder{u^k}{x^\alpha} \:, \label{eq:Hamiltonian-system-of-h-t}
        \end{align}
        where $H[u]$ is the functional of the hydrodynamic type and $h$ the Hamiltonian density.
\end{definition}

\begin{definition}[Dubrovin-Novikov bracket]
    The Dubrovin-Novikov bracket of two functionals $I_1$ a $I_2$ is defined as
        \begin{align}
            \set{I_1,I_2} = \sum_{\alpha = 1}^ d \int_{\R^d} \D x 
            \left[ \frac{\delta I_1}{\delta u^p(x)} \left( g^{pq \alpha}(u) \der{}{x^\alpha} + b^{pq\alpha}_s(u) \pder{u^s}{x^\alpha} \right) \frac{\delta I_2}{\delta u^q(x)} \right] \:, \label{eq:Dubrovin-Novikov bracket}
        \end{align}
    where $g^{ij \alpha}$ and $b_k^{ij \alpha}$ are certain functions, $i,j,k = 1, \dots, N$ and $\der{}{x^\alpha}$ is the total derivative with respect to the independent variable $x^\alpha$, denoted by
        \begin{align}
            \der{}{x^\alpha} =& \pder{}{x^\alpha} + u^i_\alpha \pder{}{u^i} + u^i_{\alpha \beta} \pder{}{u^i_\beta} + \cdots + u^i_{\beta_1 \cdots \beta_s \alpha} \pder{}{u^i_{\beta_1 \cdots \beta_s}} + \cdots \:, \\
            u^i_{\beta_1 \cdots \beta_s} =& \pder{^s u^i}{x^{\beta_1} \cdots \partial x^{\beta_s}} \:.
        \end{align}
    The summation over repeating upper and lower indices is assumed, moreover, in this case the summation over $\beta_1, \cdots, \beta_s$ is taken over distinct (up to arbitrary permutation) sets of these indices. (One can consider that these sets of indices are ordered: $\beta_1 \leq \cdots \leq \beta_s$.)
\end{definition}

It is useful to write down the explicit form of the bracket of two field variables.

\begin{proposition}[Dubrovin-Novikov bracket of field variables]
    \begin{align}
        L^{ij}(x,y) \coloneqq \set{u^i(x), u^j(y)} = g^{ij \alpha} [u(x)] \delta_\alpha(x-y) + b_k^{ij \alpha} [u(x)] u_\alpha^k(x) \delta(x-y) \:.
    \end{align}
\end{proposition}

\begin{proof}
    \begin{align}
        \set{u^i(x), u^j(y)} =& \int \D \xi \fder{u^i(x)}{u^p(\xi)} \left( g^{pq \alpha}[u\xi)] \der{}{\xi^\alpha} + b^{pq \alpha}_s[u(\xi)] \pder{u^s}{\xi^\alpha} \right) \fder{u^j(y)}{u^q(\xi)} \nonumber
        \\ =& \int \D \xi \delta(x-\xi) \left[ g^{ij \alpha}[u(\xi)] \partial_\alpha \delta(y-\xi) + b^{ij \alpha}_s[u(\xi)] u^s_\alpha \delta(y- \xi) \right] \nonumber
        \\ =& g^{ij \alpha} [u(x)] \delta_\alpha(x-y) + b_k^{ij \alpha} [u(x)] u_\alpha^k(x) \delta(x-y) \:.
    \end{align}
\end{proof}

\begin{proposition}[Time evolution of the field variables]
    The Hamiltonian system of hydrodynamic type can be written as
    \begin{align}
        \pder{u^i}{t} = \set{u^i,H} \:.
    \end{align}
\end{proposition}
\begin{proof}
    \begin{align}
        &\set{u^i(x),H} 
        = \int \D \xi \fder{u^i(x)}{u^p(\xi)} \left[ g^{pq \alpha}[u(\xi)] \der{}{\xi} + b^{pq \alpha}_m [u(\xi)] \pder{u^m}{\xi^\alpha} \right] \fder{H}{u^q(\xi)} \nonumber 
        \\ =& \int \D \xi \int \D y \delta(x-\xi) \left[  g^{ij \alpha} [u(\xi)] \der{}{\xi} \pder{h}{u^j}(\xi)  + b^{ij \alpha}_k [u(\xi)] u^k_\alpha \pder{h}{u^j} (\xi) \right] \delta(y-\xi) \nonumber
        \\ =& \left[ g^{ij \alpha} [u(x)] \pder{^2 h}{u^j \partial u^k}(x)  + b^{ij \alpha}_k(u(x)) \pder{h}{u^j}(x) \right] \pder{u^k}{x^\alpha} \:,
    \end{align}
    which is the right-hand side of \eqref{eq:Hamiltonian-system-of-h-t}.
\end{proof}

Later we shall show that the Dubrovin-Novikov bracket defined above can be seen as Poisson bracket (of hydrodynamic type), i.e. satisfies  skew-symmetry, transformation rule and Jacobi identity. In fact, this will be the key theorem of this chapter.

\section{One-dimensional case}

For the rest of this chapter we first explore the one-dimensional brackets. In that case the unknown functions $u^i$ depend on only two variables $(t,x)$. The Hamiltonian system is then
\begin{align}
    \pder{u^i}{t} (t,x) =&
    \left[ g^{ij} [u(t,x)] \ppder{h(u)}{u^j}{u^k} + b_k^{ij} [u(t,x)] \pder{h(u)}{u^j} \right] \pder{u^k}{x}(t,x) \:, 
\end{align}
the Dubrovin-Novikov bracket is of the form
\begin{align}
    \set{I_1,I_2} = \int \D x
    \left[ \frac{\delta I_1}{\delta u^p(x)} \left( g^{pq}[u(x)] \der{}{x} + b^{pq}_s[u(x)] \pder{u^s}{x} \right) \frac{\delta I_2}{\delta u^q(x)} \right] \:, \label{eq:Dubrovin-Novikov bracket 1D}
\end{align}
and in the case of two field variables
\begin{align}
    \label{eq:Poisson1D}
    L^{ij}(x,y) = \set{u^i(x), u^j(y)} =& g^{ij} [u(x)] \partial_x \delta(x-y) + b_k^{ij} [u(x)] u^k_x \delta(x-y) \:. 
\end{align}
-
The bracket of two field variables forms a tensor as well.

\begin{proposition}[Transformation of $L^{ij}$]
    Let $u^i \mapsto v^a$ be the smooth change of variables. Then $L^{ij}$ transforms as the (2,0) tensors,
    \begin{align}
        L^{ab}(v) = \pder{v^a}{u^i}(x) \pder{v^b}{u^j}(y) L^{ij}(u) \:. \label{eq:transformace-zavorka}
    \end{align}
\end{proposition}
\begin{proof}
    The coordinates are written as functionals in the form
    \begin{align}
        v^p(u^i(x)) = \int \D x' v^p(u^i(x')) \delta (x-x') \:.
    \end{align}
    Applying this to the definition of the Dubrovin-Novikov bracket \eqref{eq:Poisson1D}, we have
    \begin{align}
        \set{v^p(u^i(x)),v^q(u^j(y))} 
        =& \int \D x' \int \D y' \underbrace{\fder{v^p}{u^i(x')} }_{\pder{v^p}{u^i}(x') \delta(x-x')} \set{u^i(x'),u^j(y')} \underbrace{\fder{v^q}{u^j(y')} }_{\pder{v^q}{u^j}(y') \delta(y-y')} \nonumber
        = \\ =& \pder{v^p}{u^i}(x) \pder{v^q}{u^j}(y) \set{u^i(x),u^j(y)} \:.
    \end{align}
\end{proof}

Our goal is to specify the conditions for the Dubrovin-Novikov bracket to be Poisson bracket. The most straightforward property is the transformation rule which is satisfied as long as the product of two functionals exists.

\begin{proposition}[Transformation rule for the bracket]
    \begin{align}
        \set{I,JK} = J \set{I,K} + \set{I,J} K \:,
    \end{align}
    assuming that both sides of the equation exist.
\end{proposition}
\begin{proof}
    Assuming that all equalities below are true,
    \begin{align}
        &\int \fder{AB}{u(x)} h(x) \, \D x = \der{}{\lambda} \big|_{\lambda = 0} \left[ A(u+ \lambda h) B(u+ \lambda h) \right] \nonumber
        \\ =& B(u) \der{}{\lambda} \big|_{\lambda = 0} A(u+ \lambda h) + A(u) \der{}{\lambda} \big|_{\lambda = 0} B(u+ \lambda h) \nonumber
        \\ =& B(u) \int \fder{A}{u} h(x) \, \D x + A(u) \int \fder{B}{u} h(x) \, \D x \:,
    \end{align}
    the transformation property follows immediately from the definition of the bracket \eqref{eq:Dubrovin-Novikov bracket 1D}.
\end{proof}


Skew-symmetry and Jacobi identity for the bracket are not easy to investigate. In order to do that, we introduce a pseudo-Riemann structure on this space. We focus on the case, where the system is non-degenerate and thus allows to raise and lower indices of tensors.

\begin{definition}[Non-degenerate systems]
    We say that the Dubrovin-Novikov bracket and the corresponding system are non-degenerate if $\det(g^{ij}) \neq 0$.
\end{definition}

 We shall show later that the non-degeneracy is invariant under the transformation of coordinates. In that case, we can naturally define its inverse form by $g_{ij} \coloneqq (g^{ij})^{-1}$.

Let $g^{ij}(u)$ be non-degenerate. Then we can define functions $\Gamma^{i}_{jk}(u)$ as
\begin{align}
    \Gamma^i_{jk}(u) \coloneqq - g_{js}(u) b^{s i}_k(u) \:. \label{eq:Gamma=g*b}
\end{align}
Note that this relation enables the decomposition
\begin{align}
    b_k^{ij}(u) = -g^{is}(u) \Gamma^j_{sk}(u) \:. \label{eq:b=g*Gamma}
\end{align}
The choice of denotation $g^{ij}(u)$ and $\Gamma^{i}_{jk}(u)$ has its purpose which is justified by the next observation.

\begin{proposition}[Transformation of $g^{ij}$ and $\Gamma^i_{jk}$] \label{prop:transoformace}
    Let $u^i \mapsto v^a$ be the smooth change of variables. Then
    \begin{enumerate}
        \item The $g^{ij}$ coefficients transform as the (2,0) tensors,
        \begin{align}
            g^{pq}(v) = \pder{v^p}{u^i} \pder{v^q}{u^j} g^{ij}[u(v)] \:. \label{eq:transformace-metrika}
        \end{align}
        \item The $\Gamma^i_{jk}$ coefficients  transform as Christoffel symbols,
        \begin{align}
            \Gamma^p_{qr} (v) = \pder{v^p}{u^i} \pder{u^j}{v^q} \pder{u^k}{v^r} \Gamma^i_{jk}(u) + \pder{v^p}{u^i} \ppder{u^i}{v^q}{v^r} \:. \label{eq:transformace-konexe}
        \end{align}
    \end{enumerate}
\end{proposition}

\begin{proof}
    \begin{enumerate}
        \item Let us apply the transformation equation \eqref{eq:transformace-zavorka} to the definition of the Dubrovin-Novikov bracket \eqref{eq:Poisson1D}. We get 
        \begin{align}
            g^{pq}[v(u(x))] \partial_x \delta(x-y) + b^{pq}_s[v(u(x))] \pder{v^s}{u^k}(x) \pder{u^k}{x} \delta(x-y) \nonumber
            \\ = \pder{v^p}{u^i}(x) \pder{v^q}{u^j}(y) \left[ g^{ij}[u(x)] \partial_x \delta(x-y) + b^{ij}_k[u(x)] \pder{u^k}{x} \delta(x-y) \right] \:. \label{eq:dosad}
        \end{align}
        From now on (for readibility) we denote 
        \begin{align}
            T^a_k(x) := \pder{v^a}{u^k}(x) \:.
        \end{align}
        We apply Dirac identity (Lemma \vref{lemma:delta}) to $T^q_j(y) \partial_x \delta(x-y)$ on the right-hand side of \eqref{eq:dosad}. Hence all functions are then of variable $x$, from now on we shall not write it explicitely. We have
        \begin{align}
            g^{pq}(v) \partial_x \delta(x-y) + b^{pq}_n(v) T^n_j \pder{u^j}{x} \delta(x-y) \nonumber
            \\ = T^p_i T^q_j g^{ij}(u) \partial_x \delta(x-y) 
            + \left( \ppder{v^q}{u^j}{u^k} + T^p_i T^q_j b^{ij}_k \right) \pder{u^k}{x} \delta(x-y) \:. \label{eq:porovnani}
        \end{align}
        The terms with $\partial_x \delta$ must be equal to each other, as must $\delta$ terms. From $\partial_x \delta$ terms we get
        \begin{align}
            g^{pq}(v) = T^p_i T^q_j g^{ij}(u) \:,
        \end{align}
        which is the transformation relation for $g$.

        \item From $\delta$ terms in \eqref{eq:porovnani} we obtain
        \begin{align}
            b^{pq}_n(v) T^n_k =
            b^{ij}_k(u) T^p_i T^q_j + g^{ij}(u) T^p_i \ppder{v^q}{u^i}{u^k}
        \end{align}
        and after the decomposition \eqref{eq:b=g*Gamma}
        \begin{align}
            - g^{ps}(v) \Gamma^{q}_{sn}(v) T^n_k =
            - g^{il}(u) \Gamma^j_{lk}(u) T^p_i T^q_j + g^{ij}(u) T^p_i \ppder{v^q}{u^i}{u^k} \:.
        \end{align}
        On the left-hand side we use the transformation of $g$ \eqref{eq:transformace-metrika}
        \begin{align}
            -g^{ij}(u) \Gamma^q_{sm}(v) T^p_i T^s_j T^m_k = - g^{ij}(u) \Gamma^l_{jk} T^p_i T^q_l + g^{ij}(u) \ppder{v^q}{u^i}{u^k} T^p_i \:,
        \end{align}
        and after factoring out $g^{ij}(u) T^p_i$
        \begin{align}
            \Gamma^l_{jk}(u) T^q_l = \Gamma^q_{sm}(v) T^s_j T^m_k + \ppder{v^q}{u^i}{u^k} \:.
        \end{align}
        All that remains is to multiply this equation with $(T^{-1})^a_q$ after which we receive the final transformation relation
        \begin{align}
            \Gamma^a_{jk}(u) = \Gamma^q_{sm}(v) \pder{v^s}{u^j} \pder{v^m}{u^k} \pder{u^a}{v^q} + \ppder{v^q}{u^i}{u^k} \pder{u^a}{v^q} \:.
        \end{align}
    \end{enumerate}
\end{proof}



\section{Dubrovin-Novikov theorem}

Let us summarize what we have investigated so far. We have defined the Dubrovin-Novikov bracket in the way that it satisfies the transformation rule. It remains to investigate the skew-symmetry and Jacobi identity. For that, we have defined $g^{ij}$ and $\Gamma^i_{jk}$ which transform as tensors, and Christoffel symbols, respectively. It remains to show that $g^{ij}$ is indeed symmetric and $\Gamma^i_{jk}$ is the Levi-Civita connection, i.e., the corresponding covariant derivative $\nabla_j$ has zero torsion and is compatible with the metric.

We are now ready to prove the critical theorem of this chapter, which shows that these properties are indeed equivalent. Moreover, the corresponding covariant derivative has zero curvature, indicating that the phase space $M$ can be regarded as the Euclidean space.

\begin{theorem}[Dubrovin, Novikov;\cite{Dubrovin-Novikov}] \label{Theorem-Dubrovin-Novikov-1D}
    Let $g^{ij}$ be non-degenerate. Then the Dubrovin-Novikov bracket \eqref{eq:Poisson1D} is a Poisson bracket with skew-symmetry, transformation rule and Jacobi identity if and only if these three conditions are satisfied:
    \begin{enumerate}
        \item $g^{ij} = g^{ji}$, that is, $g^{ij}$ forms an inverse metric on the phase space $M$.
        \item $\Gamma^{i}_{jk} = \Gamma^i_{kj}$, that is, $\Gamma^{i}_{jk}$ are Christoffel symbols corresponding to the Levi-Civita covariant derivative compatible with $g^{ij}$.
        \item The covariant derivative corresponding to $\Gamma^{i}_{jk}$ has zero curvature.
    \end{enumerate}
\end{theorem}

\begin{corollary} \label{Corollary: existence of flat coordinates}
    There exist local coordinates $w^i = w^i(u^1, \dots, u^N)$, $i = 1, \dots, N$ such that the metric is flat:
    \begin{align}
        g^{ij}(w) = \eta^{ij} = \const \:, \quad \Gamma^i_{jk} (w) = 0 \:.
    \end{align}
    In these flat coordinates the Dubrovin-Novikov bracket is reduced to the constant form
    \begin{align}
        \set{w^i(x),w^j(y)} = \eta^{ij} \partial_x \delta(x-y) \:.
    \end{align}
    The only local invariant of Poisson bracket is the signature of the metric $g^{ij}(u)$.
\end{corollary}

Due to the relatively large length of the proof, we divide it into four steps.
\begin{itemize}
    \item \textbf{Step 1:} We show that the skew-symmetry of the bracket implies the symmetry of the metric $g^{ij}$ and compatibility with the Christoffel symbols $\Gamma^i_{jk}$.
    \item \textbf{Step 2:} We formulate the conditions under which the Jacobi identity for the bracket holds.
    \item \textbf{Step 3:} We show that two of these conditions implies zero torsion and zero curvature of the covariant derivative associated with $\Gamma^i_{jk}$. That finishes the implication in one direction.
    \item \textbf{Step 4:} We prove the converse implication via the existence of the flat coordinates.
\end{itemize}

\begin{lemma}[Step 1: skew-symmetry] \label{lemma:skew-symmetry}
    If the skew-symmetry of the Dubrovin-Novikov bracket holds, then
    \begin{align}
        &g^{ij} = g^{ji} \:, \label{eq:skew-1}\\
        &g^{ij}_{,m} = b^{ij}_m + b^{ji}_m \label{eq:skew-2} \:.
    \end{align}
\end{lemma}
\begin{proof}
    The skew-symmetry reads as
    \begin{align}
        \set{u^i(x),u^j(y)} + \set{u^j(y),u^i(x)} = 0 \:.
    \end{align}
    According to the definition, the exact form of the second bracket is
    \begin{align}
        \set{u^j(y),u^i(x)} =  g^{ji}[u(y)] \partial_x \delta(y-x) + b^{ji}_k[u(y)] \pder{u^k}{y} \delta(y-x) \:.
    \end{align}
    Let us apply Dirac identites (Lemma \vref{lemma:delta}) on $g^{ji}[u(y)]\partial_x \delta(y-x)$. We get
    \begin{align}
        \set{u^j(y),u^i(x)} =& \nonumber \\ 
        = - g^{ji} [u(x)] \partial_x \delta(x-y) -& \pder{g^{ji}}{u^k}[u(x)] \pder{u^k}{x} \delta(x-y) + b^{ji}_k[u(x)] \pder{u^k}{x} \delta(x-y)
        \:.
    \end{align}
    The skew-symmetry is then
    \begin{align}
        0 =& \set{u^i(x),u^j(y)} + \set{u^j(y),u^i(x)}  \nonumber
        \\=& \left[ g^{ij}[u(x)]  - g^{ji}[u(x)] \right] \partial_x \delta(x-y) + \left[ b^{ij}_k - \pder{g^{ji}}{u^k}  + b^{ji}_k  \right] \pder{u^k}{x} \delta(x-y) \:.
    \end{align}
    The right-hand side is equal to zero if and only if
    \begin{align}
        g^{ij}(u) = g^{ji}(u) \:, \label{eq:Th.Novikov-symetrie} \\
        \pder{g^{ij}}{u^k}(u) = b^{ij}_k(u) + b^{ji}_k(u)  \:. \label{eq:Th.Novikov-konexe}
    \end{align}
\end{proof}

\begin{lemma}[Step 2: Jacobi identity] \label{lemma:Jacobi}
    The Jacobi identity for the bracket is equivalent to 
    \begin{align}
        &b^{ij}_m g^{mk} = b^{ik}_m g^{mj} \:, \label{eq:Jacobi-1}\\ 
        &b^{ij}_{m} b^{mk}_s - b^{ik}_m b^{mj}_s = g^{mi} \left( b^{jk}_{s,m} - b^{jk}_{m,s} \right) \:, \label{eq:Jacobi-2}\\
        \sum_{(i,j,k)}& \left[ (b^{ij}_{t,p} - b^{ij}_{p,t}) b^{tk}_s + (b^{ij}_{t,s} - b^{ij}_{s,t}) b^{tk}_p \right] = 0 \:, \label{eq:Jacobi-3}
    \end{align}
    where the summation $\sum_{(i,j,k)}$ means summation over cyclic permutations of indices $i,j,k$.
\end{lemma}

\begin{proof}
    We denote
    \begin{align}
        \mathcal J^{ijk}(x,y,z) = J^{ijk}(x,y,z)+ J^{jki}(y,z,x)+ J^{kij}(z,x,y) = 0 \:,
    \end{align}
    where
    \begin{align}
        J^{ijk}(x,y,z) = \set{\set{u^i(x),u^j(y)}, u^k(z)} \:.
    \end{align}
    From now on we shall use the abbreviated notation
    \begin{align}
        g^{ij}_{,m} = \pder{g^{ij}}{u^m} \:, \quad b^{ij}_{k,m} = \pder{b^{ij}_k}{u^m} \:, \quad u^k_x = \pder{u^k}{x}  \:.
    \end{align}
    One of the Jacobiators reads as
    \begin{align}
        J^{ijk}(x,y,z) =
        \int \D \xi \fder{\set{u^i(x),u^j(y)}}{u^m(\xi)} 
        \left[ g^{mn}(\xi) \partial_\xi + b^{mn}_p \partial_\xi u^p \cdot \right] \fder{u^k(z)}{u^n(\xi)}  \:.
    \end{align}
    The functional derivatives are
    \begin{align}
        \fder{u^k(z)}{u^n(\xi)} = \delta^k_n \delta(z-\xi) \:,
    \end{align}
    and 
    \begin{align}
        \fder{\set{u^i(x),u^j(y)}}{u^m(\xi)} = \fder{}{u^m(\xi)} \left[ g^{ij}(x) \partial_x \delta(x-y) + b^{ij}_k(x) \partial_x u^k \delta(x-y)\right] = 
        \nonumber
        \\ = \left[ g^{ij}_{,m}(x) \partial_x \delta(x-y) + b^{ij}_{k,m}(x) u^k_x \delta(x-y) \right] \delta (x-\xi) - b^{ij}_m(x) \delta(x-y) \partial_\xi \delta(x-\xi) \:.
    \end{align}
    We have
    \begin{align}
        J^{ijk}(x,y,z) = J_1^{ijk}(x,y,z) + J_2^{ijk}(x,y,z) \:,
    \end{align}
    where
    \begin{align}
        & J_1^{ijk}(x,y,z) = \int \D \xi  \left[ g^{ij}_{,m}(x) \partial_x \delta(x-y) \delta(x-\xi) + b^{ij}_{k,m}(x) u^k_x \delta(x-y) \delta(x-\xi) \right] \times \nonumber \\ 
        & \times \left[ g^{mn}(\xi) \partial_\xi + b^{mn}_p \partial_\xi u^p \cdot \right] \delta^k_n \delta(z-\xi)  \:, 
    \end{align}
    and 
    \begin{align}
        J_2^{ijk}(x,y,z) = - \int \D \xi b^{ij}_m(x) \delta(x-y) \partial_\xi \delta(x-\xi) \left[ g^{mn}(\xi) \partial_\xi + b^{mn}_p \partial_\xi u^p \cdot \right] \delta^k_n \delta(z-\xi)  \:.
    \end{align}
    In $J_1$ we use integration with $\delta(x-\xi)$ and get rid of the integral:
    \begin{align}
        &J_1^{ijk}(x,y,z) = \nonumber
        \\=& \left[ g^{ij}_{,m}(x) \partial_x \delta(x-y) + b^{ij}_{k,m}(x) u^k_x \delta(x-y) \right] 
        \left[ g^{mk}(\xi) \partial_x + b^{mk}_p \partial_x u^p \cdot \right] \delta(z-x) \nonumber
        \\
        =& \left[ g^{ij}_{,m} g^{mk} \partial_x \delta(x-y) + b^{ij}_{k,m} g^{mk} u^k_x \delta(x-y) \right] \partial_x \delta(z-x) + \nonumber
        \\ 
        +& \left[ g^{ij}_{,m} b^{mk}_p u^p_x \partial_x \delta(x-y) + b^{ij}_{k,m} b^{mk}_p u_x^p  u_x^k \delta(x-y) \right] \delta(z-x) \:.
    \end{align}

    In $J_2$ we first apply integration by parts to eliminate $\partial_\xi \delta(x-\xi)$ and then integrate:
    \begin{align}
        &J_2^{ijk}(x,y,z) = \nonumber
        \\=& \int \D \xi b^{ij}_m(x) \delta(x-y) \delta(x-\xi) \partial_\xi \left[ g^{mk}(\xi) \partial_\xi \delta(z-\xi) + b^{mk}_p \partial_\xi u^p \delta(z-\xi) \right] \nonumber
        \\ =& b^{ij}_m(x) \delta(x-y) \partial_x \left[ g^{mn}(x) \partial_x \delta(z-x) + b^{mn}_p \partial_x u^p \delta(z-x) \right] \nonumber
        \\ =& b^{ij}_m(x) \delta(x-y) \big[ g^{mk}_{,s} u^s_x \partial_x \delta (z-x) + g^{mk} \partial_x \partial_x \delta (z-x) \nonumber
        \\ +& b^{mk}_{p,s}(x) u^s_x u^p_x \delta (z-x) + b^{mn}_{p}(x) u^p_{xx} \delta (z-x) + b^{mk}_{p}(x) u^p_x \partial_x \delta (z-x) \big] \:,
    \end{align}
    This leads to
    \begin{align}
        J_2^{ijk}(x,y,z) =& b^{ij}_m g^{mk}_{,s} u^s_x \delta (x-y) \partial_x \delta (z-x) 
        + b^{ij}_m g^{mk} \delta(x-y) \partial_x \partial_x \delta (z-x) \nonumber
        \\ +& b^{ij}_m b^{mk}_{p,s} u^s_x u^p_x \delta (x-y) \delta (z-x)
        + b^{ij}_m b^{mn}_p u^p_{xx} \delta (x-y) \delta(z-x) \nonumber 
        \\ +& b^{ij}_m b^{mk}_p u^p_x \delta(x-y) \partial_x \delta(z-x) \:.
    \end{align}

    After some adjustments we finish with the following terms
    \begin{align}
        J^{ijk}(x,y,z) =& 
        E^{ijk}(x) \delta(x-y) \delta(z-x) 
        \\+& F^{ijk}(x) \delta(x-y) \partial_x \delta(z-x) \:,
        \\+& G^{ijk}(x) \partial_x \delta(x-y) \delta (z-x) \:,
        \\+& H^{ijk}(x) \partial_x \delta(x-y) \partial_x \delta (z-x) \:,
        \\+& I^{ijk}(x) \delta(x-y) \partial_x \partial_x \delta (z-x) \:,
    \end{align}
    where
    \begin{align}
        E^{ijk} =& (b^{ij}_m b^{mk}_{p,s} + b^{ij}_{s,m} b^{mk} ) u^p_x u^s_x + b^{ij}_m b^{mk}_p u^p_{xx} \:, \\
        F^{ijk} =& (b^{ij}_{s,m} g^{mk} + g^{mk}_{,s} b^{ij}_m + b^{ij}_m b^{mk}_s)u_x^s \:, \\
        G^{ijk} =& g^{ij}_{,m} b^{mk}_s u^s_x \:,\\
        H^{ijk} =& g^{ij}_{,m} g^{mk} \:,\\
        I^{ijk} =& b^{ij}_m g^{mk} \:.
    \end{align}
    Now the procedure is the following: firstly we have to add the cyclic permutations $J^{jki}(y,z,x)$ and $J^{kij}(z,x,y)$ to obtain $\mathcal J^{ijk}(x,y,z)$. Secondly, the resulting generalized function $\mathcal J^{ijk}(x,y,z)$ is equal to zero, i.e.
    \begin{align}
        \dual{\mathcal J^{ijk}(x,y,z)}{p_i(x)q_j(y)r_k(z)} = 0
    \end{align}
    or in the integral notation 
    \begin{align}
        \iiint \D x \D y \D z J^{ijk}(x,y,z) p_i(x)q_j(y)r_k(z) = 0 \:.
    \end{align}
    We want to reduce this integral into a one-dimensional integral
    \begin{align}
        \int \D x \sum_{a,b = 0}^2 A^{ijk}_{ab} p_i(x) q^{(a)}_j(x) r^{(b)}_k (x) \:,
    \end{align}
    where $f^{(a)}$ means the $a$-th derivative. This integral is equal to zero if and only if
    \begin{align}
        A^{ijk}_{ab} = 0 \quad \forall a,b = 0,1,2 \:.
    \end{align}
    From this condition, we shall obtain zero torsion and curvature.
    We illustrate the reduction procedure to the one-dimensional integral on one particular expression. For instance, we have
    \begin{align}
        \int \big[ F^{ijk}(x) \delta(x-y) \partial_x \delta(z-x) 
        + F^{jki}(y) \delta(y-z) \partial_y \delta(x-y) \nonumber \\
        + F^{kij}(z) \delta(z-x) \partial_z \delta(y-z) \big] \,
        p_i(x) q_j(y) r_k(z) \,\D x \D y \D z \:.
    \end{align}
    We use the antisymmetry $
        \partial_x \delta(z-x) = - \partial_z \delta (z-x)$
    and similarly in each term and then eliminate the integrals with $\D y$ and $\D z$, obtaining
    \begin{align}
        \int \D x F^{ijk}(x) p_i q_j r'_k + \int \D x F^{jki}(x)  p'_i q_j r_k +
        \int \D x F^{kij}(x) p_i q'_j r_k \:.
    \end{align}
    In the term containing $p'_i$ we perform integration by parts. The result is
    \begin{align}
        \int \D x (F^{ijk}-F^{jki}) p_i q_j  r'_k +
        \int \D x (F^{kij}-F^{jki}) p_i q'_j r_k -
        \int \D x \partial_x F^{jki} p_i q_j r_k \:.
    \end{align}
    In this particular case, we end up with terms, which contribute to $A^{ijk}_{01}$, $A^{ijk}_{10}$ and $A^{ijk}_{00}$, respectively. Other terms in $\mathcal J^{ijk}(x,y,z)$ are handled in similar way.

    The list of the resulting functions is the following:
    \begin{align}
        A_{00}^{ijk} =& E^{ijk} + E^{kij} + E^{jki} + \partial_x \partial_x I^{jki} - \partial_x F^{jki} - \partial_x G^{kij} \:, \label{eq:A00}\\
        A_{01}^{ijk} =& F^{ijk} - F^{jki} + G^{jki} - G^{kij} - \partial_x H^{jki} + 2 \partial_x I^{jki} \:, \label{eq:A01}\\
        A_{10}^{ijk} =& F^{kij} - F^{jki} + G^{ijk} - G^{kij} - \partial_x H^{kij} + 2 \partial_x I^{jki} \:, \label{eq:A10}\\
        A_{11}^{ijk} =& H^{ijk} - H^{jki} - H^{kij} + 2 \partial_x I^{jki} \:, \label{eq:A11} \\
        A_{02}^{ijk} =& I^{ijk} + I^{jki} - H^{jki}\:, \label{eq:A02}\\
        A_{20}^{ijk} =& I^{jki} + I^{kij} - H^{kij}\:, \label{eq:A20} \\
        A^{ijk}_{12} =& A^{ijk}_{21} = 0 \:.
    \end{align}
    Note that \eqref{eq:A20} is just a cyclic permutation of \eqref{eq:A02}. If we also use the symmetry conditon \eqref{eq:skew-1} and the compatibility condition \eqref{eq:skew-2}, then the equation \eqref{eq:Jacobi-1} follows from \eqref{eq:A02}. Equation \eqref{eq:A00} has the form
    \begin{align}
        A_{00}^{ijk} = B^{ijk}_s u^s_{xx} + C^{ijk}_{sp} u^s_x u^p_x \:.
    \end{align}
    The expression on the right hand side is zero if and only if both $B^{ijk}_s$ and $C^{ijk}_{sp}$ are equal to zero. From the function $B^{ijk}_s$ being zero follows \eqref{eq:Jacobi-2} and from $C^{ijk}_{sp}$ follows \eqref{eq:Jacobi-3}. Finally, equations \eqref{eq:A01}, \eqref{eq:A10} and \eqref{eq:A11} are satisfied automatically if \eqref{eq:A00} and \eqref{eq:A02} hold.
\end{proof}

It is useful to write once more all the equations from lemmas \vref{lemma:skew-symmetry} and \vref{lemma:Jacobi} together. Since they first appeared in \cite{Grinberg}, we refer to them as to the Grinberg conditions.
\begin{lemma}[Grinberg conditions; \cite{Grinberg}] \label{Grinberg conditions}
    If the Dubrovin-Novikov bracket satisfies skew-symmetry and Jacobi identity, then the following conditions hold:
    \begin{align}
        &g^{ij} = g^{ji} \:, \label{eq:Grinberg-1}\\
        &g^{ij}_{,m} = b^{ij}_m + b^{ji}_m \:, \label{eq:Grinberg-2}\\
        &b^{ij}_m g^{mk} = b^{ik}_m g^{mj} \:, \label{eq:Grinberg-3}\\
        &b^{ij}_{m} b^{mk}_s - b^{ik}_m b^{mj}_s = g^{mi} \left( b^{jk}_{s,m} - b^{jk}_{m,s} \right) \:, \label{eq:Grinberg-4}\\
        \sum_{(i,j,k)}& \left[ (b^{ij}_{t,p} - b^{ij}_{p,t}) b^{tk}_s + (b^{ij}_{t,s} - b^{ij}_{s,t}) b^{tk}_p \right] = 0 \:. \label{eq:Grinberg-5}
    \end{align}
\end{lemma}

Until now, we have not assumed non-degeneracy of the bracket. Note that non-degeneracy allows to use $\Gamma^{i}_{jk} = - g_{js} b^{si}_k$, since in the degenerate case the inverse $g_{js}$ does not exist. If we reformulate the Grinberg conditions in terms of $\Gamma^{i}_{jk}$, we get the compatibility, zero torsion, and curvature.

\begin{lemma}[Step 3: Zero curvature and torsion]
    If the Grinberg conditions hold and the $g^{ij}$ is non-degenerate, then $g^{ij}$ is a metric, the covariant derivative $\nabla_i$ associate with $\Gamma^{i}_{jk}$ is compatible with $g^{ij}$ and it has zero torsion and curvature, i.e. $\nabla_i$ is a Levi-Civita covariant derivative with zero curvature.
\end{lemma}
\begin{proof}
    Equation \eqref{eq:Grinberg-1} says that $g^{ij}$ is a symmetric tensor. By assumption $\det g^{ij}$ is non-degenerate, therefore it defines a metric structure on $M$. Equation \eqref{eq:Grinberg-2} is equivalent to
    \begin{align}
        g^{ij}_{,k} + g^{is} \Gamma_{sk}^j + g^{sj} \Gamma_{sk}^i = 0 \:.
    \end{align}
    If $\nabla_k$ is the covariant derivative corresponding to $\Gamma^i_{jk}$, then $\nabla_k g^{ij} = 0$. So $\Gamma_{ij}^k$ are affine connection coefficients, compatible with $g^{ij}$.

    To obtain zero torsion, we use \eqref{eq:Grinberg-3}, which reads as 
    \begin{align}
        g^{mk} g^{sj} \Gamma^i_{ms} = g^{mk} g^{sj} \Gamma^i_{sm} \:.
    \end{align}
    Hence the Christoffel symbols are symmetric in the lower indices, implying that the covariant derivative corresponding to them has zero torsion.

    Zero curvature follows from \eqref{eq:Grinberg-4}, which reads as
    \begin{align}
        g^{in} g^{ml} (\Gamma^j_{nm} \Gamma^{k}_{ls} - \Gamma^k_{nm} \Gamma^j_{ls}) = - g^{im} \pder{}{u^m} \left( g^{jl} \Gamma^k_{ls}\right) + g^{im} \pder{}{u^s}\left( g^{jl} \Gamma^k_{lm}\right) \:,
    \end{align}
    and reduces to
    \begin{align}
        g^{im} g^{jn} \left( \Gamma^{l}_{nm} \Gamma^{k}_{ls} - \Gamma^{l}_{ns} \Gamma^k_{lm} + \Gamma^k_{nm,s} - \Gamma^k_{ns,m} \right) = 0 \:.
    \end{align}
    This means that the Riemann tensor $R\indices{_n_m^k_s}$ is equal to zero.
\end{proof}

The final step is to prove the converse statement and the corollary \vref{Corollary: existence of flat coordinates}, which turns out to be very simple due to the non-degeneracy of the bracket. 

\begin{lemma}[Step 4: The converse statement]
    If $g^{ij}$ is non-degenerate metric and $\Gamma^i_{jk}$ correspond to the Levi-Civita covariant derivative $\nabla$ with zero curvature, then the Dubrovin-Novikov bracket is a Poisson bracket, i.e. it satisfies skew-symmetry, transformation rule and Jacobi identity.
\end{lemma}

\begin{proof}
    If $\nabla$ has zero torsion and curvature, there are coordinates\\ $w^i=w^i(u^1,\dots,u^N)$ for $i=1,\dots,N$ such that $g^{ij} \eqqcolon \eta^{ij} =\const$ a $b^{ij}_k = 0$. In these coordinates
    \begin{align}
        \set{w^i(x),w^j(y)} = \eta^{ij} \partial_x \delta(x-y) \:.
    \end{align}
    Jacobi identity, skew-symmetry and transformation rule are for that bracket satisfied trivially due to the antisymmetry of the delta distribution.
\end{proof}
Thus the proof of the Dubrovin-Novikov theorem \vref{Theorem-Dubrovin-Novikov-1D} is finished.

The coordinates in which Poisson bracket can be reduced to the constant form do not always have a physical interpretation \cite{Dubrovin-Novikov}. There is another class of coordinates, which is common in various applications. Coordinates $u^1, \cdots, u^N$ are called Liouville if the metric $g^{ij}$ and the connection $b^{ij}_k$ have the form
\begin{align}
    g^{ij}(u) = \gamma^{ij}(u)+ \gamma^{ji}(u) \:, \quad b^{ij}_k(u) = \gamma^{ji}_{,k}(u) \:. 
\end{align}
In those coordinates the Poisson bivector has the form
\begin{align}
    \set{u^i(x),u^j(y)} = \left[ \gamma^{ij}(u(y)) + \gamma^{ji}(u(x)) \right] \delta'(x-y) \:.
\end{align}

\begin{example}[One-dimensional isothermal system]
    In Example \vref{example:1D isothermal} we illustrated the one-dimensional isothermal system with $u^1 = \rho$, $u^2 = m$. These coordinates are in fact Liouville:
    \begin{align}
        \gamma^{ij} = -\begin{pmatrix}
            0 & 0 \\
            \rho & m
        \end{pmatrix} \:, \quad
        g^{ij} = \begin{pmatrix}
            0 & \rho \\
            \rho & 2m 
        \end{pmatrix} \:.
    \end{align}
\end{example}


\begin{example}[More general system, \cite{Dubrovin-Novikov}]
    The one-dimensional hydrodynamics with $(u^1,u^2,u^3) = (\rho, m, s)$. The Hamiltonian has the form
    \begin{align}
        H = \int \left[ \frac{p^2}{2 \rho} + \epsilon(\rho,s) \right] \, \D x \:,
    \end{align}
    where $\epsilon(\rho,s)$ is the energy density. These coordinates are in fact Liouville:
    \begin{align}
        \gamma^{ij} = \begin{pmatrix}
            p & 0 & 0 \\ \rho & 0 & 0 \\ s & 0 & 0
        \end{pmatrix}
        \:, \quad
        g^{ij} = \begin{pmatrix}
            2p & \rho & s \\ \rho & 0 & 0 \\ s & 0 & 0
        \end{pmatrix}
        \:.
    \end{align}
\end{example}

The direct consequence of the Dubrovin-Novikov theorem \vref{Theorem-Dubrovin-Novikov-1D} is the possibility of writing the original equations in covariant form.

\begin{theorem}[Hamiltonian system in the covariant form] \label{theorem:covariant}
    Given Hamiltonian system of hydrodynamic type
    \begin{align}
        \pder{u^i}{t} = f^i_k(u) \pder{u^k}{x} \:, \quad f^i_k(u) = g^{ij}(u) \ppder{h}{u^j}{u^k} - g^{is}(u) \Gamma^j_{sk}(u) \pder{h}{u^j} \:, \label{eq:hamiltonovsky-system}
    \end{align}
    one can write
    \begin{align}
        f^i_k(u) = \nabla^i \nabla_k h(u) \:, 
    \end{align}
    where $\nabla_j$ is Levi-Civita covariant derivative compatible with $g_{ij}$ and $\nabla^i = g^{is} \nabla_s$. 
\end{theorem}
\begin{proof}
    Since $h$ is ordinary function, application of $\nabla_k$ is just the partial derivative with respect to $u^k$:
    \begin{align}
        \nabla_k h(u) =& \pder{h}{u^k} \:.
    \end{align}
    Now this term $\pder{h}{u^k}$ is a covector, so $\nabla_i$ is given by the partial derivative and the $\Gamma_{ij}^k$ part by
    \begin{align}
        \nabla^i \nabla_k h(u) =& g^{is} \nabla_s \pder{h}{u^k} = g^{is} \left( \ppder{h}{u^s}{u^k} - \Gamma^j_{sk} \pder{h}{u^j} \right) = f^i_k(u) \:.
    \end{align}
\end{proof}
