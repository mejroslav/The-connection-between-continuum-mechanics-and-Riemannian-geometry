\section{Hyperbolicity}

\begin{definition}[Quasi-linear systems]
    \begin{align}
        \pder{u^i}{t} + a^i_j(u) \pder{u^j}{x} = 0 \:.
    \end{align}
\end{definition}

Let $\lambda_1(u), \cdots, \lambda_n(u)$ be the eigenvalues of $a^i_j(u)$ and $w_1(u), \cdots, w_n(u)$ be the eigenvectors. Then the system is called at some point $(t,x)$
\begin{itemize}
    \item hyperbolic if the eigenvalues are all real,
    \item strictly hyperbolic if the eigenvalues are all real and distinct,
    \item elliptic, if none of the eigenvalues are real,
    \item diagonalisable, if the matrix $a^i_j(u)$ is diagonalisable.
\end{itemize}
