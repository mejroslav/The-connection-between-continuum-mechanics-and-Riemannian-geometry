\section{Poisson brackets}

Poisson brackets and symplectic geometry play the essential role in the Hamiltonian mechanics and can be seen from many different perspectives. An inspirable book on this topic is e.g. \cite{Marsden}.  

\begin{definition}[Poisson bracket]
    A Poisson bracket is a bilinear operation $\set{\cdot, \cdot} : \F M \times \F M \to \F M$ which satisfies
    \begin{align}
        &\set{f,g} = -\set{g,f} \quad \textbf{(skew-symmetry)} \:, \\
        &\set{fg,h} = f \set{g,h} + g \set{f,h} \quad \textbf{(Leibniz identity)} \:, \\
        &\set{\set{f,g},h} + \set{\set{g,h},f} + \set{\set{h,f},g} = 0 \quad \textbf{(Jacobi identity)} \:.
    \end{align}
\end{definition}

\begin{definition}[Symplectic 2-form]
    A skew-symmetric 2-form $\vc \omega \in \T^0_2 M$ is said to be symplectic if it is closed and nondegenerate, i.e.
    \begin{align}
        &\D \vc \omega = 0 \quad \textbf{(closedness)} \:, \\
        &\det \vc \omega \neq 0 \quad \textbf{(nondegeneracy)}\:.
    \end{align}
    A pair $(M,\vc \omega)$ is called sympletic manifold.
\end{definition}

Since in odd dimension every skew-symmetric 2-form is degenerate, the dimension $m$ of the symplectic manifold $M$ must be even number. The following statement holds.

\begin{theorem}[Darboux theorem - symplectic forms formulation]
    Let $(M,\vc \omega)$ be a $2n$-dimensional symplectic manifold and $x \in M$. Then there is a neighbourhood $U(x)$ and a coordinate system $(p_1,\cdots,p_n,q^1,\cdots,q^n)$ such that $\vc \omega$ has the form
    \begin{align}
        \vc \omega = \D p_a \wedge \D q^a \quad \text{on } U(x) \:. \label{eq:standard form}
    \end{align}
\end{theorem}
We refer to this coordinate system as to the standard coordinates \\$(x^1,\cdots, x^{2n}) = (p_1,\cdots,p_n,q^1,\cdots,q^n)$ and to the form \eqref{eq:standard form} as to the standard form of $\vc \omega$.

\begin{definition}[Poisson bivector]
    Let $(M,\vc \omega)$ be a symplectic manifold.
    The Poisson bivector $\vc P \in \T^2_0 M$ is the inverse of $\vc \omega$, i.e.
    \begin{align}
        \vc P^{ij} \vc \omega_{jk} = \vc \delta^i_k \:.
    \end{align}
\end{definition}
Thus a symplectic manifold $(M,\vc \omega)$ has its associated Poisson bracket given by
\begin{align}
    \set{f,g} \coloneqq \vc P (\D f, \D g) = \vc P^{ab} \D_a f \D_b g \quad \forall f,g \in \F M \:. 
\end{align}
In the standard coordinates
\begin{align}
    \set{f,g} = \sum_{i=1}^n \left( \pder{f}{q^i} \pder{g}{p_i} - \pder{f}{p_i} \pder{g}{q^i} \right) \:.
\end{align}

In the standard coordinates we have
\begin{align}
    \set{x^i,x^j} =& \vc P^{ij} \:, \\
    \set{\set{x^i,x^j},x^k} =& \pder{\vc P^{ij}}{x^m} \pder{x^k}{x^n} \vc P^{mn} = \pder{\vc P^{ij}}{x^m} \vc P^{m k} \:,
\end{align}
so the Jacobi identity is given by
\begin{align}
    \pder{\vc P^{ij}}{x^m} \vc P^{m k} + \pder{\vc P^{jk}}{x^m} \vc P^{m i} + \pder{\vc P^{ki}}{x^m} \vc P^{m j} = 0 \:.
\end{align}

\begin{definition}[The Casimir of the Poisson bracket]
    The function $g \in \F M$ is called a Casimir for the given Poisson bracket $\set{\cdot,\cdot}$, if
    \begin{align}
        \set{f,g} = 0 \quad \forall g \in \F M \:.
    \end{align}
\end{definition}