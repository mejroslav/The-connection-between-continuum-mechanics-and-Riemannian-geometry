\chapter*{Conclusion}
\addcontentsline{toc}{chapter}{Conclusion}

The area of geometry in hydrodynamics offers many challenging, unsolved problems. This thesis aimed to explore the Dubrovin-Novikov theorem which shows that for non-degenerate metric the existence of Poisson structure is equivalent to the existence of Levi-Civita connection with zero curvature, therefore connects the Poisson and pseudo-Riemannian geometry. The proof of this theorem in the original paper \cite{Dubrovin-Novikov} was only sketched, so the potential benefit of this thesis was to provide it in perhaps a readable, detailed form. In Chapter 3, we have first verified the tensor structure of the functions in quasi-linear systems of the first order and constructed the Dubrovin-Novikov bracket for systems of hydrodynamic type. Then we restricted ourselves to the one-dimensional case, where we have shown that the functions forming the non-degenerate bracket are in fact inverse metric and Christoffel symbols. Then the proof of the Dubrovin-Novikov theorem was divided into four steps, in which we have 1. connected the skew-symmetry of the bracket with the symmetry of the metric and the compatibility condition, 2. written down the conditions under which the Jacobi identity holds, 3. verified the zero torsion and curvature and 4. indicated the existence of the coordinates in which the bracket is trivial and therefore finished the converse statement. In Chapter 4, we have done a similar analysis for the multi-dimensional bracket, written down the conditions under which the bracket is Poisson, and reformulated the problem of finding the flat coordinates via the obstruction tensors. We end the discussion with the theorem by Mokhov, which states that the existence of flat coordinates is possible if one of the metrics forms nonsingular pairs with all the remaining metrics. We did not explore whether this condition is satisfied in the examples mentioned in Chapter 2.


Nevertheless, the connection between the two geometric structures is still slightly mysterious, as the zero torsion and zero curvature emerge among a huge amount of calculations. Is there something deeper to be identified beyond these equations? 

The assumption which is found to be crucial is the non-degeneracy of a metric. We are convinced that there exist attempts to describe the geometry with degenerate metrics. The problem with hydrodynamical systems is that the functions $g^{ij \alpha}$ are in fact the inverse metrics; therefore we need a theory which stems from the symmetric but degenerate (2,0) tensor. If the metric is degenerate, there is no Levi-Civita connection. There could be several covariant derivatives that annihilate the metric and have zero torsion. Could this be the way of proving the symmetric hyperbolicity for the degenerate systems? At least to the author's knowledge, these questions are still open to inquiry. 

To the inquisitive reader, we further recommend two sources that widen the horizons of current knowledge. For an analysis of two-dimensional systems, see \cite{Savoldi}, and for more of the one-dimensional degenerate brackets, see \cite{savoldi2014deformations}. 
