\chapter*{Conclusion}
\addcontentsline{toc}{chapter}{Conclusion}

The area of geometry in hydrodynamics is offering many challenging, unsolved problems. In this thesis we restricted ourselves to the one-dimensional Poisson brackets where the two geometrical structures meet. Yet the connection is still a little bit mysterious, as the zero torsion and zero curvature emerge among a huge amount of calculations. Is there something deeper beyond these equations? 

The assumption which is found to be crucial is the non-degeneracy of a metric. We are convinced that there are attempts to describe the geometry with degenerated metrics. The problem with hydrodynamical systems is that the functions $g^{ij \alpha}$ are in fact the inverse metrics, therefore we need a theory which is built from the symmetric but degenerate (2,0) tensor. If the metric is degenerate, there is no Levi-Civita connection. In fact, there could be several covariant derivatives that annihilate the metric and have zero torsion. 