\chapter*{List of symbols}
\todo[inline]{List of symbols.}
\addcontentsline{toc}{chapter}{List of symbols}
\noindent\vspace{-2\baselineskip}
{\renewcommand{\arraystretch}{1.3}
\begin{table}[h!]
    \begin{tabular}{c|p{0.7\linewidth}}
    \end{tabular}
\end{table}
}

\chapter*{Introduction}
\addcontentsline{toc}{chapter}{Introduction}

Geometrical formalism plays essential part in the modern classical mechanics and there are several attempts to look at other fields of physics in geometrical way, especially at hydrodynamics and thermodynamics. The Hamiltonian formalism and Poisson brackets of infinite-dimensional dynamical systems have taken hold in the mathematics community since the early seventies.

In this thesis we examine the systems of hydrodynamic type which appear in a wide range of applications, not only in hydrodynamics, but also in chemical kinetics, gas dynamics and general relativity \cite{Savoldi}. These systems are quasi-linear partial differential equations of first-order
\begin{align}
    \pder{u^i}{t} = f^{i \alpha}_j(u) \pder{u^j}{x^\alpha} \:, \quad u^i=u^i(t, x^1, \cdots, x^d) \:, \quad i = 1, \cdots, N, \quad \alpha = 1, \cdots, d \:,
\end{align}
where the standard summation rule over repeated indices is assumed. On the space of fields $u^1, \cdots, u^N$ one can build a Poisson structure via the so called bracket of hydrodynamic type, or a Dubrovin-Novikov bracket of the form
\begin{align}
    \set{u^i(x),u^j(y)} = g^{ij \alpha}(u(x)) \partial_\alpha \delta(x-y) + b^{ij \alpha}_k (u(x)) \pder{u^k}{x^\alpha} \delta(x-y) \:,
\end{align}
and more generally for arbitrary pair of functionals $J_1 = \int j_1(u,u_x,u_{xx},\cdots) \, \D x$, $J_2 = \int j_2(u,u_x,u_{xx},\cdots) \, \D x$
\begin{align}
    \set{J_1, J_2} = \int \fder{I_1}{u^i(x)} P^{ij} \fder{I_2}{u^j(x)} \, \D x \:, \quad P^{ij} = g^{ij \alpha}(u(x)) \der{}{x^\alpha} + b^{ij \alpha}_k(u(x)) \pder{u^k}{x^\alpha} \:,
\end{align}
where the first-order differential operator $P^{ij}$ is called a Hamiltonian operator or a Poisson bivector of hydrodynamic type. 


The one-dimensional case is well explored. It turns out that there is a (one would say even magical) connection between the Poisson and pseudo-Riemannian geometry. The crucial is the theorem by B. A. Dubrovin and S. P. Novikov which can be found in \cite{Dubrovin-Novikov}. The potential benefit of this thesis is to provide the proof of this theorem in detailed way due to the large ammount of technical calculations which the author could not find elsewhere. The theorem shows that the functions $g^{ij}$ and $b^{ij}_k$ form a metric and Levi-Civita connection with zero curvature, respectively. This enables writing one-dimensional systems of hydrodynamic type as
\begin{align}
    \pder{u^i}{t} = g ^{i s} (\nabla_s \nabla_j h ) \pder{u^j}{x} \:,
\end{align}
where $h$ represents the hamiltonian of the system. This forms an alternative proof of hyperbolicity, which was done by M. Pavelka, I. Peshkov and V. Klika in 2020 \cite{Pavelka}. Hyperbolicity of evolutionary equations provides local existence of solutions and also the numerical resolution which guaranties that the small perturbations of the solution due to the discretization errors do not amplify exponentially in time.

The multi-dimensional case is much more complicated and up to these days it is far from beeing completely solved.

The organisation of this thesis is the following. Chapters 1 and 2 are very consise and remind some basic facts about hydrodynamical systems and geometry. In Chapter 3 one-dimensional case is handled and the proof of the Dubrovin-Novikov theorem is done. In Chapter 4 we investigate the differences in the multi-dimensional case and end the discussion with the Mokhov classification theorem.


