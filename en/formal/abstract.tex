%%% A template for a simple PDF/A file like a stand-alone abstract of the thesis.

\documentclass[12pt]{report}

\usepackage[a4paper, hmargin=1in, vmargin=1in]{geometry}
\usepackage[a-2u]{pdfx}
\usepackage[utf8]{inputenc}
\usepackage[T1]{fontenc}
\usepackage{lmodern}
\usepackage{textcomp}

\begin{document}

%% Do not forget to edit abstract.xmpdata.

Zabýváme se systémy kvazilineárních parciálních diferenciálních rovnic hydrodynamického typu. Tyto rovnice se nejčastěji objevují v hydrodynamice a mechanice kontinua, ale jsou aplikovatelné i jinde. Při studiu těchto systémů se objevuje průnik poissonovské a riemannovské geometrie. Poissonova závorka je dána funkcemi, které mají strukturu metrik a Christoffelových symbolů. Jestliže je metrika nedegenerovaná, existence Poissonovy struktury je ekvivalentní existenci ploché metriky a Levi-Civitovy kovariantní derivace s nulovou křivostí. Navíc lze nalézt speciální souřadnice, ve kterých je závorka triviální. Tento výsledek pochází z osmdesátých let z prací Dubrovina a Novikova pro jednodimenzionální případ a později byl rozšířen do více dimenzí. V této práci poskytujeme důkaz Dubrovinova-Novikova teorému, který je v původním článku pouze naznačen. Předkládáme také přehled současných výsledků v multidimenzionálním případě, kde se teorie mnohem více komplikuje. Konkrétně, rozebíráme možnost nalezení plochých souřadnic v souvislosti s kompatibilitou závorek. Riemmanovský charakter hamiltonovských rovnic hydrodynamického typu by mohl najít využití při důkazu symetrické hyperbolicity, i přestože se rovnice nedají zapsat v konzervativním tvaru.

\end{document}
