%%% The main file. It contains definitions of basic parameters and includes all other parts.

%% Settings for single-side (simplex) printing
% Margins: left 40mm, right 25mm, top and bottom 25mm
% (but beware, LaTeX adds 1in implicitly)
\documentclass[12pt,a4paper]{report}
\setlength\textwidth{145mm}
\setlength\textheight{247mm}
\setlength\oddsidemargin{15mm}
\setlength\evensidemargin{15mm}
\setlength\topmargin{0mm}
\setlength\headsep{0mm}
\setlength\headheight{0mm}
% \openright makes the following text appear on a right-hand page
\let\openright=\clearpage

%% Settings for two-sided (duplex) printing
% \documentclass[12pt,a4paper,twoside,openright]{report}
% \setlength\textwidth{145mm}
% \setlength\textheight{247mm}
% \setlength\oddsidemargin{14.2mm}
% \setlength\evensidemargin{0mm}
% \setlength\topmargin{0mm}
% \setlength\headsep{0mm}
% \setlength\headheight{0mm}
% \let\openright=\cleardoublepage

%% Generate PDF/A-2u
\usepackage[a-2u]{pdfx}

%% Character encoding: usually latin2, cp1250 or utf8:
\usepackage[utf8]{inputenc}
\usepackage[T1]{fontenc}

%% Prefer Latin Modern fonts
\usepackage{lmodern}

%% Further useful packages (included in most LaTeX distributions)
\usepackage{amsmath}        % extensions for typesetting of math
\usepackage{amsfonts}       % math fonts
\usepackage{amsthm}         % theorems, definitions, etc.
\usepackage{bbding}         % various symbols (squares, asterisks, scissors, ...)
\usepackage{bm}             % boldface symbols (\bm)
\usepackage{graphicx}       % embedding of pictures
\usepackage{fancyvrb}       % improved verbatim environment
\usepackage{natbib}         % citation style AUTHOR (YEAR), or AUTHOR [NUMBER]
\usepackage[nottoc]{tocbibind} % makes sure that bibliography and the lists
			    % of figures/tables are included in the table
			    % of contents
\usepackage{dcolumn}        % improved alignment of table columns
\usepackage{booktabs}       % improved horizontal lines in tables
\usepackage{paralist}       % improved enumerate and itemize
\usepackage{xcolor}         % typesetting in color
\usepackage{mathrsfs}		% mathscr
\usepackage[h]{esvect}		% vv
\usepackage{makecell}		% makecell
\usepackage{calc}			% widthof
\usepackage{tensor}			% writing tensors with ordered indices 

\usepackage{todonotes}		% for todo notes
\usepackage{caption}
\usepackage[english]{varioref} % More descriptive referencing


% Experimentation with different fonts
%\usepackage{libertinus} %\usepackage[expert]{mathdesign}

%%% Basic information on the thesis
% Thesis title in English (exactly as in the formal assignment)
\def\ThesisTitle{The Connection between Continuum Mechanics and Riemannian Geometry}

% Author of the thesis
\def\ThesisAuthor{Miroslav Burýšek}

% Year when the thesis is submitted
\def\YearSubmitted{2022}

% Name of the department or institute, where the work was officially assigned
% (according to the Organizational Structure of MFF UK in English,
% or a full name of a department outside MFF)
\def\Department{Mathematical Institute, Charles University, Prague}

% Is it a department (katedra), or an institute (ústav)?
\def\DeptType{Institute}

% Thesis supervisor: name, surname and titles
\def\Supervisor{RNDr. Michal Pavelka, Ph.D.}

% Supervisor's department (again according to Organizational structure of MFF)
\def\SupervisorsDepartment{Mathematical Institute, Charles University, Prague}

% Study programme and specialization
\def\StudyProgramme{Physics}
\def\StudyBranch{FOF}

% An optional dedication: you can thank whomever you wish (your supervisor,
% consultant, a person who lent the software, etc.)
\def\Dedication{%
First of all, I would like to sincerely thank my supervisor RNDr. Michal Pavelka, Ph.D. Not only for introducing me to a topic I was not familiar with and the long sessions during which we overcame numerous difficulties to finish the core of the main proof in this thesis, but also for his approach to students and colleagues which, after all, I find to be the most valuable thing in science.

I have to express my gratitude to my friends who helped me finish this thesis, especially Michal Grňo for discussions about the mathematical ideas, Rozka Hájková for the emmotional support throughout my whole bachelor studies, and Josef Lamken for helping me with English.

Finally, I would like to thank my family and all the people I am surrounded by, who inspire me every day. I am truly standing on the shoulders of giants.
}

% Abstract (recommended length around 80-200 words; this is not a copy of your thesis assignment!)
\def\Abstract{%
We investigate the systems of quasi-linear partial differential equations of hydrodynamic type. These equations occur mainly in hydrodynamics and continuum mechanics, but they arise in other various applications. In the study of such systems, one finds an intersection of Poisson and pseudo-Riemannian geometry. The Poisson bracket is determined by functions that turn out to be metrics and Christoffel symbols. If the metric is non-degenerate, the existence of Poisson structure is equivalent to the existence of flat metric and Levi-Civita covariant derivative with zero curvature. Moreover, one can find special flat coordinates where the bracket is trivial. This result was found in the eighties by Dubrovin and Novikov for the one-dimensional case and later on extended to more dimensions. In this thesis we provide the proof of the Dubrovin-Novikov theorem, which was only sketched in the original paper. We also conducted an overview of current knowledge in the multi-dimensional case, where the theory gets much more complicated. In particular, the link between the compatible brackets and the possibility of finding flat coordinates is discussed. The Riemannian character of the Hamiltonian equations of hydrodynamic type
can be used to prove their symmetric hyperbolicity, even when the
equations are not in the conservative form.
}

% 3 to 5 keywords (recommended), each enclosed in curly braces
\def\Keywords{%
{hydrodynamics}, {Dubrovin-Novikov theorem}, {Poisson bracket}, {pseudo-Riemannian geometry}
}

%% The hyperref package for clickable links in PDF and also for storing
%% metadata to PDF (including the table of contents).
%% Most settings are pre-set by the pdfx package.
\hypersetup{unicode}
% \hypersetup{breaklinks=true} % uncomment this for breaklinks
% Definitions of macros (see description inside)

%%% This file contains definitions of various useful macros and environments %%%
%%% Please add more macros here instead of cluttering other files with them. %%%

%%% Minor tweaks of style

% These macros employ a little dirty trick to convince LaTeX to typeset
% chapter headings sanely, without lots of empty space above them.
% Feel free to ignore.
\makeatletter
\def\@makechapterhead#1{
  {\parindent \z@ \raggedright \normalfont
   \Huge\bfseries \thechapter. #1
   \par\nobreak
   \vskip 20\p@
}}
\def\@makeschapterhead#1{
  {\parindent \z@ \raggedright \normalfont
   \Huge\bfseries #1
   \par\nobreak
   \vskip 20\p@
}}
\makeatother

% This macro defines a chapter, which is not numbered, but is included
% in the table of contents.
\def\chapwithtoc#1{
\chapter*{#1}
\addcontentsline{toc}{chapter}{#1}
}

% Draw black "slugs" whenever a line overflows, so that we can spot it easily.
\overfullrule=1mm

%%% Macros for definitions, theorems, claims, examples, ... (requires amsthm package)

\theoremstyle{plain}
\newtheorem{theorem}{Theorem}
\newtheorem{lemma}[theorem]{Lemma}
\newtheorem{claim}[theorem]{Claim}
\newtheorem{definition}[theorem]{Definition}
\newtheorem{proposition}[theorem]{Proposition}
\newtheorem{example}[theorem]{Example}
\newtheorem{corollary}[theorem]{Corollary}

\theoremstyle{remark}
\newtheorem*{remark}{Remark}


%%% An environment for proofs

\newenvironment{myproof}{
  \par\medskip\noindent
  \textit{Proof}.
}{
\newline
\rightline{$\qedsymbol$}
}

%%% An environment for typesetting of program code and input/output
%%% of programs. (Requires the fancyvrb package -- fancy verbatim.)

\DefineVerbatimEnvironment{code}{Verbatim}{fontsize=\small, frame=single}

%%% The field of all real and natural numbers
\newcommand{\C}{\mathbb{C}}
\newcommand{\K}{\mathbb{K}}
\newcommand{\R}{\mathbb{R}}
\newcommand{\Z}{\mathbb{Z}}
\newcommand{\N}{\mathbb{N}}

%%% Useful operators for statistics and probability
\DeclareMathOperator{\pr}{\textsf{P}}
\DeclareMathOperator{\E}{\textsf{E}\,}
\DeclareMathOperator{\var}{\textrm{var}}
\DeclareMathOperator{\sd}{\textrm{sd}}

%%% Transposition of a vector/matrix
\newcommand{\T}[1]{#1^\top}

%%% Various math goodies
\newcommand{\goto}{\rightarrow}
\newcommand{\gotop}{\stackrel{P}{\longrightarrow}}
\newcommand{\maon}[1]{o(n^{#1})}
\newcommand{\abs}[1]{\left|{#1}\right|}
\newcommand{\isqr}[1]{\frac{1}{\sqrt{#1}}}

%%% Various table goodies
\newcommand{\pulrad}[1]{\raisebox{1.5ex}[0pt]{#1}}
\newcommand{\mc}[1]{\multicolumn{1}{c}{#1}}


%%% Custom commands
\renewcommand{\d}[1]{\, \mathrm{d}#1 \,}
\renewcommand{\i}{\mathrm{i}}
\newcommand{\Cont}{C}
\newcommand{\Hilb}{\mathcal{H}}
\newcommand{\Schwa}{\mathcal{S}}
\newcommand{\Domain}{\mathrm{D}}
\newcommand{\Fourier}{\mathcal{F}}
\newcommand{\Nbhood}{\mathcal{U}}
\newcommand{\Hf}{\mathscr{H}}
\newcommand{\Af}{\mathscr{A}}
\newcommand{\ve}{\varepsilon}
\newcommand{\hypF}{\vphantom{\mathrm{F}}_1\hspace{-0.7pt}\mathrm{F}_{\!1}}

\newcommand{\Sp}{\sigma}
\newcommand{\SpP}{\sigma_{\mathrm{p}}}
\newcommand{\SpAc}{\sigma_{\mathrm{ac}}}
\newcommand{\SpSc}{\sigma_{\mathrm{sc}}}
\newcommand{\SpD}{\sigma_{\mathrm{disc}}}
\newcommand{\SpE}{\sigma_{\mathrm{ess}}}


\def\ph{\phantom}
\def\vph{\vphantom}
\def\hph{\hphantom}
\def\rzw{\mathrlap}
\def\lzw{\mathllap}
\def\czw{\mathclap}
\newcommand*{\mask}[2]{\mathord{\makebox[\widthof{\(#1\)}]{\(#2\)}}}
\newcommand{\scalemath}[2]{\scalebox{#1}{\mbox{\ensuremath{\displaystyle #2}}}}


\newcommand{\coloneqq}{:=}
\newcommand{\eqqcolon}{=:}

\newcommand{\digamma}{\Psi} % haha dobrý

\newcommand\numberthis{\addtocounter{equation}{1}\tag{\theequation}}

%%%%%%%%%%%%%%%%%%%%%%%%%%%%%%%%%%%%%%%%%%%%%%%%%%%%%%%%%%%%%%%
% Mejroslav

%%% My own symbols
\newcommand{\vc}[1]{\boldsymbol{#1}} % vektor
\newcommand{\mat}[1]{\mathbf{#1}} % matice

\newcommand{\norm}[1]{\left \Vert #1 \right \Vert} % norma vektoru
\newcommand{\set}[1]{ \left \lbrace #1 \right \rbrace} % množina
\newcommand{\const}{\mathrm{const}} % konstanta

%Značení derivací a integrálů
\newcommand{\der}[2]{\frac{\mathrm{d}#1}{\mathrm{d}#2}} % obyčejná derivace
\newcommand{\pder}[2]{\frac{\partial #1}{\partial #2}} % parciální derivace
\newcommand{\ppder}[3]{\frac{\partial^2 #1}{\partial #2 \partial #3}} % parciální derivace
\newcommand{\fder}[2]{\frac{\delta #1}{\delta #2}} % funkcionální derivace
\newcommand{\ffder}[3]{\frac{\delta^2 #1}{\delta #2 \delta #3}} % funkcionální derivace
\newcommand{\D}{\mathrm{d} } % integrační znamení
\newcommand{\DD}{\mathrm{D} } % velké déčko


% Značení posloupností, limit a sum
\newcommand{\sequence}[2]{ \left \lbrace #1 \right \rbrace_{#2=1}^\infty} % posloupnost
\newcommand{\sumnorm}[1]{\sum_{#1}^\infty} 
\newcommand{\limplus}[1]{\lim_{#1 \rightarrow + \infty}}
\newcommand{\limminus}[1]{\lim_{#1 \rightarrow - \infty}}


%Značení distribucí
\newcommand{\dual}[2]{\left \langle #1 ,#2 \right \rangle} % dualita
\newcommand{\test}{\mathcal{D}} % prostor testovacích funkcí
\newcommand{\dis}{\mathcal{D'}} % prostor distribucí

\newcommand{\weakstar}{\rightharpoonup^*} % slabá* konvergence 
\newcommand{\supp}{\mathrm{supp}\,} % nosič funkcí

\newcommand{\Dom}{\mathrm{Dom}} % domain
\hyphenation{con-ju-gate}
\hyphenation{Di-rich-let}
\hyphenation{Hamil-to-nian}
\hyphenation{Hil-bert}
\hyphenation{La-place}
\hyphenation{La-placian}
\hyphenation{Le-bes-gue}
\hyphenation{non-di-men-sion-al-i-za-tion}


% Title page and various mandatory informational pages


% TESTING
%\usepackage[inline]{showlabels}  \showlabels[\tiny]{}  % Uncomment to output the content of \label commands to the document where they are used


\begin{document}

%\listoftodos

\include{formal/title}

%%% A page with automatically generated table of contents of the bachelor thesis

\tableofcontents

%%% Each chapter is kept in a separate file
\chapter*{List of symbols}
\addcontentsline{toc}{chapter}{List of symbols}
\noindent\vspace{-2\baselineskip}
{\renewcommand{\arraystretch}{1.3}
\begin{table}[h!]
    \begin{tabular}{c|p{0.7\linewidth}}
        $T^{ab}_{,k}$ & Partial derivative with respect to the field variable $\pder{T^{ab}}{u^k}$. \\
        $\partial_\alpha \delta(x-y)$ & Partial derivative with respect to the spatial variable $\pder{}{x^\alpha} \delta(x-y)$. \\
        $D^\alpha f$ & Derivative with respect to the multiindex $\alpha = (\alpha_1, \cdots, \alpha_k)$.\\
        $u^k_\alpha$ & Partial derivative of the field variable with respect to the spatial variable $\pder{u^k}{x^\alpha}$. \\
        $\fder{I}{u^i(x)}$ & Functional derivative of functional $I$ with respect to $u^i$ in the point $x$. \\
        $M$ & Smooth manifold. \\
        $\F M$ & The space of functions $C^\infty(M, \R)$. \\
        $T_x M$ & The tangent space of the point $x \in M$. \\
        $TM$ & The tangent bundle of $M$. \\
        $T^p_q M$ & The bundle of tensors of rank $p,q$ on $M$. \\
        $\vc A$ & Tensor or tensor field. \\
        $\vc v$ & Vector or vector field. \\
        $\nabla_{\vc a}$ & Covariant derivative in the direction $\vc a$. \\
        $\nabla_j$ & Covariant differential. \\
        $\wedge$ & Wedge product of skew-symmetric forms. \\
        $d_a$ & Exterior derivative of skew-symmetric form. \\
        $\Dom(A)$ & The domain of a functional $A$. \\
        $\delta A(x,h)$ & The Gateaux derivative of $A$ in $x$ with respect to $h$. \\
        $\DD_x A$ & The Fréchet differential of $A$ in $x$. \\
        $\mathcal{D}(\Omega)$, $\mathcal{D}'(\Omega)$ & The space of functions with compact support on $\Omega$ and its dual, the space of distributions. \\
        $\dual{T}{\psi}$ & Duality of functional $T$ and function $\psi$.
    \end{tabular}
\end{table}
}

\chapter*{Introduction}
\addcontentsline{toc}{chapter}{Introduction}

Geometrical formalism plays an essential part in modern classical mechanics and there are several attempts to look at other fields of physics in geometrical way, especially at hydrodynamics and thermodynamics. The Hamiltonian formalism and Poisson brackets of infinite-dimensional dynamical systems have taken hold in the mathematics community since the early seventies \cite{GMD}, \cite{Generic}.

In this thesis we examine the systems of hydrodynamic type which appear in a wide range of applications, not only in hydrodynamics, but also in chemical kinetics, gas dynamics and general relativity \cite{Savoldi}. These systems are quasi-linear partial differential equations of first-order
\begin{align}
    \pder{u^i}{t} = f^{i \alpha}_j(u) \pder{u^j}{x^\alpha} \:, \quad u^i=u^i(t, x^1, \cdots, x^d) \:, \quad i = 1, \cdots, N, \quad \alpha = 1, \cdots, d \:,
\end{align}
where the standard summation rule over repeated indices is assumed. On the space of fields $u^1, \cdots, u^N$ one can build a Poisson structure via the so called bracket of hydrodynamic type, or a Dubrovin-Novikov bracket of the form
\begin{align}
    \set{u^i(x),u^j(y)} = g^{ij \alpha}(u(x)) \partial_\alpha \delta(x-y) + b^{ij \alpha}_k (u(x)) \pder{u^k}{x^\alpha} \delta(x-y) \:,
\end{align}
and more generally for arbitrary pair of functionals $J_1 = \int j_1(u,u_x,u_{xx},\cdots) \, \D x$, $J_2 = \int j_2(u,u_x,u_{xx},\cdots) \, \D x$
\begin{align}
    \set{J_1, J_2} = \int \fder{I_1}{u^i(x)} P^{ij} \fder{I_2}{u^j(x)} \, \D x \:, \quad P^{ij} = g^{ij \alpha}(u(x)) \der{}{x^\alpha} + b^{ij \alpha}_k(u(x)) \pder{u^k}{x^\alpha} \:,
\end{align}
where the first-order differential operator $P^{ij}$ is called a Hamiltonian operator or a Poisson bivector of hydrodynamic type. 


The one-dimensional case is well explored. It turns out that there is a (one would say even magical) connection between the Poisson and pseudo-Riemannian geometry. The crucial is the theorem by B. A. Dubrovin and S. P. Novikov, which can be found in \cite{Dubrovin-Novikov}. The potential benefit of this thesis is to provide the proof of this theorem in a detailed way due to the large ammount of technical calculations which the author could not find elsewhere. The theorem shows that the functions $g^{ij}$ and $b^{ij}_k$ form a metric and Levi-Civita connection with zero curvature, respectively. This enables writing the one-dimensional systems of hydrodynamic type as
\begin{align}
    \pder{u^i}{t} = g ^{i s} (\nabla_s \nabla_j h ) \pder{u^j}{x} \:,
\end{align}
where $h$ represents the hamiltonian of the system. This forms an alternative proof of hyperbolicity, which was done by M. Pavelka, I. Peshkov and V. Klika in 2020 \cite{Pavelka}. Hyperbolicity of evolutionary equations provides local existence of solutions and also the numerical resolution, which guarantees that the small perturbations of the solution due to the discretization errors do not amplify exponentially in time.

The multi-dimensional case is much more complicated, and it is far from being completely solved up to these days,.

The organization of this thesis is the following. Chapters 1 and 2 are very concise and remind some basic facts about hydrodynamical systems and geometry. In Chapter 3, the one-dimensional case is handled, and the proof of the Dubrovin-Novikov theorem is done. In Chapter 4, we investigate the differences in the multi-dimensional case and end the discussion with the Mokhov classification theorem.



\chapter{Geometrical preliminarities}

The goal of this chapter is to briefly revise some basic definitions and well-known facts about the pseudo-Riemannian manifolds and the Poisson structure. This important part of geometry and topology is covered by hundreds of great books. Since for the rest of the thesis we shall study mostly the work of B. A. Dubrovin and S. P. Novikov, we will recommend three monographs on modern geometry written by our heroes and A. T. Fomenko \cite{DN-Geometry1}. Another exquisite book, which has become almost cult among Czechs and Slovaks and which is worth all of the reader's attention, is the book of M.Fecko \cite{Fecko}.

Throughout this chapter $M$ represents a finite-dimensional smooth manifold, $\mathcal F M$ the space of smooth functions $M \to \R$, $T_x M$ the tangent space in $x \in M$, $T M$ the tangent bundle on $M$, $T^p_q M$ the bundle of tensors of rank $(p,q)$ on $M$. We also automatically assume the so called Einstein summation over contravariant and covariant indices. Vectors, covectors and tensors of arbitrary rank are denoted (only in this chapter) by bold symbols, for the purpose of higliting the difference between the abstract indices (denoted by normal letters $i,j,\cdots$) and vector indices (denoted by bold letters $\vc A$, $\vc b$, $\cdots$).

\section{Pseudo-Riemannian manifolds}

\begin{definition}[Metric tensor]
    A metric tensor $\vc g_{ij} \in \T^0_2 M$ is symmetric and non-degenerate, i.e.
    \begin{align}
        \vc g_{ij} =&\, \vc g_{ji} \quad \textbf{(symmetry)}\:, \\ \det \vc g_{ij} \neq&\, 0 \quad \textbf{(non-degeneracy)}\:.
    \end{align}
\end{definition}

If the metric $\vc g_{ij}$ is positive quadratic form, the manifold $M$ is said to be Riemannian. If it is indefinite, the manifold is said to be pseudo-Riemannian.

Non-degeneracy of the metric $\vc g_{ij}$ allows us to define the inverse metric $\vc g^{ij}$ defined by the property
\begin{align}
    \vc g^{ij} \vc g_{jk} = \vc \delta^i_k \:.
\end{align}

A metric structure is important mainly for three things: 1. it allows to measure lenghts of vectors and angles between them, 2. together with its inverse allows to upper and lower indices of tensors, 3. it enables to build the parallel transport of tensors from one point to another and the concept of covariant derivative.

\begin{definition}[Covariant derivative]
    Let $\vc a \in TM$. A covariant derivative $\nabla_{\vc a}$ in the direction $\vc a$ is an operation on $T^p_q M$ satisfying these properties:
    \begin{align}
        &\nabla_{\vc a} (\vc A + r \vc B) = \nabla_{\vc a} \vc A + r \nabla_{\vc a} \vc B \quad \forall \vc A,\vc B \in T^k_l M \:,\: r \in \R \:, \quad \textbf{(linearity)} \\
        &\nabla_{\vc a} (\vc A \otimes \vc B) = (\nabla_{\vc a} \vc A) \otimes \vc B + \vc A \otimes (\nabla_{\vc a} \vc B) \quad \forall \vc A,\vc B \in T^k_l M \:, \quad \textbf{(Leibniz rule)} \\
        &\nabla_{\vc a} (\mathsf{C} \vc A) = \mathsf{C} (\nabla_{\vc a} \vc A) \quad \forall \vc A \in T^k_l M \:, \quad \textbf{(commutation with contraction)} \\
        &\nabla_{\vc a} f = \vc a[f] = \vc a \cdot \D f \quad \forall f \in \mathcal F M \:, \quad \textbf{(gradient on scalars)}
    \end{align}
    where $\mathsf{C}$ is the contraction operation.
\end{definition}
(Note that the covariant derivative $\nabla_{\vc a}$ has the lower index bold, as it refers to the vector $\vc a$.)

If we multiply the vector field $\vc a \in T M$ by some smooth function $f \in \mathcal F M$, then
\begin{align}
    \nabla_{f \vc a} \vc A = f \nabla_{\vc a} \vc A \quad \forall \vc A \in T^p_q M \quad \textbf{(ultralocality)}\:.
\end{align}
Therefore we can introduce the covariant differential $\nabla_j$ (with the normal lower index) as the operation from $\nabla : \T^p_q \to \T^p_{q+1}$ defined by the property
\begin{align}
    \nabla_{\vc a} \vc A = \vc a \cdot \nabla \vc A \:,
\end{align} 
or equivalently in the index notation
\begin{align}
    \nabla_{\vc a} \vc A^{ij\cdots}_{kl\cdots} = \vc a^k \nabla_k \vc A^{ij\cdots}_{kl\cdots} \:.
\end{align}

Any covariant diferential $\nabla_i$ can be expressed via the partial derivative with respect to the coordinates $(x^j)$ as
\begin{align}
    \nabla_i \vc A^{ab \dots}_{cd \dots} = \vc A^{ab \dots}_{cd \dots, i} + \vc \Gamma^{a}_{im} \vc A^{mb \dots}_{cd \dots} + \vc \Gamma^{b}_{im} \vc A^{am \dots}_{cd \dots} + \cdots
    - \vc \Gamma^{m}_{ic} \vc A^{ab \dots}_{md \dots} - \vc \Gamma^{m}_{id} \vc A^{ab \dots}_{cm \dots} - \cdots \:,
\end{align}
where $\vc \Gamma^{a}_{bc}$ are the affine connection coefficients (Christoffel symbols). Given two coordinate systems $(x^i)$ and $(\tilde x^a)$, the Christoffel symbols transform as
\begin{align}
    \vc {\tilde\Gamma}^{a}_{bc} = \vc {\Gamma}^i_{jk} \pder{x^j}{\tilde  x^b} \pder{x^k}{\tilde x^c}  \pder{\tilde x^a}{x^i} + \pder{\tilde x^a}{x^m} \ppder{x^m}{\tilde x^b}{\tilde x^c} \:.
\end{align}

Compared to the partial derivative in flat space, the covariant derivative brings several complicacies with it. The information on how much it differs from ordinary derivative is encoded in tensors of torsion and curvature. 

\begin{definition}[Torsion and curvature tensors]
    Let $[\cdot,\cdot]$ be the Lie bracket of two vector fields.

    \begin{enumerate}
        \item The torsion tensor $\vc T^{k}_{mn}$ is defined by
        \begin{align}
            \vc T^k_{mn} \vc a^m \vc b^n = \vc a^m \nabla_m \vc b^k - \vc b^m \nabla_m \vc a^k - [\vc a, \vc b]^k \quad \forall \vc a, \vc b \in TM \:.
        \end{align}
        \item The Riemann curvature tensor $\vc R\indices{_i_j^k_l}$ is defined by
        \begin{align}
            \vc R\indices{_i_j^k_l} \vc a^i \vc b^j \vc c^l = [\nabla_{\vc a} \nabla_{\vc b} - \nabla_{\vc b} \nabla_{\vc a} - \nabla_{[\vc a, \vc b]}] \vc c^k \quad \forall \vc a, \vc b, \vc c \in TM \:,
        \end{align}
    \end{enumerate}
\end{definition}


The torsion tensor represents the commutator of the second covariant diferentials applied on functions
\begin{align}
    [\nabla_i \nabla_j - \nabla_j \nabla_i] f = - \vc T^{m}_{ij} \D_m f \quad \forall f \in \mathcal F M \:.
\end{align}
and Riemann tensor represents the commutator on vectors
\begin{align}
    [\nabla_i \nabla_j - \nabla_j \nabla_i] \vc a^k =  - \vc T_{ij}^m \nabla_m \vc a^k + \vc R\indices{_i_j^k_m} \vc a^m \:. 
\end{align}

\begin{proposition}[Zero torsion and symmetry of the Christoffel symbols]
    The torsion tensor $\vc T^i_{jk}$ of the covariant derivative $\nabla_i$ is zero if and only if the corresponding Christoffel symbols are symmetric in the lower indices, i.e. $\vc \Gamma^i_{jk} = \vc \Gamma^i_{kj}$.
\end{proposition}

Among a whole class of covariant derivatives, there is one type that is used most often: the Levi-Civita derivative.

\begin{definition}[Levi-Civita covariant derivative]
    Let $M$ be a pseudo-Riemannian manifold with the metric $\vc g_{ij}$. Levi-Civita covariant derivative $\vc \nabla_k$ is defined by the following properties:
    \begin{align}
        \vc \nabla_k \vc g_{ij} = \vc g_{ij,k} - \vc \Gamma^m_{ki} \vc g_{mj} - \vc \Gamma^m_{kj} \vc g_{im}  =& 0 \quad \textbf{(compatibility with metric)} \:, \\
        \vc T^{i}_{jk} =& 0 \quad \textbf{(zero torsion)} \:,
    \end{align}
    where $\vc T^i_{jk}$ is the torsion corresponding to $\vc \nabla_j$.
\end{definition}

If $\vc g_{ij}$ is non-degenerate, i.e. $\det \vc g_{ij} \neq 0$, then the Levi-Civita covariant derivative is always defined and its Christoffel symbols are given by
\begin{align}
    \vc \Gamma^m_{ij} =\frac{1}{2} \vc g^{mn} \left( \vc g_{jn,i} + \vc g_{ni,j} - \vc g_{ij,n} \right) \:.
\end{align}

The covariant derivative compatible with the metric commutes with the operation of lowering and uppering indices of a tensor. For example:
\begin{align}
    \nabla_i \vc T^{ab} = \nabla_i (\vc g^{am} \vc T^b_m) = \vc g^{am} \nabla_i \vc T^b_m \:.
\end{align}

The Riemann curvature tensor for the Levi-Civita covariant derivative is
\begin{align}
    \vc R\indices{_i_j^k_l} = \vc \Gamma^k_{jl,i} - \vc \Gamma^k_{il,j} + \vc \Gamma^k_{im} \vc \Gamma^m_{jl} - \vc \Gamma^k_{jm} \vc \Gamma^m_{il} \:.
\end{align}

\section{Poisson brackets}

Poisson brackets and symplectic geometry play the essential role in the Hamiltonian mechanics and can be seen from many different perspectives. An inspirable book on this topic is e.g. \cite{Marsden}.  

\begin{definition}[Poisson bracket]
    A Poisson bracket is a bilinear operation $\set{\cdot, \cdot} : \F M \times \F M \to \F M$ which satisfies
    \begin{align}
        &\set{f,g} = -\set{g,f} \quad \textbf{(skew-symmetry)} \:, \\
        &\set{fg,h} = f \set{g,h} + g \set{f,h} \quad \textbf{(Leibniz identity)} \:, \\
        &\set{\set{f,g},h} + \set{\set{g,h},f} + \set{\set{h,f},g} = 0 \quad \textbf{(Jacobi identity)} \:.
    \end{align}
\end{definition}

\begin{definition}[Symplectic 2-form]
    A skew-symmetric 2-form $\vc \omega \in \T^0_2 M$ is said to be symplectic if it is closed and nondegenerate, i.e.
    \begin{align}
        &\D \vc \omega = 0 \quad \textbf{(closedness)} \:, \\
        &\det \vc \omega \neq 0 \quad \textbf{(nondegeneracy)}\:.
    \end{align}
    A pair $(M,\vc \omega)$ is called sympletic manifold.
\end{definition}

Since in odd dimension every skew-symmetric 2-form is degenerate, the dimension $m$ of the symplectic manifold $M$ must be even number. The following statement holds.

\begin{theorem}[Darboux theorem - symplectic forms formulation]
    Let $(M,\vc \omega)$ be a $2n$-dimensional symplectic manifold and $x \in M$. Then there is a neighbourhood $U(x)$ and a coordinate system $(p_1,\cdots,p_n,q^1,\cdots,q^n)$ such that $\vc \omega$ has the form
    \begin{align}
        \vc \omega = \D p_a \wedge \D q^a \quad \text{on } U(x) \:. \label{eq:standard form}
    \end{align}
\end{theorem}
We refer to this coordinate system as to the standard coordinates \\$(x^1,\cdots, x^{2n}) = (p_1,\cdots,p_n,q^1,\cdots,q^n)$ and to the form \eqref{eq:standard form} as to the standard form of $\vc \omega$.

\begin{definition}[Poisson bivector]
    Let $(M,\vc \omega)$ be a symplectic manifold.
    The Poisson bivector $\vc P \in \T^2_0 M$ is the inverse of $\vc \omega$, i.e.
    \begin{align}
        \vc P^{ij} \vc \omega_{jk} = \vc \delta^i_k \:.
    \end{align}
\end{definition}
Thus a symplectic manifold $(M,\vc \omega)$ has its associated Poisson bracket given by
\begin{align}
    \set{f,g} \coloneqq \vc P (\D f, \D g) = \vc P^{ab} \D_a f \D_b g \quad \forall f,g \in \F M \:. 
\end{align}
In the standard coordinates
\begin{align}
    \set{f,g} = \sum_{i=1}^n \left( \pder{f}{q^i} \pder{g}{p_i} - \pder{f}{p_i} \pder{g}{q^i} \right) \:.
\end{align}

In the standard coordinates we have
\begin{align}
    \set{x^i,x^j} =& \vc P^{ij} \:, \\
    \set{\set{x^i,x^j},x^k} =& \pder{\vc P^{ij}}{x^m} \pder{x^k}{x^n} \vc P^{mn} = \pder{\vc P^{ij}}{x^m} \vc P^{m k} \:,
\end{align}
so the Jacobi identity is given by
\begin{align}
    \pder{\vc P^{ij}}{x^m} \vc P^{m k} + \pder{\vc P^{jk}}{x^m} \vc P^{m i} + \pder{\vc P^{ki}}{x^m} \vc P^{m j} = 0 \:.
\end{align}

\begin{definition}[The Casimir of the Poisson bracket]
    The function $g \in \F M$ is called a Casimir for the given Poisson bracket $\set{\cdot,\cdot}$, if
    \begin{align}
        \set{f,g} = 0 \quad \forall g \in \F M \:.
    \end{align}
\end{definition}
\chapter{Geometry in hydrodynamics}

In this chapter we shall illustrate the systems of hydrodynamic type with examples.
For the self-consistency of the Chapter 3 we come to the definition once again (Definitions \vref{def:hamiltonian-functional} and \vref{def:hamiltonian-system}), but for now let us just mention that we operate with the systems of quasi-linear partial differential equations in the form
\begin{align}
    \pder{u^i}{t} = \left[ g^{ij \alpha} (u) \ppder{h(u)}{u^j}{u^k} + b_k^{ij \alpha} (u) \pder{h(u)}{u^j} \right] \pder{u^k}{x^\alpha} \:,
\end{align}
where $u^i = u^i(t,x^1, \cdots, x^N)$ are the field variables and $h(u)$ is the density of the Hamiltonian functional
\begin{align}
    H = \int h(u) \, \D x \:.
\end{align}
We refer to the coefficients $g^{ij \alpha}(u)$ as to the metrics and to $b^{ij \alpha}_k(u)$ as to the connections. If the metric $g^{ij \alpha}$ is nondegenerate, i.e. $\det g^{ij} \neq 0$, then we can define Christoffel symbols by the standard relation \eqref{eq:Levi-Civita-gamma} and they are connected to the functions $b^{ij \alpha}_k$ via
\begin{align}
    b^{ij \alpha}_k = - g^{i s \alpha} \Gamma^{j \alpha}_{sk} \:.
\end{align}

All the examples can be found in \cite{Pavelka}.

\begin{example}[One-dimensional isothermal system]
    The simplest case is the one-dimensional isothermal flow where the Hamiltonian reads as
    \begin{align}
        H = \int h(\rho, m) \, \D x \:,
    \end{align}
    where $\rho$ represents the matter density and $m$ the momentum density. 
    The field variables are $(u^1,u^2) \coloneqq (\rho,m)$.
    The evolution equations (in the Eulerian frame) are
    \begin{align}
        \pder{\rho}{t} =& - \pder{}{x} \left( \rho \pder{h}{m} \right) \:, \\
        \pder{m}{t} =& - \pder{}{x} \left( \rho \pder{h}{\rho} + m \pder{h}{m} \right) \:.
    \end{align}
    The metric $g^{ij}$ is non-degenerate but indefinite and has the form
    \begin{align}
        g^{ij} = -\begin{pmatrix}
            0 &  \rho \\
            \rho &  2 m
        \end{pmatrix} \:,
    \end{align}
    and the connections are zero except for
    \begin{align}
        \Gamma^{m}_{\rho \rho} = \frac{2m}{\rho^2} \:, \quad \Gamma^m_{\rho m} = \Gamma^m_{m \rho} = - \frac{1}{\rho} \:.
    \end{align}
    therefore
    \begin{align}
        b^{\rho m}_\rho = -1 \:, \quad b^{m m}_m = -1 \:,
    \end{align}
    The Christoffel symbols are symmetric in the lower indices, therefore the torsion of the corresponding covariant derivative is zero. One can also easily calculate that the Riemann curvature tensor is zero.
\end{example}

\begin{example}[General one-dimensional system]
    If we include only the entropy $S$ to the system, the metric $g^{ij}$ becomes degenerate. This can be overcome by adding the entropy flux $w$. This leads to the field variables $(u^1,u^2,u^3,u^4) = (\rho,m,s,w) $. The metric has the form
    \begin{align}
        g^{ij} = -\begin{pmatrix}
            0 & \rho & 0 & 0 \\
            \rho & 2m & s & w \\
            0 & s & 0 & 1 \\
            0 & w & 1 & 0
        \end{pmatrix} \:,
    \end{align}
    which is again symmetric, indefinite and non-degenerate.
\end{example}

\begin{example}[Three-dimensional isothermal system]
    In three dimensions, the field variables are $(u^1,u^2,u^3,u^4) = (\rho, m_x,m_y,m_z)$ and there are three metrics of the form
    \begin{align}
        g^{ij x} = -\begin{pmatrix}
            0 &  \rho & 0 & 0 \\
            \rho &  2 m_x &  m_y & m_z \\
            0 & m_y & 0 & 0 \\
            0 & m_z & 0 & 0
        \end{pmatrix}  \:, \\
        g^{ij y} = -\begin{pmatrix}
            0 & 0 & \rho & 0 \\
            0 & 0 & m_x & 0\\
            \rho & m_x & 2m_y & m_z \\
            0 & 0 & m_z & 0
        \end{pmatrix}  \:, \\
        g^{ij z} = -\begin{pmatrix}
            0 & 0 & 0 & \rho \\
            0 & 0 & 0 & m_x \\
            0 & 0 & 0 & m_y \\
            \rho & m_x & m_y & 2 m_z
        \end{pmatrix}  \:.
    \end{align}
    All of them are degenerate. Moreover, arbitrary linear combination
    \begin{align}
        c_x g^{ij x} + c_y g^{ij y} + c_z g^{ij z}
    \end{align}
    is again degenerate, which makes it impossible to define Levi-Civita connection.
\end{example}



\section{Hyperbolicity}




\chapter{Systems of hydrodynamic type: one-dimensional case}

In this chapter we prove the theorem of the connection between the hydrodynamic Poisson bracket and the metrics found in the homogeneous system of hydrodynamic type. The original result was given in 1983 by B. A. Dubrovin and S. P. Novikov and can be found in \cite{Dubrovin-Novikov}. Since then, many authors have contributed to the further development of this theory, e.g., Grinberg in \cite{Grinberg}, Ferapontov in \cite{Ferapontov} and Mochov in \cite{Mochov}.
The proof itself and the preceding lemmas are based on a considerable amount of technical computations which we shall clarify in detail to offer the reader a comprehensive understanding of the chapter in question.


\section{Systems of hydrodynamic type}

\begin{definition}[Homogeneous system of hydrodynamic type]
    A homogeneous system of hydrodynamic type is an equation of the form
    \begin{align}
        \label{eq:h-system}
        \pder{u^i}{t} = f_j^{i \alpha} (u) \pder{u^j}{x^\alpha} \:,
    \end{align}
    where $u^i(t,x^\alpha)$ are unknown functions and $i = 1, \dots , N$, $\alpha = 1, \dots , d$.
\end{definition}

Sometimes we refer to the number of equations $N$ as the dimension of the hydrodynamical phase space and to the number of spatial variables $d$ as the dimension of the configuration space.

It was Riemann who noticed that the functions $f^{i \alpha}_j$ in \eqref{eq:h-system} are in fact tensors \cite{Dubrovin-Novikov}.

\begin{proposition}[Transformation of $f^{i \alpha}_j$] 
    \label{prop:transformace-A} 
    Under smooth change of variables $u^i \mapsto v^a$ of the form
    \begin{align}
        u^i = u^i(v^1,\cdots, v^N) \:,
    \end{align}
    the functions $f^{i\alpha}_{j}$ transform for each $\alpha$ according to the tensor law
    \begin{align}
        f^{a \alpha}_{b}(v) = \pder{v^a}{u^i} \pder{u^j}{v^b} f^{i \alpha}_j (u) \:.
    \end{align}
\end{proposition}
\begin{proof}
    Using a direct calculation, we obtain
    \begin{align}
        \pder{u^i}{t}(v) = \pder{u^i}{v^a} \pder{v^a}{t} \:, \\
        \pder{u^j}{x^\alpha}(v) = \pder{u^j}{v^b} \pder{v^b}{x^\alpha}\:.
    \end{align}
    Hence
    \begin{align}
        \pder{u^i}{v^a} \pder{v^a}{t} = \pder{u^j}{v^b} f^{i \alpha}_j (v(u)) \pder{v^b}{x^\alpha} \:,
    \end{align}
    therefore
    \begin{align}
        \pder{v^a}{t} = \underbrace{\pder{v^a}{u^i} \pder{u^j}{v^b} f^{i \alpha}_j(v(u))}_{f^{a \alpha}_b(v)} \pder{v^b}{x^\alpha} \:.
    \end{align}
\end{proof}

According to the previous lemma if we denote the space of points $(t,x^\alpha) \subset \R \times \R^N$ as some manifold $M$, then the unknown variables $u^i(t,x^\alpha)$ can be seen as the local coordinates of that manifold. The functions $f^{i \alpha}_j$ are transformed as tensors of the type $(1,1)$ in the $i$ and $j$ indices.

In hydrodynamics, the functions $f^{i \alpha}_j$ are of a special form given by the energy of the system. Such systems will be the object of our investigation, and we shall introduce a much richer geometry for them - the geometry of Poisson brackets.

\begin{definition}[Functional of hydrodynamic type] \label{def:hamiltonian-functional}
    The functional of hydrodynamic type is a functional of the form
        \begin{align}
            H[u] = \int_{\R^d} h(u(x)) \D x \:,
        \end{align}
    where $h$ is independent of $u_\alpha$, $u_{\alpha \beta}$. The function $h$ is called a Hamiltonian density.
\end{definition}


\begin{definition}[Hamiltonian system of hydrodynamic type] \label{def:hamiltonian-system}
    The Hamiltonian system of hydrodynamic type is a system of the form
        \begin{align}
            \pder{u^i}{t} = 
            \left[ g^{ij \alpha} [u(x)] \ppder{h(u)}{u^j}{u^k} + b_k^{ij \alpha} [u(x)] \pder{h(u)}{u^j} \right] \pder{u^k}{x^\alpha} \:, \label{eq:Hamiltonian-system-of-h-t}
        \end{align}
        where $H[u]$ is the functional of the hydrodynamic type and $h$ the Hamiltonian density.
\end{definition}

\begin{definition}[Dubrovin-Novikov bracket]
    The Dubrovin-Novikov bracket of two functionals $I_1$ a $I_2$ is defined as
        \begin{align}
            \set{I_1,I_2} = \sum_{\alpha = 1}^ d \int_{\R^d} \D x 
            \left[ \frac{\delta I_1}{\delta u^p(x)} \left( g^{pq \alpha}(u) \der{}{x^\alpha} + b^{pq\alpha}_s(u) \pder{u^s}{x^\alpha} \right) \frac{\delta I_2}{\delta u^q(x)} \right] \:, \label{eq:Dubrovin-Novikov bracket}
        \end{align}
    where $g^{ij \alpha}$ and $b_k^{ij \alpha}$ are certain functions, $i,j,k = 1, \dots, N$ and $\der{}{x^\alpha}$ is the total derivative with respect to the independent variable $x^\alpha$, denoted by
        \begin{align}
            \der{}{x^\alpha} =& \pder{}{x^\alpha} + u^i_\alpha \pder{}{u^i} + u^i_{\alpha \beta} \pder{}{u^i_\beta} + \cdots + u^i_{\beta_1 \cdots \beta_s \alpha} \pder{}{u^i_{\beta_1 \cdots \beta_s}} + \cdots \:, \\
            u^i_{\beta_1 \cdots \beta_s} =& \pder{^s u^i}{x^{\beta_1} \cdots \partial x^{\beta_s}} \:.
        \end{align}
    The summation over repeating upper and lower indices is assumed, moreover, in this case the summation over $\beta_1, \cdots, \beta_s$ is taken over distinct (up to arbitrary permutation) sets of these indices. (One can consider that these sets of indices are ordered: $\beta_1 \leq \cdots \leq \beta_s$.)
\end{definition}

It is useful to write down the explicit form of the bracket of two field variables.

\begin{proposition}[Dubrovin-Novikov bracket of field variables]
    \begin{align}
        L^{ij}(x,y) \coloneqq \set{u^i(x), u^j(y)} = g^{ij \alpha} [u(x)] \delta_\alpha(x-y) + b_k^{ij \alpha} [u(x)] u_\alpha^k(x) \delta(x-y) \:.
    \end{align}
\end{proposition}

\begin{proof}
    \begin{align}
        \set{u^i(x), u^j(y)} =& \int \D \xi \fder{u^i(x)}{u^p(\xi)} \left( g^{pq \alpha}[u\xi)] \der{}{\xi^\alpha} + b^{pq \alpha}_s[u(\xi)] \pder{u^s}{\xi^\alpha} \right) \fder{u^j(y)}{u^q(\xi)} \nonumber
        \\ =& \int \D \xi \delta(x-\xi) \left[ g^{ij \alpha}[u(\xi)] \partial_\alpha \delta(y-\xi) + b^{ij \alpha}_s[u(\xi)] u^s_\alpha \delta(y- \xi) \right] \nonumber
        \\ =& g^{ij \alpha} [u(x)] \delta_\alpha(x-y) + b_k^{ij \alpha} [u(x)] u_\alpha^k(x) \delta(x-y) \:.
    \end{align}
\end{proof}

\begin{proposition}[Time evolution of the field variables]
    The Hamiltonian system of hydrodynamic type can be written as
    \begin{align}
        \pder{u^i}{t} = \set{u^i,H} \:.
    \end{align}
\end{proposition}
\begin{proof}
    \begin{align}
        &\set{u^i(x),H} 
        = \int \D \xi \fder{u^i(x)}{u^p(\xi)} \left[ g^{pq \alpha}[u(\xi)] \der{}{\xi} + b^{pq \alpha}_m [u(\xi)] \pder{u^m}{\xi^\alpha} \right] \fder{H}{u^q(\xi)} \nonumber 
        \\ =& \int \D \xi \int \D y \delta(x-\xi) \left[  g^{ij \alpha} [u(\xi)] \der{}{\xi} \pder{h}{u^j}(\xi)  + b^{ij \alpha}_k [u(\xi)] u^k_\alpha \pder{h}{u^j} (\xi) \right] \delta(y-\xi) \nonumber
        \\ =& \left[ g^{ij \alpha} [u(x)] \pder{^2 h}{u^j \partial u^k}(x)  + b^{ij \alpha}_k(u(x)) \pder{h}{u^j}(x) \right] \pder{u^k}{x^\alpha} \:,
    \end{align}
    which is the right-hand side of \eqref{eq:Hamiltonian-system-of-h-t}.
\end{proof}

Later we shall show that the Dubrovin-Novikov bracket defined above can be seen as Poisson bracket (of hydrodynamic type), i.e. satisfies  skew-symmetry, transformation rule and Jacobi identity. In fact, this will be the key theorem of this chapter.

\section{One-dimensional case}

For the rest of this chapter we first explore the one-dimensional brackets. In that case the unknown functions $u^i$ depend on only two variables $(t,x)$. The Hamiltonian system is then
\begin{align}
    \pder{u^i}{t} (t,x) =&
    \left[ g^{ij} [u(t,x)] \ppder{h(u)}{u^j}{u^k} + b_k^{ij} [u(t,x)] \pder{h(u)}{u^j} \right] \pder{u^k}{x}(t,x) \:, 
\end{align}
the Dubrovin-Novikov bracket is of the form
\begin{align}
    \set{I_1,I_2} = \int \D x
    \left[ \frac{\delta I_1}{\delta u^p(x)} \left( g^{pq}[u(x)] \der{}{x} + b^{pq}_s[u(x)] \pder{u^s}{x} \right) \frac{\delta I_2}{\delta u^q(x)} \right] \:, \label{eq:Dubrovin-Novikov bracket 1D}
\end{align}
and in the case of two field variables
\begin{align}
    \label{eq:Poisson1D}
    L^{ij}(x,y) = \set{u^i(x), u^j(y)} =& g^{ij} [u(x)] \partial_x \delta(x-y) + b_k^{ij} [u(x)] u^k_x \delta(x-y) \:. 
\end{align}
-
The bracket of two field variables forms a tensor as well.

\begin{proposition}[Transformation of $L^{ij}$]
    Let $u^i \mapsto v^a$ be the smooth change of variables. Then $L^{ij}$ transforms as the (2,0) tensors,
    \begin{align}
        L^{ab}(v) = \pder{v^a}{u^i}(x) \pder{v^b}{u^j}(y) L^{ij}(u) \:. \label{eq:transformace-zavorka}
    \end{align}
\end{proposition}
\begin{proof}
    The coordinates are written as functionals in the form
    \begin{align}
        v^p(u^i(x)) = \int \D x' v^p(u^i(x')) \delta (x-x') \:.
    \end{align}
    Applying this to the definition of the Dubrovin-Novikov bracket \eqref{eq:Poisson1D}, we have
    \begin{align}
        \set{v^p(u^i(x)),v^q(u^j(y))} 
        =& \int \D x' \int \D y' \underbrace{\fder{v^p}{u^i(x')} }_{\pder{v^p}{u^i}(x') \delta(x-x')} \set{u^i(x'),u^j(y')} \underbrace{\fder{v^q}{u^j(y')} }_{\pder{v^q}{u^j}(y') \delta(y-y')} \nonumber
        = \\ =& \pder{v^p}{u^i}(x) \pder{v^q}{u^j}(y) \set{u^i(x),u^j(y)} \:.
    \end{align}
\end{proof}

Our goal is to specify the conditions for the Dubrovin-Novikov bracket to be Poisson bracket. The most straightforward property is the transformation rule which is satisfied as long as the product of two functionals exists.

\begin{proposition}[Transformation rule for the bracket]
    \begin{align}
        \set{I,JK} = J \set{I,K} + \set{I,J} K \:,
    \end{align}
    assuming that both sides of the equation exist.
\end{proposition}
\begin{proof}
    Assuming that all equalities below are true,
    \begin{align}
        &\int \fder{AB}{u(x)} h(x) \, \D x = \der{}{\lambda} \big|_{\lambda = 0} \left[ A(u+ \lambda h) B(u+ \lambda h) \right] \nonumber
        \\ =& B(u) \der{}{\lambda} \big|_{\lambda = 0} A(u+ \lambda h) + A(u) \der{}{\lambda} \big|_{\lambda = 0} B(u+ \lambda h) \nonumber
        \\ =& B(u) \int \fder{A}{u} h(x) \, \D x + A(u) \int \fder{B}{u} h(x) \, \D x \:,
    \end{align}
    the transformation property follows immediately from the definition of the bracket \eqref{eq:Dubrovin-Novikov bracket 1D}.
\end{proof}


Skew-symmetry and Jacobi identity for the bracket are not easy to investigate. In order to do that, we introduce a pseudo-Riemann structure on this space. We focus on the case, where the system is non-degenerate and thus allows to raise and lower indices of tensors.

\begin{definition}[Non-degenerate systems]
    We say that the Dubrovin-Novikov bracket and the corresponding system are non-degenerate if $\det(g^{ij}) \neq 0$.
\end{definition}

 We shall show later that the non-degeneracy is invariant under the transformation of coordinates. In that case, we can naturally define its inverse form by $g_{ij} \coloneqq (g^{ij})^{-1}$.

Let $g^{ij}(u)$ be non-degenerate. Then we can define functions $\Gamma^{i}_{jk}(u)$ as
\begin{align}
    \Gamma^i_{jk}(u) \coloneqq - g_{js}(u) b^{s i}_k(u) \:. \label{eq:Gamma=g*b}
\end{align}
Note that this relation enables the decomposition
\begin{align}
    b_k^{ij}(u) = -g^{is}(u) \Gamma^j_{sk}(u) \:. \label{eq:b=g*Gamma}
\end{align}
The choice of denotation $g^{ij}(u)$ and $\Gamma^{i}_{jk}(u)$ has its purpose which is justified by the next observation.

\begin{proposition}[Transformation of $g^{ij}$ and $\Gamma^i_{jk}$] \label{prop:transoformace}
    Let $u^i \mapsto v^a$ be the smooth change of variables. Then
    \begin{enumerate}
        \item The $g^{ij}$ coefficients transform as the (2,0) tensors,
        \begin{align}
            g^{pq}(v) = \pder{v^p}{u^i} \pder{v^q}{u^j} g^{ij}[u(v)] \:. \label{eq:transformace-metrika}
        \end{align}
        \item The $\Gamma^i_{jk}$ coefficients  transform as Christoffel symbols,
        \begin{align}
            \Gamma^p_{qr} (v) = \pder{v^p}{u^i} \pder{u^j}{v^q} \pder{u^k}{v^r} \Gamma^i_{jk}(u) + \pder{v^p}{u^i} \ppder{u^i}{v^q}{v^r} \:. \label{eq:transformace-konexe}
        \end{align}
    \end{enumerate}
\end{proposition}

\begin{proof}
    \begin{enumerate}
        \item Let us apply the transformation equation \eqref{eq:transformace-zavorka} to the definition of the Dubrovin-Novikov bracket \eqref{eq:Poisson1D}. We get 
        \begin{align}
            g^{pq}[v(u(x))] \partial_x \delta(x-y) + b^{pq}_s[v(u(x))] \pder{v^s}{u^k}(x) \pder{u^k}{x} \delta(x-y) \nonumber
            \\ = \pder{v^p}{u^i}(x) \pder{v^q}{u^j}(y) \left[ g^{ij}[u(x)] \partial_x \delta(x-y) + b^{ij}_k[u(x)] \pder{u^k}{x} \delta(x-y) \right] \:. \label{eq:dosad}
        \end{align}
        From now on (for readibility) we denote 
        \begin{align}
            T^a_k(x) := \pder{v^a}{u^k}(x) \:.
        \end{align}
        We apply Dirac identity (Lemma \vref{lemma:delta}) to $T^q_j(y) \partial_x \delta(x-y)$ on the right-hand side of \eqref{eq:dosad}. Hence all functions are then of variable $x$, from now on we shall not write it explicitely. We have
        \begin{align}
            g^{pq}(v) \partial_x \delta(x-y) + b^{pq}_n(v) T^n_j \pder{u^j}{x} \delta(x-y) \nonumber
            \\ = T^p_i T^q_j g^{ij}(u) \partial_x \delta(x-y) 
            + \left( \ppder{v^q}{u^j}{u^k} + T^p_i T^q_j b^{ij}_k \right) \pder{u^k}{x} \delta(x-y) \:. \label{eq:porovnani}
        \end{align}
        The terms with $\partial_x \delta$ must be equal to each other, as must $\delta$ terms. From $\partial_x \delta$ terms we get
        \begin{align}
            g^{pq}(v) = T^p_i T^q_j g^{ij}(u) \:,
        \end{align}
        which is the transformation relation for $g$.

        \item From $\delta$ terms in \eqref{eq:porovnani} we obtain
        \begin{align}
            b^{pq}_n(v) T^n_k =
            b^{ij}_k(u) T^p_i T^q_j + g^{ij}(u) T^p_i \ppder{v^q}{u^i}{u^k}
        \end{align}
        and after the decomposition \eqref{eq:b=g*Gamma}
        \begin{align}
            - g^{ps}(v) \Gamma^{q}_{sn}(v) T^n_k =
            - g^{il}(u) \Gamma^j_{lk}(u) T^p_i T^q_j + g^{ij}(u) T^p_i \ppder{v^q}{u^i}{u^k} \:.
        \end{align}
        On the left-hand side we use the transformation of $g$ \eqref{eq:transformace-metrika}
        \begin{align}
            -g^{ij}(u) \Gamma^q_{sm}(v) T^p_i T^s_j T^m_k = - g^{ij}(u) \Gamma^l_{jk} T^p_i T^q_l + g^{ij}(u) \ppder{v^q}{u^i}{u^k} T^p_i \:,
        \end{align}
        and after factoring out $g^{ij}(u) T^p_i$
        \begin{align}
            \Gamma^l_{jk}(u) T^q_l = \Gamma^q_{sm}(v) T^s_j T^m_k + \ppder{v^q}{u^i}{u^k} \:.
        \end{align}
        All that remains is to multiply this equation with $(T^{-1})^a_q$ after which we receive the final transformation relation
        \begin{align}
            \Gamma^a_{jk}(u) = \Gamma^q_{sm}(v) \pder{v^s}{u^j} \pder{v^m}{u^k} \pder{u^a}{v^q} + \ppder{v^q}{u^i}{u^k} \pder{u^a}{v^q} \:.
        \end{align}
    \end{enumerate}
\end{proof}



\section{Dubrovin-Novikov theorem}

Let us summarize what we have investigated so far. We have defined the Dubrovin-Novikov bracket in the way that it satisfies the transformation rule. It remains to investigate the skew-symmetry and Jacobi identity. For that, we have defined $g^{ij}$ and $\Gamma^i_{jk}$ which transform as tensors, and Christoffel symbols, respectively. It remains to show that $g^{ij}$ is indeed symmetric and $\Gamma^i_{jk}$ is the Levi-Civita connection, i.e., the corresponding covariant derivative $\nabla_j$ has zero torsion and is compatible with the metric.

We are now ready to prove the critical theorem of this chapter, which shows that these properties are indeed equivalent. Moreover, the corresponding covariant derivative has zero curvature, indicating that the phase space $M$ can be regarded as the Euclidean space.

\begin{theorem}[Dubrovin, Novikov;\cite{Dubrovin-Novikov}] \label{Theorem-Dubrovin-Novikov-1D}
    Let $g^{ij}$ be non-degenerate. Then the Dubrovin-Novikov bracket \eqref{eq:Poisson1D} is a Poisson bracket with skew-symmetry, transformation rule and Jacobi identity if and only if these three conditions are satisfied:
    \begin{enumerate}
        \item $g^{ij} = g^{ji}$, that is, $g^{ij}$ forms an inverse metric on the phase space $M$.
        \item $\Gamma^{i}_{jk} = \Gamma^i_{kj}$, that is, $\Gamma^{i}_{jk}$ are Christoffel symbols corresponding to the Levi-Civita covariant derivative compatible with $g^{ij}$.
        \item The covariant derivative corresponding to $\Gamma^{i}_{jk}$ has zero curvature.
    \end{enumerate}
\end{theorem}

\begin{corollary} \label{Corollary: existence of flat coordinates}
    There exist local coordinates $w^i = w^i(u^1, \dots, u^N)$, $i = 1, \dots, N$ such that the metric is flat:
    \begin{align}
        g^{ij}(w) = \eta^{ij} = \const \:, \quad \Gamma^i_{jk} (w) = 0 \:.
    \end{align}
    In these flat coordinates the Dubrovin-Novikov bracket is reduced to the constant form
    \begin{align}
        \set{w^i(x),w^j(y)} = \eta^{ij} \partial_x \delta(x-y) \:.
    \end{align}
    The only local invariant of Poisson bracket is the signature of the metric $g^{ij}(u)$.
\end{corollary}

Due to the relatively large length of the proof, we divide it into four steps.
\begin{itemize}
    \item \textbf{Step 1:} We show that the skew-symmetry of the bracket implies the symmetry of the metric $g^{ij}$ and compatibility with the Christoffel symbols $\Gamma^i_{jk}$.
    \item \textbf{Step 2:} We formulate the conditions under which the Jacobi identity for the bracket holds.
    \item \textbf{Step 3:} We show that two of these conditions implies zero torsion and zero curvature of the covariant derivative associated with $\Gamma^i_{jk}$. That finishes the implication in one direction.
    \item \textbf{Step 4:} We prove the converse implication via the existence of the flat coordinates.
\end{itemize}

\begin{lemma}[Step 1: skew-symmetry] \label{lemma:skew-symmetry}
    If the skew-symmetry of the Dubrovin-Novikov bracket holds, then
    \begin{align}
        &g^{ij} = g^{ji} \:, \label{eq:skew-1}\\
        &g^{ij}_{,m} = b^{ij}_m + b^{ji}_m \label{eq:skew-2} \:.
    \end{align}
\end{lemma}
\begin{proof}
    The skew-symmetry reads as
    \begin{align}
        \set{u^i(x),u^j(y)} + \set{u^j(y),u^i(x)} = 0 \:.
    \end{align}
    According to the definition, the exact form of the second bracket is
    \begin{align}
        \set{u^j(y),u^i(x)} =  g^{ji}[u(y)] \partial_x \delta(y-x) + b^{ji}_k[u(y)] \pder{u^k}{y} \delta(y-x) \:.
    \end{align}
    Let us apply Dirac identites (Lemma \vref{lemma:delta}) on $g^{ji}[u(y)]\partial_x \delta(y-x)$. We get
    \begin{align}
        \set{u^j(y),u^i(x)} =& \nonumber \\ 
        = - g^{ji} [u(x)] \partial_x \delta(x-y) -& \pder{g^{ji}}{u^k}[u(x)] \pder{u^k}{x} \delta(x-y) + b^{ji}_k[u(x)] \pder{u^k}{x} \delta(x-y)
        \:.
    \end{align}
    The skew-symmetry is then
    \begin{align}
        0 =& \set{u^i(x),u^j(y)} + \set{u^j(y),u^i(x)}  \nonumber
        \\=& \left[ g^{ij}[u(x)]  - g^{ji}[u(x)] \right] \partial_x \delta(x-y) + \left[ b^{ij}_k - \pder{g^{ji}}{u^k}  + b^{ji}_k  \right] \pder{u^k}{x} \delta(x-y) \:.
    \end{align}
    The right-hand side is equal to zero if and only if
    \begin{align}
        g^{ij}(u) = g^{ji}(u) \:, \label{eq:Th.Novikov-symetrie} \\
        \pder{g^{ij}}{u^k}(u) = b^{ij}_k(u) + b^{ji}_k(u)  \:. \label{eq:Th.Novikov-konexe}
    \end{align}
\end{proof}

\begin{lemma}[Step 2: Jacobi identity] \label{lemma:Jacobi}
    The Jacobi identity for the bracket is equivalent to 
    \begin{align}
        &b^{ij}_m g^{mk} = b^{ik}_m g^{mj} \:, \label{eq:Jacobi-1}\\ 
        &b^{ij}_{m} b^{mk}_s - b^{ik}_m b^{mj}_s = g^{mi} \left( b^{jk}_{s,m} - b^{jk}_{m,s} \right) \:, \label{eq:Jacobi-2}\\
        \sum_{(i,j,k)}& \left[ (b^{ij}_{t,p} - b^{ij}_{p,t}) b^{tk}_s + (b^{ij}_{t,s} - b^{ij}_{s,t}) b^{tk}_p \right] = 0 \:, \label{eq:Jacobi-3}
    \end{align}
    where the summation $\sum_{(i,j,k)}$ means summation over cyclic permutations of indices $i,j,k$.
\end{lemma}

\begin{proof}
    We denote
    \begin{align}
        \mathcal J^{ijk}(x,y,z) = J^{ijk}(x,y,z)+ J^{jki}(y,z,x)+ J^{kij}(z,x,y) = 0 \:,
    \end{align}
    where
    \begin{align}
        J^{ijk}(x,y,z) = \set{\set{u^i(x),u^j(y)}, u^k(z)} \:.
    \end{align}
    From now on we shall use the abbreviated notation
    \begin{align}
        g^{ij}_{,m} = \pder{g^{ij}}{u^m} \:, \quad b^{ij}_{k,m} = \pder{b^{ij}_k}{u^m} \:, \quad u^k_x = \pder{u^k}{x}  \:.
    \end{align}
    One of the Jacobiators reads as
    \begin{align}
        J^{ijk}(x,y,z) =
        \int \D \xi \fder{\set{u^i(x),u^j(y)}}{u^m(\xi)} 
        \left[ g^{mn}(\xi) \partial_\xi + b^{mn}_p \partial_\xi u^p \cdot \right] \fder{u^k(z)}{u^n(\xi)}  \:.
    \end{align}
    The functional derivatives are
    \begin{align}
        \fder{u^k(z)}{u^n(\xi)} = \delta^k_n \delta(z-\xi) \:,
    \end{align}
    and 
    \begin{align}
        \fder{\set{u^i(x),u^j(y)}}{u^m(\xi)} = \fder{}{u^m(\xi)} \left[ g^{ij}(x) \partial_x \delta(x-y) + b^{ij}_k(x) \partial_x u^k \delta(x-y)\right] = 
        \nonumber
        \\ = \left[ g^{ij}_{,m}(x) \partial_x \delta(x-y) + b^{ij}_{k,m}(x) u^k_x \delta(x-y) \right] \delta (x-\xi) - b^{ij}_m(x) \delta(x-y) \partial_\xi \delta(x-\xi) \:.
    \end{align}
    We have
    \begin{align}
        J^{ijk}(x,y,z) = J_1^{ijk}(x,y,z) + J_2^{ijk}(x,y,z) \:,
    \end{align}
    where
    \begin{align}
        & J_1^{ijk}(x,y,z) = \int \D \xi  \left[ g^{ij}_{,m}(x) \partial_x \delta(x-y) \delta(x-\xi) + b^{ij}_{k,m}(x) u^k_x \delta(x-y) \delta(x-\xi) \right] \times \nonumber \\ 
        & \times \left[ g^{mn}(\xi) \partial_\xi + b^{mn}_p \partial_\xi u^p \cdot \right] \delta^k_n \delta(z-\xi)  \:, 
    \end{align}
    and 
    \begin{align}
        J_2^{ijk}(x,y,z) = - \int \D \xi b^{ij}_m(x) \delta(x-y) \partial_\xi \delta(x-\xi) \left[ g^{mn}(\xi) \partial_\xi + b^{mn}_p \partial_\xi u^p \cdot \right] \delta^k_n \delta(z-\xi)  \:.
    \end{align}
    In $J_1$ we use integration with $\delta(x-\xi)$ and get rid of the integral:
    \begin{align}
        &J_1^{ijk}(x,y,z) = \nonumber
        \\=& \left[ g^{ij}_{,m}(x) \partial_x \delta(x-y) + b^{ij}_{k,m}(x) u^k_x \delta(x-y) \right] 
        \left[ g^{mk}(\xi) \partial_x + b^{mk}_p \partial_x u^p \cdot \right] \delta(z-x) \nonumber
        \\
        =& \left[ g^{ij}_{,m} g^{mk} \partial_x \delta(x-y) + b^{ij}_{k,m} g^{mk} u^k_x \delta(x-y) \right] \partial_x \delta(z-x) + \nonumber
        \\ 
        +& \left[ g^{ij}_{,m} b^{mk}_p u^p_x \partial_x \delta(x-y) + b^{ij}_{k,m} b^{mk}_p u_x^p  u_x^k \delta(x-y) \right] \delta(z-x) \:.
    \end{align}

    In $J_2$ we first apply integration by parts to eliminate $\partial_\xi \delta(x-\xi)$ and then integrate:
    \begin{align}
        &J_2^{ijk}(x,y,z) = \nonumber
        \\=& \int \D \xi b^{ij}_m(x) \delta(x-y) \delta(x-\xi) \partial_\xi \left[ g^{mk}(\xi) \partial_\xi \delta(z-\xi) + b^{mk}_p \partial_\xi u^p \delta(z-\xi) \right] \nonumber
        \\ =& b^{ij}_m(x) \delta(x-y) \partial_x \left[ g^{mn}(x) \partial_x \delta(z-x) + b^{mn}_p \partial_x u^p \delta(z-x) \right] \nonumber
        \\ =& b^{ij}_m(x) \delta(x-y) \big[ g^{mk}_{,s} u^s_x \partial_x \delta (z-x) + g^{mk} \partial_x \partial_x \delta (z-x) \nonumber
        \\ +& b^{mk}_{p,s}(x) u^s_x u^p_x \delta (z-x) + b^{mn}_{p}(x) u^p_{xx} \delta (z-x) + b^{mk}_{p}(x) u^p_x \partial_x \delta (z-x) \big] \:,
    \end{align}
    This leads to
    \begin{align}
        J_2^{ijk}(x,y,z) =& b^{ij}_m g^{mk}_{,s} u^s_x \delta (x-y) \partial_x \delta (z-x) 
        + b^{ij}_m g^{mk} \delta(x-y) \partial_x \partial_x \delta (z-x) \nonumber
        \\ +& b^{ij}_m b^{mk}_{p,s} u^s_x u^p_x \delta (x-y) \delta (z-x)
        + b^{ij}_m b^{mn}_p u^p_{xx} \delta (x-y) \delta(z-x) \nonumber 
        \\ +& b^{ij}_m b^{mk}_p u^p_x \delta(x-y) \partial_x \delta(z-x) \:.
    \end{align}

    After some adjustments we finish with the following terms
    \begin{align}
        J^{ijk}(x,y,z) =& 
        E^{ijk}(x) \delta(x-y) \delta(z-x) 
        \\+& F^{ijk}(x) \delta(x-y) \partial_x \delta(z-x) \:,
        \\+& G^{ijk}(x) \partial_x \delta(x-y) \delta (z-x) \:,
        \\+& H^{ijk}(x) \partial_x \delta(x-y) \partial_x \delta (z-x) \:,
        \\+& I^{ijk}(x) \delta(x-y) \partial_x \partial_x \delta (z-x) \:,
    \end{align}
    where
    \begin{align}
        E^{ijk} =& (b^{ij}_m b^{mk}_{p,s} + b^{ij}_{s,m} b^{mk} ) u^p_x u^s_x + b^{ij}_m b^{mk}_p u^p_{xx} \:, \\
        F^{ijk} =& (b^{ij}_{s,m} g^{mk} + g^{mk}_{,s} b^{ij}_m + b^{ij}_m b^{mk}_s)u_x^s \:, \\
        G^{ijk} =& g^{ij}_{,m} b^{mk}_s u^s_x \:,\\
        H^{ijk} =& g^{ij}_{,m} g^{mk} \:,\\
        I^{ijk} =& b^{ij}_m g^{mk} \:.
    \end{align}
    Now the procedure is the following: firstly we have to add the cyclic permutations $J^{jki}(y,z,x)$ and $J^{kij}(z,x,y)$ to obtain $\mathcal J^{ijk}(x,y,z)$. Secondly, the resulting generalized function $\mathcal J^{ijk}(x,y,z)$ is equal to zero, i.e.
    \begin{align}
        \dual{\mathcal J^{ijk}(x,y,z)}{p_i(x)q_j(y)r_k(z)} = 0
    \end{align}
    or in the integral notation 
    \begin{align}
        \iiint \D x \D y \D z J^{ijk}(x,y,z) p_i(x)q_j(y)r_k(z) = 0 \:.
    \end{align}
    We want to reduce this integral into a one-dimensional integral
    \begin{align}
        \int \D x \sum_{a,b = 0}^2 A^{ijk}_{ab} p_i(x) q^{(a)}_j(x) r^{(b)}_k (x) \:,
    \end{align}
    where $f^{(a)}$ means the $a$-th derivative. This integral is equal to zero if and only if
    \begin{align}
        A^{ijk}_{ab} = 0 \quad \forall a,b = 0,1,2 \:.
    \end{align}
    From this condition, we shall obtain zero torsion and curvature.
    We illustrate the reduction procedure to the one-dimensional integral on one particular expression. For instance, we have
    \begin{align}
        \int \big[ F^{ijk}(x) \delta(x-y) \partial_x \delta(z-x) 
        + F^{jki}(y) \delta(y-z) \partial_y \delta(x-y) \nonumber \\
        + F^{kij}(z) \delta(z-x) \partial_z \delta(y-z) \big] \,
        p_i(x) q_j(y) r_k(z) \,\D x \D y \D z \:.
    \end{align}
    We use the antisymmetry $
        \partial_x \delta(z-x) = - \partial_z \delta (z-x)$
    and similarly in each term and then eliminate the integrals with $\D y$ and $\D z$, obtaining
    \begin{align}
        \int \D x F^{ijk}(x) p_i q_j r'_k + \int \D x F^{jki}(x)  p'_i q_j r_k +
        \int \D x F^{kij}(x) p_i q'_j r_k \:.
    \end{align}
    In the term containing $p'_i$ we perform integration by parts. The result is
    \begin{align}
        \int \D x (F^{ijk}-F^{jki}) p_i q_j  r'_k +
        \int \D x (F^{kij}-F^{jki}) p_i q'_j r_k -
        \int \D x \partial_x F^{jki} p_i q_j r_k \:.
    \end{align}
    In this particular case, we end up with terms, which contribute to $A^{ijk}_{01}$, $A^{ijk}_{10}$ and $A^{ijk}_{00}$, respectively. Other terms in $\mathcal J^{ijk}(x,y,z)$ are handled in similar way.

    The list of the resulting functions is the following:
    \begin{align}
        A_{00}^{ijk} =& E^{ijk} + E^{kij} + E^{jki} + \partial_x \partial_x I^{jki} - \partial_x F^{jki} - \partial_x G^{kij} \:, \label{eq:A00}\\
        A_{01}^{ijk} =& F^{ijk} - F^{jki} + G^{jki} - G^{kij} - \partial_x H^{jki} + 2 \partial_x I^{jki} \:, \label{eq:A01}\\
        A_{10}^{ijk} =& F^{kij} - F^{jki} + G^{ijk} - G^{kij} - \partial_x H^{kij} + 2 \partial_x I^{jki} \:, \label{eq:A10}\\
        A_{11}^{ijk} =& H^{ijk} - H^{jki} - H^{kij} + 2 \partial_x I^{jki} \:, \label{eq:A11} \\
        A_{02}^{ijk} =& I^{ijk} + I^{jki} - H^{jki}\:, \label{eq:A02}\\
        A_{20}^{ijk} =& I^{jki} + I^{kij} - H^{kij}\:, \label{eq:A20} \\
        A^{ijk}_{12} =& A^{ijk}_{21} = 0 \:.
    \end{align}
    Note that \eqref{eq:A20} is just a cyclic permutation of \eqref{eq:A02}. If we also use the symmetry conditon \eqref{eq:skew-1} and the compatibility condition \eqref{eq:skew-2}, then the equation \eqref{eq:Jacobi-1} follows from \eqref{eq:A02}. Equation \eqref{eq:A00} has the form
    \begin{align}
        A_{00}^{ijk} = B^{ijk}_s u^s_{xx} + C^{ijk}_{sp} u^s_x u^p_x \:.
    \end{align}
    The expression on the right hand side is zero if and only if both $B^{ijk}_s$ and $C^{ijk}_{sp}$ are equal to zero. From the function $B^{ijk}_s$ being zero follows \eqref{eq:Jacobi-2} and from $C^{ijk}_{sp}$ follows \eqref{eq:Jacobi-3}. Finally, equations \eqref{eq:A01}, \eqref{eq:A10} and \eqref{eq:A11} are satisfied automatically if \eqref{eq:A00} and \eqref{eq:A02} hold.
\end{proof}

It is useful to write once more all the equations from lemmas \vref{lemma:skew-symmetry} and \vref{lemma:Jacobi} together. Since they first appeared in \cite{Grinberg}, we refer to them as to the Grinberg conditions.
\begin{lemma}[Grinberg conditions; \cite{Grinberg}] \label{Grinberg conditions}
    If the Dubrovin-Novikov bracket satisfies skew-symmetry and Jacobi identity, then the following conditions hold:
    \begin{align}
        &g^{ij} = g^{ji} \:, \label{eq:Grinberg-1}\\
        &g^{ij}_{,m} = b^{ij}_m + b^{ji}_m \:, \label{eq:Grinberg-2}\\
        &b^{ij}_m g^{mk} = b^{ik}_m g^{mj} \:, \label{eq:Grinberg-3}\\
        &b^{ij}_{m} b^{mk}_s - b^{ik}_m b^{mj}_s = g^{mi} \left( b^{jk}_{s,m} - b^{jk}_{m,s} \right) \:, \label{eq:Grinberg-4}\\
        \sum_{(i,j,k)}& \left[ (b^{ij}_{t,p} - b^{ij}_{p,t}) b^{tk}_s + (b^{ij}_{t,s} - b^{ij}_{s,t}) b^{tk}_p \right] = 0 \:. \label{eq:Grinberg-5}
    \end{align}
\end{lemma}

Until now, we have not assumed non-degeneracy of the bracket. Note that non-degeneracy allows to use $\Gamma^{i}_{jk} = - g_{js} b^{si}_k$, since in the degenerate case the inverse $g_{js}$ does not exist. If we reformulate the Grinberg conditions in terms of $\Gamma^{i}_{jk}$, we get the compatibility, zero torsion, and curvature.

\begin{lemma}[Step 3: Zero curvature and torsion]
    If the Grinberg conditions hold and the $g^{ij}$ is non-degenerate, then $g^{ij}$ is a metric, the covariant derivative $\nabla_i$ associate with $\Gamma^{i}_{jk}$ is compatible with $g^{ij}$ and it has zero torsion and curvature, i.e. $\nabla_i$ is a Levi-Civita covariant derivative with zero curvature.
\end{lemma}
\begin{proof}
    Equation \eqref{eq:Grinberg-1} says that $g^{ij}$ is a symmetric tensor. By assumption $\det g^{ij}$ is non-degenerate, therefore it defines a metric structure on $M$. Equation \eqref{eq:Grinberg-2} is equivalent to
    \begin{align}
        g^{ij}_{,k} + g^{is} \Gamma_{sk}^j + g^{sj} \Gamma_{sk}^i = 0 \:.
    \end{align}
    If $\nabla_k$ is the covariant derivative corresponding to $\Gamma^i_{jk}$, then $\nabla_k g^{ij} = 0$. So $\Gamma_{ij}^k$ are affine connection coefficients, compatible with $g^{ij}$.

    To obtain zero torsion, we use \eqref{eq:Grinberg-3}, which reads as 
    \begin{align}
        g^{mk} g^{sj} \Gamma^i_{ms} = g^{mk} g^{sj} \Gamma^i_{sm} \:.
    \end{align}
    Hence the Christoffel symbols are symmetric in the lower indices, implying that the covariant derivative corresponding to them has zero torsion.

    Zero curvature follows from \eqref{eq:Grinberg-4}, which reads as
    \begin{align}
        g^{in} g^{ml} (\Gamma^j_{nm} \Gamma^{k}_{ls} - \Gamma^k_{nm} \Gamma^j_{ls}) = - g^{im} \pder{}{u^m} \left( g^{jl} \Gamma^k_{ls}\right) + g^{im} \pder{}{u^s}\left( g^{jl} \Gamma^k_{lm}\right) \:,
    \end{align}
    and reduces to
    \begin{align}
        g^{im} g^{jn} \left( \Gamma^{l}_{nm} \Gamma^{k}_{ls} - \Gamma^{l}_{ns} \Gamma^k_{lm} + \Gamma^k_{nm,s} - \Gamma^k_{ns,m} \right) = 0 \:.
    \end{align}
    This means that the Riemann tensor $R\indices{_n_m^k_s}$ is equal to zero.
\end{proof}

The final step is to prove the converse statement and the corollary \vref{Corollary: existence of flat coordinates}, which turns out to be very simple due to the non-degeneracy of the bracket. 

\begin{lemma}[Step 4: The converse statement]
    If $g^{ij}$ is non-degenerate metric and $\Gamma^i_{jk}$ correspond to the Levi-Civita covariant derivative $\nabla$ with zero curvature, then the Dubrovin-Novikov bracket is a Poisson bracket, i.e. it satisfies skew-symmetry, transformation rule and Jacobi identity.
\end{lemma}

\begin{proof}
    If $\nabla$ has zero torsion and curvature, there are coordinates\\ $w^i=w^i(u^1,\dots,u^N)$ for $i=1,\dots,N$ such that $g^{ij} \eqqcolon \eta^{ij} =\const$ a $b^{ij}_k = 0$. In these coordinates
    \begin{align}
        \set{w^i(x),w^j(y)} = \eta^{ij} \partial_x \delta(x-y) \:.
    \end{align}
    Jacobi identity, skew-symmetry and transformation rule are for that bracket satisfied trivially due to the antisymmetry of the delta distribution.
\end{proof}
Thus the proof of the Dubrovin-Novikov theorem \vref{Theorem-Dubrovin-Novikov-1D} is finished.

The coordinates in which Poisson bracket can be reduced to the constant form do not always have a physical interpretation \cite{Dubrovin-Novikov}. There is another class of coordinates, which is common in various applications. Coordinates $u^1, \cdots, u^N$ are called Liouville if the metric $g^{ij}$ and the connection $b^{ij}_k$ have the form
\begin{align}
    g^{ij}(u) = \gamma^{ij}(u)+ \gamma^{ji}(u) \:, \quad b^{ij}_k(u) = \gamma^{ji}_{,k}(u) \:. 
\end{align}
In those coordinates the Poisson bivector has the form
\begin{align}
    \set{u^i(x),u^j(y)} = \left[ \gamma^{ij}(u(y)) + \gamma^{ji}(u(x)) \right] \delta'(x-y) \:.
\end{align}

\begin{example}[One-dimensional isothermal system]
    In Example \vref{example:1D isothermal} we illustrated the one-dimensional isothermal system with $u^1 = \rho$, $u^2 = m$. These coordinates are in fact Liouville:
    \begin{align}
        \gamma^{ij} = -\begin{pmatrix}
            0 & 0 \\
            \rho & m
        \end{pmatrix} \:, \quad
        g^{ij} = \begin{pmatrix}
            0 & \rho \\
            \rho & 2m 
        \end{pmatrix} \:.
    \end{align}
\end{example}


\begin{example}[More general system, \cite{Dubrovin-Novikov}]
    The one-dimensional hydrodynamics with $(u^1,u^2,u^3) = (\rho, m, s)$. The Hamiltonian has the form
    \begin{align}
        H = \int \left[ \frac{p^2}{2 \rho} + \epsilon(\rho,s) \right] \, \D x \:,
    \end{align}
    where $\epsilon(\rho,s)$ is the energy density. These coordinates are in fact Liouville:
    \begin{align}
        \gamma^{ij} = \begin{pmatrix}
            p & 0 & 0 \\ \rho & 0 & 0 \\ s & 0 & 0
        \end{pmatrix}
        \:, \quad
        g^{ij} = \begin{pmatrix}
            2p & \rho & s \\ \rho & 0 & 0 \\ s & 0 & 0
        \end{pmatrix}
        \:.
    \end{align}
\end{example}

The direct consequence of the Dubrovin-Novikov theorem \vref{Theorem-Dubrovin-Novikov-1D} is the possibility of writing the original equations in covariant form.

\begin{theorem}[Hamiltonian system in the covariant form] \label{theorem:covariant}
    Given Hamiltonian system of hydrodynamic type
    \begin{align}
        \pder{u^i}{t} = f^i_k(u) \pder{u^k}{x} \:, \quad f^i_k(u) = g^{ij}(u) \ppder{h}{u^j}{u^k} - g^{is}(u) \Gamma^j_{sk}(u) \pder{h}{u^j} \:, \label{eq:hamiltonovsky-system}
    \end{align}
    one can write
    \begin{align}
        f^i_k(u) = \nabla^i \nabla_k h(u) \:, 
    \end{align}
    where $\nabla_j$ is Levi-Civita covariant derivative compatible with $g_{ij}$ and $\nabla^i = g^{is} \nabla_s$. 
\end{theorem}
\begin{proof}
    Since $h$ is ordinary function, application of $\nabla_k$ is just the partial derivative with respect to $u^k$:
    \begin{align}
        \nabla_k h(u) =& \pder{h}{u^k} \:.
    \end{align}
    Now this term $\pder{h}{u^k}$ is a covector, so $\nabla_i$ is given by the partial derivative and the $\Gamma_{ij}^k$ part by
    \begin{align}
        \nabla^i \nabla_k h(u) =& g^{is} \nabla_s \pder{h}{u^k} = g^{is} \left( \ppder{h}{u^s}{u^k} - \Gamma^j_{sk} \pder{h}{u^j} \right) = f^i_k(u) \:.
    \end{align}
\end{proof}

\chapter{Systems of hydrodynamic type:  multi-dimensional case}

\section{Multi-dimensional brackets. Mokhov conditions}

In the case with multi-dimensional bracket
\begin{align}
    \set{I_1,I_2} = \sum_{\alpha = 1}^ d \int \D x 
    \left[ \frac{\delta I_1}{\delta u^p(x)} \left( g^{pq \alpha}(u) \der{}{x^\alpha} + b^{pq\alpha}_s(u) \pder{u^s}{x^\alpha} \right) \frac{\delta I_2}{\delta u^q(x)} \right] \:, \label{eq:multi-dimensional bracket}
\end{align}
we have a set of metrics $g^{ij \alpha}$ and connections $b^{ij \alpha}_k$, where $\alpha = 1 , \cdots, d$. We can again introduce Christoffel symbols defined by
\begin{align}
    b^{i j \alpha}_k = - g^{is \alpha} \Gamma^{j \alpha}_{sk}\:,
\end{align}
(there is no summation over $\alpha$) and the corresponding set of covariant derivatives $\nabla^{\alpha}_j$. The first observation is that these coefficients represent coordinates of a vector under an transformation of the space coordinates $x^\alpha$.
\todo[inline]{Skutečně se mění jenom závislost $u$ na $x^\alpha$, nebo i něco jiného? Může to platit i při nějaké obecné transformaci?}
\begin{proposition}[Transformation of the space coordinates]
    For any changes of the spatial variables $\tilde x^\alpha \coloneqq c^\alpha_\beta x^\beta$, where $\alpha = 1, \cdots, d$ and $\det (c^\alpha_\beta) = 1$, the metrics $g^{ij \alpha}$ and $b^{ij \alpha}_k$ transform as components of a vector.
\end{proposition}

\begin{proof}
    The only variable that changes is $\pder{u^i}{x^\alpha}$
    \begin{align}
        \pder{u^i}{x^\beta} = \pder{u^i}{\tilde x^\alpha} \pder{\tilde x^\alpha}{x^\beta} = \pder{u^i}{\tilde x^\alpha} c^\alpha_\beta \:.
    \end{align}
    Hence
    \begin{align}
        g^{ij \alpha}[u(\tilde x)] = c^\alpha_\beta g^{ij \beta}[u(x)] \:, \quad b^{ij \alpha}_k [u(\tilde x)] = c^\alpha_\beta b^{ij \beta}_k [u(x)] \:.
    \end{align}
\end{proof}


The formulation of the generalized form of the Dubrovin-Novikov theorem (Theorem \vref{Theorem-Dubrovin-Novikov-1D}) is given (without proof) by Mokhov in \cite{Mochov}.

\begin{theorem}[Mokhov conditions]
    A Dubrovin-Novikov bracket \eqref{eq:Dubrovin-Novikov bracket} is a Poisson bracket if and only if the following relations for the coefficients are fulfilled:
    \begin{align}
        g^{ij \alpha} = g^{ji \alpha} \:, \label{eq:Mochov-A} \\
        g^{ij \alpha}_{,k} = b^{ij \alpha}_k + b^{ji \alpha}_k \:, \label{eq:Mochov-B}
    \end{align}
    \begin{align}
        \sum_{(\alpha, \beta)} (g^{mi \alpha} b^{jk \beta}_m - g^{m j \beta} b^{i k \alpha}_m) = 0 \:, \label{eq:Mochov-C}\\
        \sum_{(i,j,k)} (g^{mi \alpha} b^{jk \beta}_m - g^{m j \beta} b^{ik \alpha}_m) = 0 \:, \label{eq:Mochov-D}\\
        \sum_{(\alpha, \beta)} \left[ g^{si \alpha} (b^{jk \beta}_{s,q} - b^{jk \beta}_{q,s} ) + b^{ij \alpha}_s b^{s k \beta}_q - b^{i k \alpha}_s b^{k j \beta}_q \right] = 0 \label{eq:Mochov-E}\:,
    \end{align}
    \begin{align}
        g^{si \beta} b^{j k \alpha}_{q,s} - b^{ij \beta}_s b^{s k \alpha}_q - b^{i k \beta}_s b^{j s \alpha}_q = g^{s j \alpha} b^{i k \beta}_{q,s} - b^{ji \alpha}_s b^{s k \beta}_q - b^{is \beta}_q b^{jk \alpha}_s \label{eq:Mochov-F}\:,
    \end{align}
    \begin{align}
        \pder{}{u^r} \left[ g^{si \alpha} (b^{j k \beta}_{s,q} - b^{j k \beta}_{q,s}) + b^{ij \alpha}_s b^{s k \beta}_q - b^{i k \alpha}_s b^{s j \beta}_q \right] 
        \nonumber \\+ \pder{}{u^q} \left[ g^{si \beta} (b^{j k  \alpha}_{s,k} - b^{j k \alpha}_{r,s}) + b^{ij \beta}_s b^{s k \alpha}_{s,r} + b^{ij \beta}_s b^{s k \alpha}_r - b^{i k \beta}_s b^{k j \alpha}_s \right] \label{eq:Mochov-G}
        \\+ \sum_{(i,j,k)} \left[ b^{si \beta}_q (b^{j k \alpha}_{r,s} - b^{j k \alpha}_{s,r}) + b^{s i \alpha}_r (b^{j k \beta}_{q,s} - b^{j k \beta}_{s,q}) \right] \nonumber
        = 0 \:.
    \end{align}
\end{theorem}
The summations $\sum_{(\alpha, \beta)}$ and $\sum_{(i,j,k)}$ mean summations over all cyclic permutations of the indicated indices. Note that in \eqref{eq:Mochov-C}-\eqref{eq:Mochov-G} the indices $\alpha$ and $\beta$ do not have to be distinct indices (the cyclic permutation $\sum_{(\alpha, \beta)}$ can be ommited in that case). Thus, in particular, we get an important lemma.
\begin{lemma}
    Every summand on the right-side of the multi-dimensional bracket \eqref{eq:multi-dimensional bracket} is one-dimensional Dubrovin-Novikov bracket \eqref{eq:Dubrovin-Novikov bracket 1D}, i.e. each multidimensional bracket of the form \eqref{eq:multi-dimensional bracket} is always the sum of one-dimensional brackets \eqref{eq:Dubrovin-Novikov bracket 1D} with respect to each of the independent variables $x^\alpha$.
\end{lemma}


\begin{remark}
    Let us take once again a closer look to the case of $d=1$ and compare it to the Grinberg conditions (\vref{Grinberg conditions}). The Mokhov conditions are following:
    \begin{align}
        g^{ij} =& g^{ji} \quad &\textbf{(symmetry of metric)} \:, \label{eq:Mochov1D-A}\\
        g^{ij}_{,k} =& b^{ij}_k + b^{ji}_k \quad &\textbf{(compatibility condition)} \label{eq:Mochov1D-B}\:, \\
        g^{mi} b^{jk}_m - g^{mj} b^{ik}_m =& 0 \quad &\textbf{(zero torsion)} \:, \label{eq:Mochov1D-C}\\
        \sum_{(i,j,k)} g^{mi} b^{jk}_m - g^{mj} b^{ik}_m =& 0 \:,\quad& \label{eq:Mochov1D-D}
    \end{align}
    \begin{align}
        g^{si} (b^{jk}_{s,q} - b^{jk}_{q,s}) + b^{ij}_s b^{sk}_q - b^{ik}_s b^{kj}_q = 0 \quad &\textbf{(zero curvature)}\:, \label{eq:Mochov1D-E}
    \end{align}
    \begin{align}
        g^{si} b^{j k}_{q,s} - b^{ij}_s b^{s k}_q - b^{i k}_s b^{j s}_q = g^{s j } b^{i k}_{q,s} - b^{ji}_s b^{s k}_q - b^{is}_q b^{jk}_s \:, \label{eq:Mochov1D-F}\\
        \pder{}{u^r} \left[ g^{si} (b^{j k }_{s,q} - b^{j k}_{q,s}) + b^{ij}_s b^{s k}_q - b^{i k}_s b^{s j}_q \right] 
        \nonumber \\+ \pder{}{u^q} \left[ g^{si } (b^{j k}_{s,k} - b^{j k}_{r,s}) + b^{ij \beta}_s b^{s k}_{s,r} + b^{ij}_s b^{s k}_r - b^{ik}_s b^{kj}_s \right] \label{eq:Mochov1D-G}
        \\+ \sum_{(i,j,k)} \left[ b^{si}_q (b^{j k}_{r,s} - b^{j k}_{s,r}) + b^{s i}_r (b^{j k}_{q,s} - b^{j k}_{s,q}) \right] \nonumber
        = 0 \:.
    \end{align}
    It is evident that \eqref{eq:Mochov1D-D} is just \eqref{eq:Mochov1D-C} with a cyclic permutation hence it holds trivially. It is not so obvious, but also true, that \eqref{eq:Mochov1D-F} follows from the compatibility condition, zero torsion and zero curvature. 
     The last relation \eqref{eq:Mochov1D-G} is the one remaining Grinberg condition \eqref{eq:Grinberg-5}. Only if we assume non-degeneracy, it follows from the zero torsion and curvature, otherwise it must be involved as a separate assumption.
\end{remark}


\section{Tensor obstructions}

\begin{definition}[The obstruction tensors]
    We define the obstruction tensors
    \begin{align}
        T^{i \alpha \beta}_{jk} (u) \coloneqq \Gamma^{i \beta}_{jk}(u) - \Gamma^{i \alpha}_{jk}(u) \:,
    \end{align}
    for each pair of distinct indices $\alpha$ and $\beta$.
\end{definition}
We shall also use their fully contravariant form
\begin{align}
    T^{ijk \alpha \beta}(u) \coloneqq g^{i r \alpha}(u) g^{ks \beta}(u) T^{j \alpha \beta}_{rs} (u) \:.
\end{align}
Note that $T^{i \alpha \beta}_{jk} (u)$ are transformed as tensors only in the upper $\alpha$, $\beta$ indices (since in the remaining they transform as Christoffel symbols).

\begin{lemma} \label{lemma:T=0}
    Non-degenerate Dubrovin-Novikov brackets can be reduced to a constant bracket by a local change of coordinates if and only if all the obstruction tensors $T^{ij \alpha}_{jk}(u)$ are identically equal to zero.
\end{lemma}
\begin{proof}
    For any constant bracket all the connections $\Gamma^{i\alpha}_{jk}(u)$ are equal to zero identically, therefore all obstruction tensors are zero. Conversly, if all the $T^{ij \alpha}_{jk}(u)$ are identically zero, then all the connections $\Gamma^{i\alpha}_{jk}(u)$ are equal to each other. Consequently, they are all zero in any flat coordinates of the metric $g^{ij 1}(u)$. All the remaining metrics $g^{ij \beta}(u)$ are necessarily constant in these coordinates by virtue of compatiblity of the metrics with the corresponding connections.
\end{proof}


\begin{theorem}[Mokhov]
    Flat non-degenerate metrics $g^{ij \alpha}$ define a multidimensional Poisson bracket if and only if the following relations are fulfilled:
    \begin{align}
        T^{ijk \alpha \beta}(u) =& T^{kji \alpha \beta}(u) \:, \label{eq:Mochov-1}\\
        \sum_{(i,j,k)} T^{ijk \alpha \beta}(u) =& 0 \:, \label{eq:Mochov-2}\\
        T^{ij s \alpha \beta}(u) T^{k \alpha \beta}_{st}(u) =& T^{i k s \alpha \beta}(u) T^{j \alpha \beta}_{st} (u) \:, \label{eq:Mochov-3}\\
        \nabla^\alpha_r T^{ijk \alpha \beta}(u) =& 0 \:, \label{eq:Mochov-4}
    \end{align}
    where $\nabla^\alpha_k$ is the Levi-Civita covariant derivative generated by the metric $g^{ij \alpha}(u)$ and $\Gamma^{i \alpha}_{jk}(u)$ are the corresponding Christoffel symbols. 
\end{theorem}

\begin{remark}
    The relations \eqref{eq:Mochov-1} and \eqref{eq:Mochov-3} were found by Dubrovin and Novikov in \cite{Dubrovin-Novikov}. In \cite{Mochov} it was proved that this set of tensor relations is not complete and the remaining two were obtained.
\end{remark}

\begin{proof}
    The first equation \eqref{eq:Mochov-1} reads as
    \begin{align}
        g^{ir \alpha} g^{ks \beta} (\Gamma^{j\beta}_{rs} - \Gamma^{j \alpha}_{rs}) = g^{kr \alpha} g^{is \beta} (\Gamma^{j\beta}_{rs} - \Gamma^{j \alpha}_{rs}) \:,
    \end{align}
    which is equivalent to
    \begin{align}
        \Gamma^{j \beta}_{rs} - \Gamma^{j \alpha}_{rs} = 0 \:,
    \end{align}
    thus follows from \eqref{eq:Mochov-C}.

    The equation \eqref{eq:Mochov-2} is equivalent to \eqref{eq:Mochov-D}, \eqref{eq:Mochov-3} is equivalent to \eqref{eq:Mochov-E}, \eqref{eq:Mochov-4} is equivalent to \eqref{eq:Mochov-F}. The last condition \eqref{eq:Mochov-G} is a direct consequence of the relations \eqref{eq:Mochov-A}-\eqref{eq:Mochov-F} in the non-degenerate case.
    \todo[inline]{Je potřeba to rozepisovat podrobněji?}
\end{proof}

\begin{example}[Classification of one-component brackets]
    

It is not complicated to obtain a complete classification of all one-component brackets
\begin{align}
    \pder{u}{t} = \left( g^{\alpha}(u) \pder{^2h}{u^2} + b^{\alpha}(u) \pder{h}{u} \right) \pder{u}{x^\alpha} \:, \quad \alpha = 1, \cdots, d \:.
\end{align}
The multi-dimensional bracket is 
\begin{align}
    \set{u(x),u(y)} = g^{\alpha}(u) \partial_\alpha \delta(x-y) + b^\alpha(u) \delta(x-y) \:.
\end{align}
and the decomposition is
\begin{align}
    b^\alpha = - g^\alpha \Gamma^\alpha \:.
\end{align}
The Mokhov conditions \eqref{eq:Mochov-A}, \eqref{eq:Mochov-C}, \eqref{eq:Mochov-E} and \eqref{eq:Mochov-G} are automatically fulfilled. The relations \eqref{eq:Mochov-B} and \eqref{eq:Mochov-D} are
\begin{align}
    \pder{g^\alpha}{u} = 2 b^\alpha(u) \:, \label{eq:1comp-1} \\
    g^\alpha(u) b^\beta(u) = g^\beta(u) g^\alpha(u) \label{eq:1comp-2}\:.
\end{align}
The last condition \eqref{eq:Mochov-F} follows from them. If we apply \eqref{eq:1comp-1} into \eqref{eq:1comp-2}, we get
\begin{align}
    \frac{1}{g^\alpha} \pder{g^\alpha}{u}= \frac{1}{g^\beta}\pder{g^\beta}{u} \:,
\end{align}
hence
\begin{align}
    g^\alpha(u) = 2c^\alpha g(u) \:, \quad b^\alpha(u) = c^\alpha \quad \forall \alpha = 1, \cdots d \:,
\end{align}
where $c^\alpha$ are arbitrary constants.
Thus, for $g(u) \neq 0$ all the obstruction tensors are identically zero:
\begin{align}
    T^{\alpha \beta} = \Gamma^\beta - \Gamma^\alpha = -\frac{b^\alpha}{g^\alpha} + \frac{b^\beta}{g^\beta} = - \frac{1}{2 g} + \frac{1}{2g} = 0 \:.
\end{align}
By the virtue of lemma \vref{lemma:T=0} by local changing of coordinates we can always find a transform in witch the bracket has a constant form. The 
\begin{align}
    \pder{w}{t} = \eta^\alpha \pder{w}{x^\alpha} \:.
\end{align}

The degenerate case is also trivial. If for some $\alpha$ is $g^\alpha(u) = 0$, then from \eqref{eq:Mochov-B} follows $b^\alpha(u) = 0$.

\end{example}

\section{Compatible brackets}

In this section we introduce the theory of compatible metrics constructed by Mochov in \cite{Mochov} and show that the study of non-degenerate multi-dimensional brackets is connected to the study of compatible one-dimensional brackets. 

\begin{definition}[Compatible metrics]
    Let $g^{ij1}(u)$ and $g^{ij2}(u)$ be non-degenerate psudo-Riemannian metrics with Levi-Civita connections $\Gamma^{ij1}_k(u)$, $\Gamma^{ij2}_k(u)$. We say that $g^{ij1}$ and $g^{ij2}$ are compatible if for any linear combination of these metrics
    \begin{align}
        g^{ij}(u) = \lambda_1 g^{ij1}(u) + \lambda_2 g^{ij2}(u) \:,
    \end{align}
    where $\lambda_{1,2}$ are arbitrary constant such that $\det g^{ij} \neq 0$, the coefficients of the corresponding Levi-Civita connections and the components of the corresponding Riemann tensors are related by the same linear formula:
    \begin{align}
        \Gamma^{ij}_k = \lambda_1 \Gamma^{ij1}_k + \lambda_2 \Gamma^{ij2}_k \:, \label{eq:compatible-Gamma}\\
        R\indices{_a_b^k_c} = \lambda_1 R\indices{_a_b^k_c^1} + \lambda_2 R\indices{_a_b^k_c^2} \label{eq:compatible-R}\:.
    \end{align}
    If for any linear combination of metrics only \eqref{eq:compatible-Gamma} is fulfilled, then the metrics are called almost compatible.
\end{definition}


\begin{definition}[Compatible Poisson brackets]
    We say that two Poisson brackets are compatible if each their linear combination is a Poisson bracket.
\end{definition}

\begin{definition}[L-affinor]
    Let $g^{ij1}(u)$ and $g^{ij2}(u)$ be two metrics.
    We define the L-affinor $L^i_j$ by
    \begin{align}
        L^i_j(u) = g^{ik2}(u) g_{kj1}(u) \:,
    \end{align}
\end{definition}

\begin{theorem}[Vanishing of the Nijenhuis tensor]
    The metrics $g^{ij1}(u)$ and $g^{ij2}(u)$ are almost compatible if and only if the Nijenhuis tensor of the L-affinor is zero, i.e.
    \begin{align}
        N^i_{jk} \coloneqq L^s_j L^{i}_{k,s} - L^s_k L^{i}_{j,s} - L^i_s L^s_{k,j} - L^i_s L^s_{j,k} = 0 \:.
    \end{align}
\end{theorem}

Note that that there are examples of metrics that are only almost compatible, but not compatible, and for which the Nijenhuis tensor is zero \cite{Mokhov-compatibility}. 

\begin{theorem}[Mutual compatibility of the metrics]
    All metrics $g^{ij \alpha}(u)$, $1 \leq \alpha \leq N$, defining a multi-dimensional Poisson bracket \eqref{eq:multi-dimensional bracket} are mutually compatible. All one-dimensional Dubrovin-Novikov brackets forming a multi-dimensional bracket are mutually compatible.
\end{theorem}
\todo[inline]{Podívat se na důkaz tohoto.}
\begin{proof}
    The equation \eqref{eq:Mochov-1} is equivalent to \eqref{eq:Mochov-C} and also to the vanishing of the obstruction tensor, witch means
    \begin{align}
        \Gamma^{i\alpha}_{jk} = \Gamma^{i \beta}_{jk} \:.
    \end{align}
    Hence their linear combination is also Levi-Civita connection for the same linear combination of metrics.
    The equation \eqref{eq:Mochov-3} is equivalent to \eqref{eq:Mochov-E} and thus to
    \begin{align}
        R\indices{_a_b^k_l^\alpha} =  R\indices{_a_b^k_l^\beta} \:.
    \end{align}
\end{proof}

\begin{theorem}[Mokhov classification theorem]
    If for a nondegenerate multidimensional Poisson bracket one of the metrics $g^{ij \alpha}(u)$ forms nonsingular pairs with all the remaining metrics, i.e.
    \begin{align}
        \:,
    \end{align}
    then this Poisson bracket can be reduced to a constant form by a local change of coordinates.
\end{theorem}

\begin{proof}
    \todo{inline}[podívat se na důkaz]
\end{proof}


\chapter*{Conclusion}
\addcontentsline{toc}{chapter}{Conclusion}

The area of geometry in hydrodynamics offers many challenging, unsolved problems. This thesis aimed to explore the Dubrovin-Novikov theorem which shows that for non-degenerate metric the existence of Poisson structure is equivalent to the existence of Levi-Civita connection with zero curvature, therefore connects the Poisson and pseudo-Riemannian geometry. The proof of this theorem in the original paper \cite{Dubrovin-Novikov} was only sketched, so the potential benefit of this thesis was to provide it in perhaps a readable, detailed form. In Chapter 3, we have first verified the tensor structure of the functions in quasi-linear systems of the first order and constructed the Dubrovin-Novikov bracket for systems of hydrodynamic type. Then we restricted ourselves to the one-dimensional case, where we have shown that the functions forming the non-degenerate bracket are in fact inverse metric and Christoffel symbols. Then the proof of the Dubrovin-Novikov theorem was divided into four steps, in which we have 1. connected the skew-symmetry of the bracket with the symmetry of the metric and the compatibility condition, 2. written down the conditions under which the Jacobi identity holds, 3. verified the zero torsion and curvature and 4. indicated the existence of the coordinates in which the bracket is trivial and therefore finished the converse statement. In Chapter 4, we have done a similar analysis for the multi-dimensional bracket, written down the conditions under which the bracket is Poisson, and reformulated the problem of finding the flat coordinates via the obstruction tensors. We end the discussion with the theorem by Mokhov, which states that the existence of flat coordinates is possible if one of the metrics forms nonsingular pairs with all the remaining metrics. We did not explore whether this condition is satisfied in the examples mentioned in Chapter 2.


Nevertheless, the connection between the two geometric structures is still slightly mysterious, as the zero torsion and zero curvature emerge among a huge amount of calculations. Is there something deeper to be identified beyond these equations? 

The assumption which is found to be crucial is the non-degeneracy of a metric. We are convinced that there exist attempts to describe the geometry with degenerate metrics. The problem with hydrodynamical systems is that the functions $g^{ij \alpha}$ are in fact the inverse metrics; therefore we need a theory which stems from the symmetric but degenerate (2,0) tensor. If the metric is degenerate, there is no Levi-Civita connection. There could be several covariant derivatives that annihilate the metric and have zero torsion. Could this be the way of proving the symmetric hyperbolicity for the degenerate systems? At least to the author's knowledge, these questions are still open to inquiry. 

To the inquisitive reader, we further recommend two sources that widen the horizons of current knowledge. For an analysis of two-dimensional systems, see \cite{Savoldi}, and for more of the one-dimensional degenerate brackets, see \cite{savoldi2014deformations}. 


%%% Bibliography
\include{bibliography}

%%% Figures used in the thesis (consider if this is needed)
%\listoffigures

%%% Tables used in the thesis (consider if this is needed)
%%% In mathematical theses, it could be better to move the list of tables to the beginning of the thesis.
% \listoftables

%%% Abbreviations used in the thesis, if any, including their explanation
%%% In mathematical theses, it could be better to move the list of abbreviations to the beginning of the thesis.
% \chapwithtoc{List of Abbreviations}

%%% Attachments to the bachelor thesis, if any. Each attachment must be
%%% referred to at least once from the text of the thesis. Attachments
%%% are numbered.
%%%
%%% The printed version should preferably contain attachments, which can be
%%% read (additional tables and charts, supplementary text, examples of
%%% program output, etc.). The electronic version is more suited for attachments
%%% which will likely be used in an electronic form rather than read (program
%%% source code, data files, interactive charts, etc.). Electronic attachments
%%% should be uploaded to SIS and optionally also included in the thesis on a~CD/DVD.
%%% Allowed file formats are specified in provision of the rector no. 72/2017.
\appendix
\chapter{"Mathematical framework"}
\todo[inline]{vymyslet lepší název kapitoly}

In this chapter we shall revise some basic definitions and propositions of the functional calculus, which we will need for the proper formulation of the hydrodynamic geometry. Most of the existing literature on this topic often lacks clarity due to the omittance of certain details. This can include the domains on which functionals are defined or regularity conditions.

Over the last century, modern mathematics has worked to develop the theory of generalized functions - distributions, including Dirac's delta distributions and operations with them, such as in [CITATION]. The same applies to functional variations and the conditions on which they exist. 
Among physists, however, it is convential to operate with delta distributions as they would with "ordinary" functions, using dualities as integrals and functional variations as ordinary derivatives.


After introducing the concept of Gateaux and Frechet derivatives, which are suitable for arbitrary normed space, we shall take a closer look on the space of smooth functions and functionals on it.

\section{Functional derivatives}

\todo[inline]{přidat nějaký kontext}

\begin{definition}[Gateaux and Frechet derivative]
    Let $X$ be a normed space, $A: X \to \R$ is a functional and $x \in D(A)$.
    \begin{enumerate}
        \item Gateaux derivative $\delta A(x,h)$ in the direction $h$ is defined as
        \begin{align}
            \delta A (x,h) = \der{}{t} A(x+th) |_{t=0} = \lim_{t \rightarrow 0} \frac{A(x+th)-A(x)}{t} \:.
        \end{align}

        \item Fréchet derivative $DA(x)$ is defined as the unique linear bounded functional satisfying the relation
        \begin{align}
            \lim_{h \rightarrow 0} \frac{A(x+h)-A(x)- DA(h)}{\norm{h}} = 0 \:.
        \end{align}
    \end{enumerate}
\end{definition}

We are intersted in such functionals, where $X$ is the subspace of functions $ f \in C^k(\Omega)$, $\Omega$ is a subset of $\R^n$, satisfying $f,f',\cdots,f^{(k-1)} = 0$ on $\partial \Omega$, equipped with the norm
\begin{align}
    \norm{u}_X = \sum_{j=1}^n \sup_{x \in \Omega} |u^{(j)}(x)| \:.
\end{align}
There are two main types of functionals on that space. 

\subsection{Integral functionals}

The first type is the integral functional
\begin{align}
    A[u] = \int_\Omega a(x,u(x),u'(x),\cdots,u^{(k)}(x)) \, \D^n x \:, \label{eq:integral-functional}
\end{align}
where $a : \R \times \R^{k} \to \R$ is an analytic function.
For simplicity we often write this using a physical notation
\begin{align}
    A[u] = \int a(u) \, \D^n x \:,
\end{align}
meaning that $a(u)$ is in fact function $a(x,u(x),u'(x),\cdots, u^{(k)}(x))$ and also the domain $\Omega$ is specified by the context. We call $a(u)$ the density of functional $A[u]$.

Gateaux derivative of $A$ is computed as follows:
\begin{align}
    \delta A[u,h] = \int_\Omega \sum_{i=1}^k \pder{a}{u^{(i)}}(x) h^{(i)}(x) \, \D^n x \:. \label{eq:Gateaux}
\end{align}
The functional $\delta A[u,h]$ is linear and bounded in $h$, so the Fréchet derivative of $A$ exists and is given by
\begin{align}
    DA[u] \cdot h = \delta A[u,h] \:.
\end{align}
To get rid of the derivatives of $h(x)$ in \eqref{eq:Gateaux}, we use integration by parts in each term of the sum, obtaining
\begin{align}
    DA[u] \cdot h = \int_\Omega \left[ \pder{a}{u}(x) - \der{}{x} \pder{a}{u'}(x) + \der{^2}{x^2} \pder{a}{u''}(x) - \cdots + (-1)^k \der{^k}{x^k} \pder{a}{u^{(k)}}(x) \right] h(x) \, \D^n x \:. \label{eq:perpartes}
\end{align}

It is natural to give a special name to the integrand in \eqref{eq:perpartes}.

\begin{definition}[Functional derivative]
    Let $A$ be the integral functional given by \eqref{eq:integral-functional}. The functional derivative o $A$ is the function
    \begin{align}
        \fder{A}{u(x)} = \pder{a}{u}(x) - \der{}{x} \pder{a}{u'}(x) + \der{^2}{x^2} \pder{a}{u''} - \cdots + (-1)^n \der{^k}{x^k} \pder{a}{u^{(k)}}(x) \:.
    \end{align}
\end{definition}

The expression \eqref{eq:perpartes} is very similar to the scalar product on $L^2(\Omega)$ given by
\begin{align}
    \dual{f}{g}_{L^2(\Omega)} = \int_\Omega f(x) g(x) \, \D x \:.
\end{align}
Thus, it makes sense to write
\begin{align}
    DA[u] \cdot h = \int_\Omega \fder{A}{u(x)} h(x) \, \D x = \dual{\fder{A}{u(x)}}{h(x)} \:.
\end{align}



It follows from theory (CITATION) that the functional derivative beeing zero is a necessery condition for the function beeing the extremal point of the functional (given that the corresponding function is smooth enough).

\begin{theorem}[Euler-Lagrange equations]
    Let $v \in C^\infty(\Omega)$ such that $v(x)$ is stationary point of the functional $A$ in \eqref{eq:integral-functional}, i.e
    \begin{align}
        \delta A[v,h] = 0 \:.
    \end{align}
    Then  
    \begin{align}
        \fder{A}{v(x)} = 0 \:.
    \end{align}
\end{theorem}

The computation above gives sense to the 
\begin{align}
    A[u+h] = A[u] + \dual{\fder{A}{u}}{h} + o(h)\:,
\end{align}
where
\begin{align}
    \lim_{\norm{h} \rightarrow 0} \frac{o(h)}{\norm{h}} = 0 \:.
\end{align}
In physical notation, the function $h$ is often denoted as $\delta u$.


\subsection{Distributions}

\begin{definition}[Distributions]
    Let $\test(\Omega)$ be the space of functions with compact support in $\Omega \subseteq \R^n$. The space of distributions $\dis(\Omega)$ is the dual space, i.e. the space of bounded linear functionals on $\test(\Omega)$. An element $T \in \dis(\Omega)$ is called a distribution.
\end{definition}

\begin{definition}[Regular distribution]
    Let $a \in L^1_{\text{loc}}(\Omega)$. A regular distribution is a distribution $T_a \in \dis(\Omega)$ defined as
    \begin{align}
        \dual{T_a}{\phi(x)} = \int_\Omega a(x) \phi(x) \D^n x \:.
    \end{align}
\end{definition}

Note that if two regular distributions satisfy $T_a = T_b$, then $a=b$ almost everywhere and vice versa.

\begin{definition}[Dirac delta distribution]
    Let $y \in \Omega$. The Dirac delta distribution $\delta(x-y)$ is defined as
    \begin{align}
        \dual{\delta(x-y)}{\phi(x)} = \phi(y) \:. 
    \end{align}
\end{definition}


\todo[inline]{Zmínit se o derivaci distribucí a o tom, že se diraci dají testovat klidně $C^1$ funkcemi, takže souvisejí s teorií v minulé kapitole a můžeme aplikovat teorii funkcionálních derivací i na distribuce.}

\begin{definition}[Derivatives of distributions]
    Let $T \in \dis(\Omega)$ be a distribution. We define its derivative by
    \begin{align}
        \dual{D^\alpha T}{\phi} = (-1)^{|\alpha|} \dual{T}{D^\alpha \phi} q \:,
    \end{align}
    where $\alpha$ is a multiindex and
    \begin{align}
        D^\alpha f = \frac{\partial^{|\alpha|} f}{\partial x_1^{\alpha_1} \partial x_2^{\alpha_2} \cdots \partial x^{\alpha_N}_N} \:.
    \end{align}
\end{definition}



Let $a \in C^\infty(\Omega)$. Then it makes sense to define the product of $a$ and arbitrary distribution $T$ by
\begin{align}
    \dual{a T}{\phi} = \dual{T}{a \phi} \:.
\end{align}

In the next chapter we shall frequently use the following identities of the delta distributions, which are written in the next lemma and which we will refer to as the Dirac identities.

\begin{lemma}[Dirac identities] \label{lemma:delta}
    \begin{align}
        f(y) \delta (x-y) =& f(x) \delta(x-y) \:, \\
        f(y) \delta'(x-y) =& f(x) \delta'(x-y) + f'(x) \delta(x-y) \:.
    \end{align}
\end{lemma}
\begin{proof}
    The first relation is trivial.
    Direct calculation gives
    \begin{align*}
        \dual{f(x) \delta'(x-y)}{\psi(x)} 
        =& \dual{\delta'(x-y)}{f(x) \psi(x)} 
         \\ =& -\dual{\delta(x-y)}{f'(x)\psi(x)} - \dual{\delta(x-y)}{f(x)\psi'(x)} 
         \\ =& -f'(y) \psi(y) - f(y) \dual{\delta(x-y)}{\psi'(x)} 
         \\ =& - \dual{f'(x)  \delta(x-y)}{\psi(x)} + f(y) \dual{\delta'(x-y)}{\psi(x)} 
         \\ =& \dual{f(y) \delta(x-y) - f'(x) \delta'(x-y)}{\psi(x)} \:.
    \end{align*}
    Therefore
    \begin{align}
        \dual{f(y) \delta'(x-y)}{\psi(x)} = \dual{f'(x) \delta(x-y)}{\psi(x)} + \dual{f(x) \delta'(x-y)}{\psi(x)} \:.
    \end{align} 
\end{proof}


\todo[inline]{Vysvětlit, jak by se dal matematicky přesněji odvodit vztah $ A(u(x)) = \int A(u(\xi)) \delta (x-\xi) \, \D \xi$ pomocí konvoluce distribucí.}
The function $a(u)$ can be thus identified with a functional $A[u]$ defined by the relation
\begin{align}
    \dual{A[u(x)]}{\phi(x)}  = \dual{a(u(\xi)) \delta(x-\xi)}{\phi(\xi)}
\end{align}
In the physical notation
\begin{align}
    A[u(x)] = \int a(u(\xi)) \delta(x-\xi) \, \D^n \xi \:.
\end{align}
Therefore it is also common not to differentiate between the function itself and its corresponding functional form and denote them by the same letter $A$:
\begin{align}
    A[u(x)] = \int A(u(\xi)) \delta(x-\xi) \, \D^n \xi \:.
\end{align}



\begin{example}
    Let $u(x)$ be a field variable and $F[u(x)]$ a function(al). 
    To compute its functional derivative w.r.t. $u(y)$ in different position $y$, we use the integral expression of $F$
    \begin{align}
        F[u(x)] = \int \D y F(u(y)) \delta (x-y) \:.
    \end{align}
    Its Gateaux diferential is
    \begin{align}
        \delta F[u,h] = \der{}{t} F[u+th]|_{t=0} = \int \D y \pder{F}{u}(y) \delta(x-y) h(y) \:,
    \end{align}
    therefore
    \begin{align}
        \fder{F(u(x))}{u(y)} = \pder{F}{u}(x) \delta(x-y) \:.
    \end{align}
\end{example}

\begin{example}
    In the special case of performing functional derivative w.r.t. the same field but in the different position,
    \begin{align}
        \fder{u^i(x)}{u^j(y)} = \delta^i_j \delta(x-y) \:.   
    \end{align}
\end{example}

\begin{example}
    \begin{align}
        \set{I,J} = \int_\Omega \D x' \int_\Omega \D y' \fder{I}{u^i(x)} P^{ij}(x',y') \fder{J}{u^j(y)} \:,
    \end{align}
    where
    \begin{align}
        I = I[u], \: J = J[u] \:, \quad P^{ij}(x',y') = \alpha^{ij}(x',y') \der{}{x'} \:.
    \end{align}
    Then
    \begin{align}
        \set{u^a(x),u^b y)} = \int_\Omega \D x' \int_\Omega \D y' \delta^a_i \delta(x-x') \alpha^{ij}(x',y') \delta^b_j \delta(y-y') = \alpha^{ab}(x,y) \:.
    \end{align}
\end{example}


\section{Geometrical aspects}
\subsection{Riemannian manifolds and covariant derivative}
\subsection{Lie algebras and Poisson brackets}


\openright
\end{document}
